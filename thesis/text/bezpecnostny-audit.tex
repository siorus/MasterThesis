\chapter{Bezpečnostný audit}
\phantomsection
\addcontentsline{toc}{chapter}{Bezpečnostný audit}

\section{Manažment rizík}
Manažment rizík je proces pozostávajúci z analýzy rizík a riadenia rizík\cite{McMillan2018}. Dôležitým faktom je, že riziko nie je možné eliminovať, ale ho iba znížiť.

Pri analýze rizík zisťujeme, aké riziká existujú, ako medzi sebou súvisia a aké škody môžu spôsobiť. Analýza rizík môže byť vykonávaná kvalitatívne a kvantitatívne.\\ 
\noindent
Štandard NIST SP 800-30\cite{7TVhmfuQFbsOANAz} definuje nasledujúce kroky pri analýze rizík:

\begin{enumerate}
	\item Identifikácia informačných aktív a ich význam
	\item Identifikácia hrozieb
	\item Identifikácia zraniteľností
	\item Zistenie pravdepodobnosti
	\item Identifikovanie dopadu
	\item Definovanie rizika ako súčinu pravdepodobnosti a dopadu
\end{enumerate} 
\vspace{2em}
Riadenie rizík má za úlohu minimalizáciu potenciálnych škôd odhalených pri analýze rizík s ohľadom na vyváženie nákladov na riadenie rizika. 
\\\\
\noindent
Prístupy k nájdenému riziku\cite{Vyncke2008}\cite{McMillan2018}:
\begin{itemize}
	\item Zníženie\,--\,aplikovanie protiopatrenia na odstránenie hrozby alebo zraniteľnosti prípadne zníženie pravdepodobnosti rizika. Nikdy nie je však možné riziko eliminovať.
	
	\item Akceptovanie\,--\,v prípade neexistujúceho protiopatrenia alebo veľmi nízkeho rizika
	
	\item Presun\,--\,riziko je možné presunúť na inú organizáciu, napr. poistenie v prípade škody spôsobenej nedostatočným zabezpečením.
	
	\item Ignorácia\,--\,úplné vypustenie faktu, že dochádza k riziku, tento prístup sa považuje za iracionálny.
\end{itemize}
\vspace{2em}
\noindent
Na ohodnotenie rizika slúžia rôzne systémy hodnotenia, jedným z nich je  \zkratka{zkCVSS}, ktorý definuje riziká podľa definovaných metrík na základe dosiahnutého skóre do nasledujúcich tried:

\begin{itemize}
	\item 0: No issue
	\item 0,1\,--\,3,9: Low
	\item 4,0\,--\,6,9: Medium
	\item 7,0\,--\,8,9: High
	\item 9,0\,--\,10,0: Critical
\end{itemize} 