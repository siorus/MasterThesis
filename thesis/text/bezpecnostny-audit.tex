\chapter{Bezpečnostný audit}
\label{bezpecnostny-audit}
\phantomsection

Auditovanie je veľmi dôležitým prvkom správy informačných systémov a infraštruktúry, pretože umožňuje zaistiť bezpečnosť týchto informačných aktív porovnávaním s vytvorenými štandardmi, odporúčaniami a predpismi. Zaoberá sa otázkami čo a ako zabezpečiť, vyhodnocovaním a riadením rizík a následným dokazovaním, že náprava znížila riziko hrozby.
\\\\
\noindent
Auditovanie sa skladá z piatich pilierov \cite{Jackson2010}:

\begin{enumerate}
	\item Posúdenie
	\item Prevencia
	\item Detekcia
	\item Reakcia
	\item Zotavenie
\end{enumerate}

\vspace{1em}
\noindent
Pri posudzovaní si je potreba klásť otázky či sú prístupové práva dostatočne špecifikované, aká je pravdepodobnosť útoku na zraniteľnosť a podobne. Prevencia nespočíva iba v technológiách ako firewall prípadne \zk{zkIDS} a \zk{zkIPS}, ale aj v politikách, procesoch a povedomí o probléme. Detekcia a reakcia spolu úzko súvisia a je potrebné skrátiť dobu medzi týmito dvoma bodmi, bez dôkladnej detekcie nie je možné vykonať reakciu.	Mnohé reakcie na detekciu problému sú už rôznymi technológiami implementované automatizovane. Posledný článkom je zotavenie, ktoré je dôležité pri službách vysokej dostupnosti. Výborným príkladom detekcie, reakcie a zotavenia z problému sú protokoly z rodiny \zkratka{zkFHRP}.

\vspace{1em}
\noindent
Proces auditu pozostáva z niekoľkých fází: \cite{Jackson2010} 
\begin{enumerate}
	\item Plánovanie\,--\,stanovenie cieľov a predmetu auditu. Definuje sa rozsah, teda čo všetko je v pláne auditom pokryť. 
	\item Výskum\,--\,vytváranie auditného plánu na základe štandardov a odporúčaní a špeciálnych expertíz. Kontaktujú sa tiež dotknuté strany, ktoré nám môžu byť nápomocné pri plnení cieľov.
	\item Zbieranie dát\,--\,vyžiadanie potrebných podkladov a dát na vykonanie auditu, zozbieranie dôkazov. V tejto fáze sa tiež vyberajú rôzne softvérové nástroje na vykonanie auditu a vytvorí sa checklist na základe auditného plánu a zozbieraných dôkazov.  
	\item Analýza dát\,--\,posúdenie všetkých dôkazových dát pomocou checklistu a softvéru na podporu auditu. Na základe nájdených nedostatkov sa vytvoria odporúčania, ktoré by mali znížiť riziká hrozieb.
	\item Vytváranie správy\,--\,súpis nájdených nedostatkov, možných riešení na zníženie rizík do auditnej správy a prezentácia tejto správy dotknutým stranám.
	\item Aplikácia opatrení\,--\,nasadenie a použitie protiopatrení prezentovaných alebo vyplývajúcich z auditnej správy. Následne sa môže vykonať monitorovanie a hlásenie o úspešnosti zmien.
\end{enumerate}

\vspace{1em}
\noindent
Typy auditov podľa zistení, hĺbky a rozsahu auditu:
\begin{itemize}
	\item Bezpečnostná kontrola\,--\,je najzákladnejšia forma analýzy bezpečnosti, na základe ktorej sa následne formujú ďalšie aktivity na zaistenie bezpečnosti. Do tejto kategórie spadajú automatizované nástroje na skenovanie zraniteľností a penetračné nástroje, ktoré generujú zoznam potenciálnych zraniteľností, ale je potrebné ďalšie podrobnejšie preskúmanie výsledkov a zistení a stanovenie, ako sa k ním zachovať. Patria sem nástroje ako napríklad Nmap, Nessus a podobne. Za bezpečnostnú kontrolu možno považovať preskúmanie politík alebo architektúry daného systému a infraštruktúry. Dá sa povedať, že ide o akýsi rýchly náhľad na bezpečnosť, ktorého výstupom je poznanie a identifikovanie problému.
	\item Hodnotenie bezpečnosti\,--\,je ďalším stupňom, pričom ide o podrobnejší pohľad na problém z profesionálnejšieho hľadiska. Kvalifikuje sa riziko k jednotlivým zisteniam a stanovuje sa relevantnosť a kritickosť týchto zistení na konkrétnu organizáciu a prípad použitia.
	\item Bezpečnostný audit\,--\,je štandardizovanou a najdôkladnejšou formou posúdenia bezpečnosti. Bezpečnosť sa porovnáva so štandardmi alebo benchmark-mi, v niektorých prípadoch aj s predpismi dohliadahúcich orgánov. Výsledkom je posúdenie, na koľko je organizácia alebo skúmaný objekt v zhode s porovnávaným štandardom. Typickým príkladom štandardov sú ISO27001 a COBIT.
\end{itemize}

\section{Manažment rizík}
\label{riskmanagement}
Manažment rizík je proces pozostávajúci z analýzy rizík a riadenia rizík \cite{McMillan2018}. Dôležitým faktom je, že riziko nie je možné eliminovať, ale ho iba znížiť.

Pri analýze rizík zisťujeme, aké riziká existujú, ako medzi sebou súvisia a aké škody môžu spôsobiť. Analýza rizík môže byť vykonávaná kvalitatívne a kvantitatívne.\\ 
\newpage
\noindent
Štandard NIST SP 800-30 \cite{7TVhmfuQFbsOANAz} definuje nasledujúce kroky pri analýze rizík:

\begin{enumerate}
	\item Identifikácia informačných aktív a ich význam
	\item Identifikácia hrozieb
	\item Identifikácia zraniteľností
	\item Analýza riadenia a kontroly 
	\item Zistenie pravdepodobnosti
	\item Identifikovanie dopadu
	\item Definovanie rizika ako súčinu pravdepodobnosti a dopadu
	\item Odporúčanie na zavedenie riadenia a kontroly na zníženie rizika  
	\item Zdokumentovanie výsledkov
\end{enumerate} 
\vspace{2em}
Riadenie rizík má za úlohu minimalizáciu potenciálnych škôd odhalených pri analýze rizík s ohľadom na vyváženie nákladov na riadenie rizika. 
\\\\
\noindent
Prístupy k nájdenému riziku \cite{Vyncke2008}\cite{McMillan2018}\cite{Jackson2010}:
\begin{itemize}
	\item Vyhnutie sa riziku\,--\,je uplatnené ak prítomnosť a funkčnosť informačného aktíva nestojí za podstúpenie rizika, a teda toto aktívum vôbec nepoužijeme. Napríklad vypnutie menej bezpečných a nevyužívaných sieťových služieb.  
	
	\item Zníženie\,--\,aplikovanie protiopatrenia na odstránenie hrozby alebo zraniteľnosti prípadne zníženie pravdepodobnosti rizika. Nikdy nie je však možné riziko eliminovať. Príkladom môže byť obmedzenie prístupu k sieťovému prvku.
	
	\item Akceptovanie\,--\,v prípade neexistujúceho protiopatrenia alebo veľmi nízkeho rizika. Častokrát ide o bezpečnostnú chybu softvéru v službe, ktorú využívame a nie je možné ju vypnúť ani aplikovať protiopatrenie.
	
	\item Presun\,--\,riziko je možné presunúť na inú organizáciu, napr. poistenie v prípade škody spôsobenej nedostatočným zabezpečením.
	
	\item Ignorácia\,--\,úplné vypustenie faktu, že dochádza k riziku, tento prístup sa považuje za iracionálny.
\end{itemize}
\vspace{2em}
\noindent
Na ohodnotenie rizika slúžia rôzne systémy hodnotenia, jedným z nich je  \zkratka{zkCVSS}, ktorý definuje riziká podľa definovaných metrík na základe dosiahnutého skóre do nasledujúcich tried:

\begin{itemize}
	\item 0: No issue
	\item 0,1\,--\,3,9: Low
	\item 4,0\,--\,6,9: Medium
	\item 7,0\,--\,8,9: High
	\item 9,0\,--\,10,0: Critical
\end{itemize} 