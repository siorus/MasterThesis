\chapter*{Záver}
\phantomsection
\addcontentsline{toc}{chapter}{Záver}

TODO \\
%Buducnost - oznacovanie flase possive + komenty\\
%Dorobit backend - server \\
TOTO ZO SEMESTRALKY\\

V~rámci semestrálnej práce bola naštudovaná problematika bezpečnosti a prevádzky sieťových zariadení a prišlo k~porovnaniu existujúcich riešení. Z~dostupnej literatúry a odborných textov boli vytvorené tabuľky s~odporúčaniami rešpektujúce topológiu, teda hierarchický model siete a tiež ohodnotenie závažnosti pri absencii odporúčaní. Bol vytvorený návrh aplikácie, princíp fungovania, navrhnutá štruktúra ukladania a práce s~konfiguráciami zariadení a ukladanie výsledkov. V~rámci implementácie kódu prišlo k~vytvoreniu konfiguračných súborov na ukladanie informácií zariadení a šablóny konfiguračného súboru popisujúceho modul zodpovedný za nález, vykonanie nápravy a záznam nájdených nedostatkov. Ako implementačný programovací jazyk bol vybraný jazyk Python a jazyk YAML pre konfigurácie potrebné pre moduly.

V~naväzujúcej diplomovej práci príde k~celkovej implementácii aplikácie a jej následnému otestovaniu na konfiguráciách z~virtuálnych zariadení ako aj na zariadeniach z~reálnej prevádzky. Bude vytvorená dokumentácia a API, ktoré budú umožňovať pridávanie modulov ďalšími prispievateľmi a tým rozšíriť použiteľnosť aplikácie na ďalších výrobcov. 
