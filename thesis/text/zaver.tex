\chapter*{Záver}
\phantomsection
\addcontentsline{toc}{chapter}{Záver}


%Buducnost - oznacovanie flase possive + komenty\\
%Dorobit backend - server \\
%co som mal spravit
%co som spravilk

% co sa nepodarili
% v skratke vyhody
% co bolo nutne nastudovat plus z kolko guidov som cerpal





Cieľom tejto diplomovej práce bol návrh a následná implementácia programu na nájdenie bezpečnostných a prevádzkových nedostatkov v sieťových zariadeniach, ako aj ich náprava pomocou generovania opravnej konfigurácie. Z tohto dôvodu bola naštudovaná problematika bezpečnosti a prevádzky sieťových zariadení a ich správna konfigurácia. Z množstva dostupnej literatúry, štandardov a odporúčaní bol vytvorený zoznam odporúčaní, na základe ktorého boli zostavované YAML moduly pre výsledný program. Tento zoznam odporúčaní bol rozšírený aj o hodnotenie závažnosti nedostatkov a priradenie prvkov zoznamu k relevantným zariadeniam v hierarchickom modely siete. Jedná sa teda o unikátne riešenie medzi bezplatnými zoznamami odporúčaní. Tento zoznam môže byť použitý aj na rozšírenie programu o podporu iných výrobcov, ale aj separátne bez akéhokoľvek využitia v programe. Vzhľadom na časovú náročnosť boli vytvorené moduly zatiaľ pre zariadenia značky Cisco. Bolo nutné zanalyzovať niekoľko stovák príkazov a ich povolených kombinácií, vytvoriť pre ne korešpondujúce regulárne výrazy a následne vytvoriť viac ako 230 YAML modulov zodpovedných za nájdenie problémov v konfiguráciách. 

Ako implementačný jazyk bol využitý Python 3.7, ktorý zaisťuje prenositeľnosť programu na viaceré platformy. Program na rozdiel od konkurencie umožňuje rozšírenie vďaka modularite aj na ďalších výrobcov sieťových zariadení a pridáva kontrolu aj pre topológie využívajúce IPv6. Taktiež rešpektuje hierarchický model siete, a teda generuje oveľa menej falošne pozitívnych správ, keďže kontroluje iba nastavenia typické pre danú vrstvu, na ktorej zariadenie operuje. Jeho výstupom je okrem iného aj prehľadná správa o kontrole zobraziteľná v PDF alebo vo webovom prehliadači. Naviac ako jediný z porovnávaných bezplatných riešení umožňuje automatické vygenerovanie nápravy, pokiaľ je to z charakteru príkazu možné. 

Program bol otestovaný pomocou vyexportovaných konfigurácií z viacerých testovacích topológií vytvorených v nástroji GNS3. Je však žiadúce aplikáciu otestovať komplexnejšie aj na reálnych a rozsiahlejších topológiách pre čo najlepšie vyladenie jej funkčnosti. 

Nástroj je možné vďaka modularite v budúcnosti rozšíriť aj o podporu na ďalších výrobcov ako Juniper a HP. Využiteľná by bola taktiež možnosť editovať záverečné správy pridaním back-endu pre HTML správy, teda označovaním falošne pozitívnych nálezov a komentovaním nálezov priamo vo webovom prehliadači. Tiež by bolo možné program rozšíriť o podporu notifikovania nedostatkov a chýb na základe známych hrozieb pre aktuálne bežiace verzie firmwérov.
