\chapter{Návrh}
\phantomsection

\section{Požiadavky na aplikáciu a existujúce riešenia}
%TODO api cisco verzie problemy
Kľúčovou vlastnosťou je modularita navrhovanej aplikácie, vďaka ktorej bude možné pridávať a definovať nové moduly na základe zmien v syntaxi a sémantike príkazov. Modularita taktiež umožňuje vytvorenie a podporu ďalších výrobcov a operačných systémov sieťových zariadení. Existujúce riešenia sú zväčša zamerané iba na jedného výrobcu a operačný systém, pričom program je jeden zdrojový súbor, ktorý bez dobrej znalosti kódu je problematické upraviť a rozšíriť. Preto jednotlivé overovania odporúčaní a ich následná oprava bude každé v separátnom module, ktorý budú musieť dodržať určité vstupy a výstupy, teda akési \zkratka{zkAPI}. Existujúce riešenia nedisponujú žiadnym generovaním opravnej konfigurácie na základe nálezu nedostatku, preto vzniknutá aplikácia bude podporovať aj vygenerovanie nápravy.

Príkladom open-source riešenia je \texttt{Cisco Config Analysis Tool}, ktorý čerpá odporúčania z jednej z kníh \cite{Singh2018}, pomocou ktorej boli vytvorené aj odporúčania v tejto práci. V tomto riešení však chýba veľa dôležitých prevádzkových a bezpečnostných odporúčaní z dôvodu, že námetom na kontrolný zoznam pri zostavovaní aplikácie bola iba jedna kniha. Taktiež podporuje iba jedného výrobcu sieťových zariadení a chýba mu modularita, nerozlišuje odporúčania a kontrolu ich prítomnosti na základe umiestnenia sieťového zariadenia v hierarchickom modeli. Nástrojom s podobnými vlasnosťami a nedostatkami je aj \texttt{Router Auditing Tool}, ktorý má naviac aj \zkratka{zkGUI}. Existuje niekoľko rozšírení aj pre nástroj \texttt{Nessus}, ktoré overujú dodržiavanie odporúčaní a podľa zistení čerpajú z CIS Benchmarku \cite{CIS_DrTLsgXv24lxeIIM} prípadne z ekvivalentu benchmarku pre zariadenia od výrobcu Juniper. Taktiež však nepodporujú zjednanie nápravy a ignorujú umiestnenie zariadenia v topológii.     

Výhodou výsledného programu je aj, že kontrolný zoznam vznikol z viacerých knižných odporúčaní a benchmarkov organizácií zaoberajúcimi sa danou problematikou. Program bude umožňovať spúšťanie modulov zodpovedných za nájdenie a odstránenie nedostatkov na základe definovaného umiestnenia zariadenia v hierarchickom modely siete. Tým sa zamedzí generovaniu falošne pozitívnych správ, ktoré by vznikli v dôsledku overovania nerelevantných požiadavkov na zariadenie v danej vrstve modelu. V neposlednom rade bude riešenie zdarma s možnosťou nahliadnuť a modifikovať respektíve rozšíriť kód.


\section{Rozdelenie príkazov}
Na zariadeniach od firmy Cisco s operačným systémom IOS bol vykonaný rozbor možných príkazov a ich foriem zápisu a početnosti výskytu v konfigurácií. Tento rozbor bol spravený z dôvodu, že niektoré príkazy sa môžu opakovať a zároveň jeden druh príkazu môže byť konfigurovaný v rôznych kontextoch a teda neprítomnosť v jednom kontexte automaticky neznamená nedostatok v konfigurácií. Na základe rozboru boli rozdelené príkazy na konfiguráciu sieťových zariadení do nasledujúcich štyroch kategórií:
\\
\begin{enumerate}
	\item Maximálne s jedným výskytom v konfigurácii\,--\,príkladom môže byť verzia protokolu \zk{zkSSH}.
	
\begin{minipage}{\linewidth}		
\begin{lstlisting}[frame=single,numbers=right,caption={Konfigurácia verzie protokolu SSH},label=lst:ver-ssh,basicstyle=\ttfamily\small, keywordstyle=\color{black}\bfseries\underbar,language=,]
Router(config)#ssh version 2
\end{lstlisting}
\end{minipage}
	
	\item \vspace{2em} Viacnásobný výskyt viazaný na rozhranie\,--\,typickým príkladom je zabezpečenie portu s definovaním maximálneho počtu povolených \zkratka{zkMAC} adries.
	
\begin{minipage}{\linewidth}		
\begin{lstlisting}[frame=single,numbers=right,caption={Konfigurácia maximálneho počtu povolených MAC adries na porte},label=lst:mac-max,basicstyle=\ttfamily\small, keywordstyle=\color{black}\bfseries\underbar,language=,breaklines=true]
Router(config)#interface FastEthernet0/1
Router(config-if)#switchport port-security mac address maximum 1
\end{lstlisting}
\end{minipage}

	\item \vspace{2em} Viacnásobný výskyt v konfigurácii\,--\,tieto príkazy konfigurujú rôzne služby, napríklad autentizáciu správ \zkratka{zkOSPF}.

\begin{minipage}{\linewidth}		
\begin{lstlisting}[frame=single,numbers=right,caption={Konfigurácia autentizácie OSPF na porte alebo v proccese},label=lst:ospf-auth,basicstyle=\ttfamily\small, keywordstyle=\color{black}\bfseries\underbar,language=,breaklines=true]
Router(config)#interface FastEthernet0/1
Router(config-if)#ip ospf message-digest-key 1 md5 heslo
Router(config-if)#ip ospf authentication message-digest

Router(config)#router ospf 1
Router(config)#area 0 authentication message-digest
Router(config)#area 0 authentication key-chain 1
\end{lstlisting}
\end{minipage}
	
	
	\item \vspace{2em} Všeobecný príkaz pre celé zariadenie a zároveň viacnásobný výskyt viazaný na rozhranie\,--\,s týmto nastavením je možné sa stretnúť pri protokole \zkratka{zkLLDP}, ktorý je možné zapnúť pre všetky porty globálne a následne selektovať porty, na ktorých nebude bežať.
	
\begin{minipage}{\linewidth}		
\begin{lstlisting}[frame=single,numbers=right,caption={Konfigurácia protokolu LLDP a vypnutie protokolu pre jeden port},label=lst:lldp,basicstyle=\ttfamily\small, keywordstyle=\color{black}\bfseries\underbar,language=,breaklines=true]
Router(config)#lldp run
Router(config)#interface FastEthernet0/1
Router(config-if)#no lldp receive
Router(config-if)#no lldp transmit
		

\end{lstlisting}
\end{minipage}
	
\end{enumerate}

\section{Rozdelenie sieťových prvkov}
Sieť je dnes navrhovaná zväčša podľa hierarchického modelu opísaného v kapitole \ref{hierarchicky-model}. Preto sa aj problémy a útoky v návrhu zatrieďujú podľa vrstvy, ktorú ovplyvňujú. V praxi sa však v menších sieťach funkcie jednotlivých vrstiev zlučujú, a preto boli okrem štandardných vrstiev nad rámec hierarchického modelu definované nasledujúce:

\begin{itemize}
	\item CORE/EDGE\,--\,core vrstva, prípadne s funkciou hraničného prvku.
	\item DIST\,--\,distribučná vrstva.
	\item ACC\,--\,prístupová vrstva.
	\item COLALL\,--\,všetky vyššie zmienené vrstvy zlúčené do jednej.
	\item COLDISTACC\,--\,zlúčená distribučná a prístupová vrstva.
	\item COLCOREDIST\,--\,zlúčená core a distribučná vrstva.
\end{itemize}

\section{Princíp fungovania}
vyvojovy diagram plus slovny opis
\subsection*{Hierarchická štruktúra}
Stromová štruktúra a koncept fungovania, Možno fungovanie cez nejaký UML diagram (sekvenčný?) alebo skôr niečo zjednodušené

\section{Zoznam odporúčaní}
 TODO: citacie k jednotlivym riadkom, prejst este raz planes a severity, eliminovat viac riadkov s loopback, skratky z tabulky treba vypisat                                                                                                                                                                                                                                                                                                                                                                                                                                                                                                                                                                                                     

V súčasnej dobe existuje mnoho odporúčaní, štandardov a benchmarkov, ktoré sa zaoberajú bezpečnosťou a správnou konfiguráciou sieťových zariadení. V mnohých prípadoch sú buď príliš všeobecné a teda sieťoví inžinieri majú problém zistiť, čo daným odporúčaním autor myslel a ako ho implementovať, alebo sú určené iba pre zariadenia od jedného výrobcu. Problémom je taktiež, že väčšina odporúčaní, štandardov a benchmarkov sa nie úplne prekrývajú, a teda je potrebné pri nastavovaní a audite zariadení čerpať s mnohých naraz. Výsledná tabuľka obsahuje odporúčania z odbornej literatúry a štandardov a benchmarkov verejne dostupných a používaných v produkčnom nasadení. Výhodou je aj fakt, že obsahuje odporúčania vychádzajúce z problémov IPv6, ktoré nie sú často v štandardoch a benchmarkoch dostupné. Podrobná tabuľka s mapovaním odporúčaní na príkazy pre zariadenia Cisco s operačným systémom IOS je v prílohe TODO priloha %TODO priloha.

Zariadenia Cisco boli pre túto prácu vybrané z dôvodu, že spoločnosť Cisco je lídrom ktorý udáva trend, ich zariadenia sú celosvetovo v korporáciách veľmi rozšírené a mnoho literatúry a benchmarkov sa odvoláva na nastavenia týchto prístrojov s udávanými príkladmi konfigurácie. Taktiež sú tieto zariadenia dobrým referenčným príkladom pre hľadanie alternatívy v zariadeniach od iných výrobcov.

V tabuľke \ref{checklist} je možné vidieť, že odporúčania sú rozdelené podľa viacerých kritérií. V prvom rade sú to roviny (plane), ktoré nie sú dôležité pre následnú automatickú konfiguráciu a odhaľovanie problémov, ale na vytvorenie si obrazu, ktorá časť rovín je kritická a postihnuteľná najviac. 

Stĺpec závažnosť (severity) vznikol na základe predpokladaných závažností. Tento atribút bude možné zmeniť v konfiguračnom súbore každého modulu v závislosti na riziku, ktoré sa pre danú topológiu a firmu vyhodnotí za pomoci manažmentu rizík opísaného v kapitole \ref{bezpecnostny-audit}. Tento atribút sa nenachádza v žiadnom štandarde ani benchmarku, z ktorého vytvorený zoznam odporúčaní čerpal, no je veľmi dôležitý z hľadiska, že nie všetky nedostatky sú rovnako závažné a nemajú rovnaký dopad. Hodnoty, ktoré nadobúda sú prebrané zo štandardu \zk{zkCVSS}, pričom posledný interval \texttt{none} reprezentujúci nulové riziko respektíve závažnosť je zamenený za kľúčové slovo \texttt{notify}. K tejto zmene prišlo z dôvodu, že problémy s nulovým rizikom nie sú súčasťou návrhu a nemá zmysel ich riešiť. V prípade, že bude nález falošne pozitívny alebo riziko bude akceptované, tak sa táto skutočnosť uloží do konfiguračného súboru. Závažnosť \texttt{notify} bude použitá v prípade prítomnosti monitorovania portu pomocou zrkadlenia portu alebo NetFlow/sFlow. Jedná sa totiž o technológie potrebné na monitorovanie prevádzky z legislatívnych alebo bezpečnostných dôvodov. Riziko existuje iba pri nesprávnom nastavení zdrojov monitorovania a cieľu pre zber dát, a preto je dobré vedieť pri audite o prítomnosti tohto nastavenia.
 

Posledným rozdelením je vrstva, na ktorej zariadenie pracuje (facility layer), nakoľko rozdelenie podľa zariadení nie je dostatočné, pretože napríklad L3 prepínač môže byť použitý na ktorejkoľvek vrstve hierarchického modelu a každá vrstva má určité špecifiká, ktoré neobsahuje iná vrstva. Každý konfiguračný súbor popisujúci zariadenie bude obsahovať informáciu, do ktorej vrstvy patrí a na základe toho bude môcť program rozhodnúť, ktoré moduly zodpovedné za nájdenie problému a jeho vyriešenie budú nad konfiguračným súborom zariadenia spustené. Taktiež bude možné meniť, dopĺňať a zakázať spúšťanie modulov pre jednotlivé zariadenia, pokiaľ by v danej topológii nevyhovovalo rozdelenie z tabuliek uvedených nižšie. 

Vrstva, na ktorej zariadenie operuje, ako aj definované zariadenie, ktorého sa odporúčanie a opatrenie týka nie sú súčasťou žiadneho kontrolného zoznamu, benchmarku ani štandardu, z ktorého bolo čerpané. Sieťový administrátor preto musí sám vyvodiť záver, ktoré odporúčania a postupy bude aplikovať na jednotlivé zariadenia a vrstvy hierarchického modelu. Preto vytvorená tabuľka odporúčaní už obsahuje aj zoznam zariadení, ktorých sa opatrenie týka.

\newpage


%>{\columncolor[rgb]{ .933,  .933,  .933}}
% Table generated by Excel2LaTeX from sheet 'Hárok1'

\footnotesize
\newgeometry{left=2.5cm,bottom=3.4cm}
\begin{longtable}[!htbp]{|C{2.5em}L{9em}L{9em}C{2em}C{2.5em}C{7.5em}L{10em}|}
	
	\hline
	\centering
	
	Riadok č.	&Útok / Problém	&Mitigácia / Nastavenie	&\rot{Plane} 	&\rot{Severity}	&Facility layer	&Zdroj\\
	\endhead
	
	\rowcolor[rgb]{ .933,  .933,  .933} 1	&Nepovolený prístup k manažovaniu zariadenia	&Vytvoriť a aplikovať ACL pre OOB, Telnet, SSH a pod. a zaznamenať v logu prístupy	&M	&C	&A	&Hardening Cisco Routers \cite{Akin2002}
	
	CIS Cisco IOS 15 Benchmark \cite{CIS_DrTLsgXv24lxeIIM}	\\
	2	&Neautorizovaný prístup cez nepoužívané a nezabezpečené protokoly na manažment zariadení	&Vypnúť nepoužívané protokoly na prístup k manažovaniu zariadení (telnet a pod.)	&M	&H	&A	&CIS Cisco IOS 15 Benchmark \cite{CIS_DrTLsgXv24lxeIIM}
	
	Cisco Guide to Harden Cisco IOS Devices \cite{Singh2018}
	\\
	\rowcolor[rgb]{ .933,  .933,  .933} 3	&Nepovolený prístup k manažmentu konfigurácie zariadenia	&Vypnutie odchádzajúcich spojení pre protokoly na manažment zariadení pokiaľ sa nepoužívajú (telnet a pod.)	&M	&H	&A	&Cisco Guide to Harden Cisco IOS Devices \cite{Singh2018}	\\
	4	&Prítup bez požadovaných prístupových údajov	&Nakonfigruovanie protokolov na manažment zariadení, aby požadovali prístupové údaje (telnet a pod.)	&M	&C	&A	&CIS Cisco IOS 15 Benchmark \cite{CIS_DrTLsgXv24lxeIIM}	\\
	\rowcolor[rgb]{ .933,  .933,  .933} 5	&Nekonzistenia konfiguračných súborov pri zmenách konfigurácie viac ako jedným administrátorom	&Povolit súčasne iba jednému administrátorovi vykonávanie zmien v konfigurácii	&M	&H	&A	&Cisco Guide to Harden Cisco IOS Devices \cite{Singh2018}	\\
	6	&Nepoužívanie zabezpečeného protokolu na manažment zariadení môže viesť k odposluchu	&Zapnutie SSH	&M	&C	&A	&CIS Cisco IOS 15 Benchmark \cite{CIS_DrTLsgXv24lxeIIM}
	
	Cisco Router Hardening \cite{Graesser2001}	\\
	\rowcolor[rgb]{ .933,  .933,  .933} 7	&Nebezpečná verzia 1 protokolu SSH	&SSH verzia 2	&M	&C	&A	&Cisco CCNA Security Study Guide \cite{McMillan2018}	\\
	8	&Útok na krátky RSA kĺúč	&Dĺžka RSA kľúča minimálne 2048 bitov	&M	&C	&A	&CIS Cisco IOS 15 Benchmark \cite{CIS_DrTLsgXv24lxeIIM}
	
	Transitioning the Use of Cryptographic Algorithms and Key Lengths \cite{Barker2019} 
	\\
	\rowcolor[rgb]{ .933,  .933,  .933} 9	&Dlhé neaktívne sedenie môže byť zneužité alebo aj fyzický prístup útočníka k aktívnemu sedeniu môže viesť k zmene konfigurácie	&SSH čas vypršania sedenia	&M	&M	&A	&CIS Cisco IOS 15 Benchmark \cite{CIS_DrTLsgXv24lxeIIM}
	
	Cisco Router Hardening \cite{Graesser2001}	\\
	10	&Hádanie hesla k RSA kľúču	&SSH maximálny počet neúspešných pokusov	&M	&H	&A	&Cisco router configuration handbook \cite{Hucaby2010}
	Cisco IOS 23 - Autentizace uživatele na switchi vůči Active Directory \cite{Bouska2009}	\\
	\rowcolor[rgb]{ .933,  .933,  .933} 11	&Útok hrubou silou na zistenie prihlasovacích údajov	&Špecifikovať čas po ktorý nie je možné po N pokusoch sa prihlásiť	&M	&H	&A	&Cisco router configuration handbook \cite{Hucaby2010}
	
	Cisco IOS 23 - Autentizace uživatele na switchi vůči Active Directory \cite{Bouska2009}	\\
	12	&Prihlásenie na zariadenie nie je možné kvôli zablokovaniu pre príliš veľa neúspešných pokusov	&Povolenie prístupu administrátorovi na základe IP adresy, keď je protokol na manažovanie zariadení nedostupný kvôli DOS útoku	&M	&M	&A	&Cisco router configuration handbook \cite{Hucaby2010}
	
	Cisco IOS 23 - Autentizace uživatele na switchi vůči Active Directory \cite{Bouska2009}	\\
	\rowcolor[rgb]{ .933,  .933,  .933} 13	&Dlhé neaktívne sedenie môže byť zneužité alebo aj fyzický prístup útočníka k aktívnneum sedeniu môže viesť k zmene konfigurácie	&Čas vypršania sedenia pre protokol na manažovanie zariadení	&M	&M	&A	&CIS Cisco IOS 15 Benchmark \cite{CIS_DrTLsgXv24lxeIIM}
	Cisco Router Hardening \cite{Graesser2001}
	Cisco SAFE Reference Guide \cite{uYLsMtQInofenpV3}
	\\
	14	&Možné prihlásenie do zariadenia cez telnet keď je prítomné SSH	&Zakázať telnet ak je SSH aktívne	&M	&C	&A	&CIS Cisco IOS 15 Benchmark \cite{CIS_DrTLsgXv24lxeIIM}
	
	Cisco Router Hardening \cite{Graesser2001}	\\
	\rowcolor[rgb]{ .933,  .933,  .933} 15	&Útočník nie je informovaný o právnych následkoch	&Právne upozornenie pri prístupe k zariadeniu	&M	&L	&A	&Cisco CCNA Security Study Guide \cite{McMillan2018}
	
	CIS Cisco IOS 15 Benchmark \cite{CIS_DrTLsgXv24lxeIIM}
	
	Cisco Router Hardening \cite{Graesser2001}	\\
	16	&Nepovolená zmena konfigurácie zariadenia	&Vytvorenie hesla na editovanie konfigurácie zariadenia	&M	&C	&A	&CIS Cisco IOS 15 Benchmark \cite{CIS_DrTLsgXv24lxeIIM}
	
	Cisco Router Hardening \cite{Graesser2001}	\\
	\rowcolor[rgb]{ .933,  .933,  .933} 17	&Nepovolený prístup k manažmentu konfigurácie zariadenia	&Lokálne zabezpečené účty	&M	&C	&A	&Cisco Guide to Harden Cisco IOS Devices \cite{Singh2018}
	
	CIS Cisco IOS 15 Benchmark \cite{CIS_DrTLsgXv24lxeIIM}	\\
	18	&Centrálna správa prihlásení a dohľadateľnosť zmien v konfigurácií	&Definovanie a povolenie AAA serveru na prihlásenie a definovanie záložného prihlásenia	&M	&H	&A	&Hardening Cisco Routers \cite{Akin2002}
	
	CIS Cisco IOS 15 Benchmark \cite{CIS_DrTLsgXv24lxeIIM} 
	
	Cisco CCNA Security Study Guide \cite{McMillan2018}
	
	Cisco Router Hardening \cite{Graesser2001}	\\
	\rowcolor[rgb]{ .933,  .933,  .933} 19	&Centrálna správa prihlásení a dohľadateľnosť zmien v konfigurácií	&Definovanie a povolenie AAA serveru na editáciu konfigurácií a definovanie záložného prihlásenia	&M	&M	&A	&CIS Cisco IOS 15 Benchmark \cite{CIS_DrTLsgXv24lxeIIM}
	
	Cisco Router Hardening \cite{Graesser2001}	\\
	20	&Hádanie prístupových údajov	&Definovanie maximálneho počtu neúspešných pokusov o prihlásenie a následné zablokovanie účtu	&M	&H	&A	&CIS Cisco IOS 15 Benchmark \cite{CIS_DrTLsgXv24lxeIIM}
	
	Cisco Router Hardening \cite{Graesser2001}	\\
	\rowcolor[rgb]{ .933,  .933,  .933} 21	&Prihlásenie bez prihlasovacích údajov	&Zakázať záložné prihlásenie bez poskynutia autentizačných prostriedkov	&M	&C	&A	&Cisco Guide to Harden Cisco IOS Devices \cite{Singh2018}	\\
	22	&AAA používa primárne lokálne účty namiesto centralizovaných na serveri	&AAA nesmie používať ako prvú možnosť prihlásenia lokálny účet 	&M	&H	&A	&CIS Cisco IOS 15 Benchmark \cite{CIS_DrTLsgXv24lxeIIM}
	
	Cisco Router Hardening \cite{Graesser2001}	\\
	\rowcolor[rgb]{ .933,  .933,  .933} 23	&Používateľ prihlásený do zariadenia môže spúšťať akékoľvek príkazy	&Nastavenie AAA autorizácie pre spúštanie príkazov. V prípade výpadku AAA serveru, bude užívateľ odhlásený a následne prihlásený podľa  záložného prihlásenia, aby mu nebolo pridelené vysoké oprávnenie umožňujúce vykonávať príkazy, na ktoré nemá právo	&M	&H	&A	&Cisco Router Hardening \cite{Graesser2001}
	
	Cisco Guide to Harden Cisco IOS Devices \cite{Singh2018}	\\
	24	&Administrátor vloží zlý príkaz a po čase je ho nemožné dohľadať a zjednať nápravu	&Nastavenie AAA účtovania respektíve logovania pripojení a vykonaných príkazov	&M	&H	&A	&CIS Cisco IOS 15 Benchmark \cite{CIS_DrTLsgXv24lxeIIM}	\\
	\rowcolor[rgb]{ .933,  .933,  .933} 25	&AAA zdrojové rozhranie nie je rovnaké pri každom reštarte	&Definovanie loopback zdrojového rozhrania pre AAA	&M	&M	&A	&CIS Cisco IOS 15 Benchmark \cite{CIS_DrTLsgXv24lxeIIM}	\\
	26	&SSH zdrojové rozhranie nie je rovnaké pri každom reštarte	& Definovanie loopback zdrojového rozhrania pre SSH	&M	&M	&A	&CIS Cisco IOS 15 Benchmark \cite{CIS_DrTLsgXv24lxeIIM}	\\
	\rowcolor[rgb]{ .933,  .933,  .933} 27	&DOS útok na štandardný SSH port 22	&Špecifikovanie iného portu pre SSH ako štandardného alebo aplikovanie Port Knocking \cite{MJVmQiKUgZl92S8u}	&M	&H	&A	&Port Knocking \cite{MJVmQiKUgZl92S8u}	\\
	
	\hline
	\caption{Odporúčania k prístupu na manažment zariadení}
	\label{tab:managemnet}%
\end{longtable}%

\begin{longtable}[!htbp]{|C{2.5em}L{9em}L{9em}C{2em}C{2.5em}C{7.5em}L{10em}|}
	
	\hline
	\centering
	
	Riadok č.	&Útok / Problém	&Mitigácia / Nastavenie	&\rot{Plane} 	&\rot{Severity}	&Facility layer	&Zdroj\\
	\endhead
	
	\rowcolor[rgb]{ .933,  .933,  .933} 1	&Vloženie a manipulácia so smerovacími informáciami	&Autentizácia smerovacích protokolov (nie heslá v otvorenej podobe)	&C	&H	&CORE/EDGE
	DIST
	COLCOREDIST
	COLDISTACC
	COLALL	&Cisco CCNA Security Study Guide \cite{McMillan2018}
	
	Cisco Guide to Harden Cisco IOS Devices \cite{Singh2018}
	
	CIS Cisco IOS 15 Benchmark \cite{CIS_DrTLsgXv24lxeIIM}\\
	2	&OSPF virtuálne linky degradujú výkon	&Vypnutie virtuálnych liniek pre OSPF	&C	&H	&CORE/EDGE
	DIST
	COLCOREDIST
	COLDISTACC
	COLALL	&Designing Cisco network service architectures \cite{Tiso2012}\\
	\rowcolor[rgb]{ .933,  .933,  .933} 3	&Koncové zariadenie, užívateľ a útočník môžu vidiet smerovacie správy a topológiu siete alebo pripojenie škodlivého zariadenia, ktoré vysielať a prijímať smerovacie správy	&Špecifikovanie rozhraní, ktoré nebudú prijímať routovacie informácie	&C	&H	&CORE/EDGE
	DIST
	COLCOREDIST
	COLDISTACC
	COLALL	&OSPF Security: Attacks and Defenses \cite{Khandelwal2016}\\
	4	&Nemožnosť sprevádzkovať procesy smerovacích protokolov v určitých prípadoch pri použití IPv6	&Špecifikovanie identifikátorov smerovacích protokolov pre každý router (router ID)	&C	&M	&CORE/EDGE
	DIST
	COLCOREDIST
	COLDISTACC
	COLALL	&Protocol-Independent Routing Properties Feature Guide \cite{q7WZuvqA1fZEsYyL}
	
	CCNA Routing and Switching Study Guide \cite{Lammle2013}\\
	\rowcolor[rgb]{ .933,  .933,  .933} 5	&Vysledovateľnosť nefunkčnosti routovacieho protokolu a nesprávneho nastavenia	&Zaznamenie zmeny v logu pri zmenách v smerovaní	&C	&M	&CORE/EDGE
	DIST
	COLCOREDIST
	COLDISTACC
	COLALL	&Cisco SAFE Reference Guide \cite{uYLsMtQInofenpV3}
	\\
	6	& Škodlivé vloženie smerovacích informácií informácií, vzdialený útok	&TTL security	&C	&H	&CORE/EDGE
	DIST
	COLCOREDIST
	COLDISTACC
	COLALL	&OSPF Security: Attacks and Defenses \cite{Khandelwal2016}
	
	Cisco Guide to Harden Cisco IOS Devices \cite{Singh2018}\\
	\rowcolor[rgb]{ .933,  .933,  .933} 7	&Nesprávne smerovanie kvôli sumarizácií	&Vypnutie automatickej sumarizácie smerovacích protokolov	&C	&H	&CORE/EDGE
	DIST
	COLCOREDIST
	COLDISTACC
	COLALL	&CCNA Routing and Switching Study Guide \cite{Lammle2013}\\
	8	&DOS útok na stanicu, cez ktorú bola špecifikovaná cesta a teda nemožnosť komunikácie s koncovým bodom. Alebo zosnovanie MITM útoku	&Vypnutie IP source routing	&C	&C	&CORE/EDGE
	DIST
	COLCOREDIST
	COLDISTACC
	COLALL	&CIS Cisco IOS 15 Benchmark \cite{CIS_DrTLsgXv24lxeIIM}\\
	\rowcolor[rgb]{ .933,  .933,  .933} 9	&DOS útok pomocou podvrhnutej IP adresy alebo vzdialený útok na smerovací protokol	&Zapnutie reverse path forwarding strict/loose mode	&C	&H	&CORE/EDGE
	DIST
	COLCOREDIST
	COLDISTACC
	COLALL	&OSPF Security: Attacks and Defenses \cite{Khandelwal2016}
	
	Network Security Auditing \cite{Jackson2010}
	
	CIS Cisco IOS 15 Benchmark \cite{CIS_DrTLsgXv24lxeIIM}\\
	
	\hline
	\caption{Odporúčania pre smerovanie}
	\label{tab:routing}%
\end{longtable}%

\begin{longtable}[!htbp]{|C{2.5em}L{9em}L{9em}C{2em}C{2.5em}C{7.5em}L{10em}|}

	\hline
	\centering
	
	Riadok č.	&Útok / Problém	&Mitigácia / Nastavenie	&\rot{Plane} 	&\rot{Severity}	&Facility layer	&Zdroj\\
	\endhead
	

	\rowcolor[rgb]{ .933,  .933,  .933} 1	&IP spoofing	&Špecifikácia ACL na zakázanie a logovanie privátnych a špeciálnych IP adries z RFC 1918, RFC 3330	&C	&C	&CORE/EDGE
	COLCOREDIST
	COLALL	&Network Security Auditing \cite{Jackson2010}
	
	Cisco Guide to Harden Cisco IOS Devices \cite{Singh2018}
	
	CIS Cisco IOS 15 Benchmark \cite{CIS_DrTLsgXv24lxeIIM}\\
	2	&IP spoofing	&Špecifikácia ACL na zakázanie a logovanie špeciálnych IPv6 adries z RFC 5156	&C	&C	&CORE/EDGE
	COLCOREDIST
	COLALL	&Network Security Auditing \cite{Jackson2010}
	
	Cisco Guide to Harden Cisco IOS Devices \cite{Singh2018}
	
	CIS Cisco IOS 15 Benchmark \cite{CIS_DrTLsgXv24lxeIIM}\\
	\rowcolor[rgb]{ .933,  .933,  .933} 3	&IPv6 Next Header,
	IPv6 Fragmentation útok	&ACL blokujúce nerozpoznateľne rozšírené hlavičky	&C	&C	&A	&Bezpečné IPv6: trable s hlavičkami \cite{Podermanski1922015}
	
	Bezpečné IPv6: vícehlavý útočník \cite{Gregr2622015}\\
	4	&DOS útok alebo pokus o prístup k tomu, čo nie je povolené	&Logovanie pravidiel zahodenia paketov v ACL	&M	&M	&A	&Hardening Cisco Routers \cite{Akin2002}\\
	\rowcolor[rgb]{ .933,  .933,  .933} 5	&Packety budú spracovávané v CPU, ktoré môže byť preťažené a môže byť zmenené smerovanie na obídenie bezpečnostnej kontroly	&Zahadzovanie IPv4 paketov s rozšírenou hlavičkou (IP Options filtering)	&C	&C	&CORE/EDGE
	DIST
	COLCOREDIST
	COLDISTACC
	COLALL	&Cisco Guide to Harden Cisco IOS Devices \cite{Singh2018}\\
	6	&Komplexné bezpečnostné hrozby a narušenie bezpečnosti	&Nastavenie IDS/IPS ak to zariadenie podporuje	&C	&H	&CORE/EDGE COLCOREDIST
	COLALL	&Cisco router configuration handbook \cite{Hucaby2010}\\
	\hline
\caption{Odporúčania pre filtrovanie prevádzky}
\label{tab:filtering}%
\end{longtable}%

\begin{longtable}[!htbp]{|C{2.5em}L{9em}L{9em}C{2em}C{2.5em}C{7.5em}L{10em}|}
	
	\hline
	\centering
	
	Riadok č.	&Útok / Problém	&Mitigácia / Nastavenie	&\rot{Plane} 	&\rot{Severity}	&Facility layer	&Zdroj\\
	\endhead
	
	\rowcolor[rgb]{ .933,  .933,  .933} 1	&Nízky stav voľnej pamäte	&Nastavenie notifikácie pri dochádzaní pamäte	&M	&M	&A	&Cisco Guide to Harden Cisco IOS Devices \cite{Singh2018}
	
	Cisco SAFE Reference Guide \cite{uYLsMtQInofenpV3}\\
	2	&Logovacie správy nemôžu byť zaznamenané kvôli nedostatku pamäte	&Rezervovanie pamäte pre kritické notifikácie pri nedostatku pamäte	&M	&H	&A	&Cisco Guide to Harden Cisco IOS Devices \cite{Singh2018}
	
	Cisco SAFE Reference Guide \cite{uYLsMtQInofenpV3}\\
	\rowcolor[rgb]{ .933,  .933,  .933} 3	&Vysoké zaťaženie CPU	&Nastavenie notifikácie vysokom zaťažení CPU	&M	&M	&A	&Cisco Guide to Harden Cisco IOS Devices \cite{Singh2018}
	
	Cisco SAFE Reference Guide \cite{uYLsMtQInofenpV3}\\
	4	&Vysoké zaťaženie zariadenia spôsobilo nemožnosť prihlásenia k nemu	&Rezervovanie pamäte preprotokoly na manažment zariadení pri nedostatku pamäte	&M	&H	&A	&Cisco Guide to Harden Cisco IOS Devices \cite{Singh2018}\\
	
	
	\hline
	\caption{Odporúčania pri vysokom zaťažení}
	\label{tab:highload}%
\end{longtable}%

\begin{longtable}[!htbp]{|C{2.5em}L{9em}L{9em}C{2em}C{2.5em}C{7.5em}L{10em}|}
	
	\hline
	\centering
	
	Riadok č.	&Útok / Problém	&Mitigácia / Nastavenie	&\rot{Plane} 	&\rot{Severity}	&Facility layer	&Zdroj\\
	\endhead
	
	\rowcolor[rgb]{ .933,  .933,  .933} 1	&Skenovanie a zistenie informácií o sieti za pomoci protokolu CDP a využitie bezpečnostných chýb	&Zakázanie protokolu CDP	&M	&C	&A	&CIS Cisco IOS 15 Benchmark \cite{CIS_DrTLsgXv24lxeIIM}
	
	Cisco Router Hardening \cite{Graesser2001}\\
	2	&Skenovanie a zistenie informácií o sieti za pomoci protokolu LLDP a využitie bezpečnostných chýb	&Zakázanie protokolu LLDP	&M	&C	&A	&Cisco CCNA Security Study Guide \cite{McMillan2018}\\
	\rowcolor[rgb]{ .933,  .933,  .933} 3	&Proxy ARP môže viesť k obídeniu PVLAN a rozširuje broadcast doménu	&Vypnutie Proxy ARP	&C	&C	&CORE/EDGE
	DIST
	COLCOREDIST
	COLDISTACC
	COLALL	&CIS Cisco IOS 15 Benchmark \cite{CIS_DrTLsgXv24lxeIIM}
	
	Cisco Router Hardening \cite{Graesser2001}\\
	4	&Útočník môže zistiť, že IP adresa, na ktorú skušal ping je nesprávna	&Vypnutie spáv ICMP Unreachable	&D	&H	&CORE/EDGE
	DIST
	COLCOREDIST
	COLDISTACC
	COLALL	&Cisco Guide to Harden Cisco IOS Devices \cite{Singh2018}
	\\
	\rowcolor[rgb]{ .933,  .933,  .933} 5	&Útočník môže zistiť masku podsiete pomocou ICMP Mask reply	&Vypnutie spáv ICMP Mask reply	&D	&H	&CORE/EDGE
	DIST
	COLCOREDIST
	COLDISTACC
	COLALL	&Hardening Cisco Routers \cite{Akin2002}
	\\
	6	&Umožňuje DOS Smurf útok, mapovanie siete pomocou ping na broadcast adresu vzdialenej siete	&Vypnutie ICMP echo správ na broadcast adresu, vypnutie directed broadcasts	&D	&C	&CORE/EDGE
	DIST
	COLCOREDIST
	COLDISTACC
	COLALL	&Cisco Router Hardening \cite{Graesser2001}
	
	Hardening Cisco Routers \cite{Akin2002}
	\\
	\rowcolor[rgb]{ .933,  .933,  .933} 7	&Útočník môže zistiť smerovacie informácie alebo vyťažiť CPU	&Vypnutie spáv ICMP Redirects	&D	&H	&CORE/EDGE
	DIST
	COLCOREDIST
	COLDISTACC
	COLALL	&Cisco Guide to Harden Cisco IOS Devices \cite{Singh2018}
	\\
	8	&Mapovanie sete pomocou pingu na multicast adresu všetkých uzlov a MLD/IGMP Query Overload a Smurf útok	& ACL blokujúce ICMP echo request na multicast adresu všetkých uzlov a MLD/IGMP Query na prístupových portoch	&C	&M	&DIST
	COLDISTACC
	ACC	&Bezpečné IPv6: trable s multicastem \cite{Podermanski532015}
	
	MLD Considered Harmful \cite{Rey2016}
	\\
	
	\hline
	\caption{Odporúčania na zamedzenie mapovanie siete}
	\label{tab:mapping}%
\end{longtable}%

\begin{longtable}[!htbp]{|C{2.5em}L{9em}L{9em}C{2em}C{2.5em}C{7.5em}L{10em}|}
	
	\hline
	\centering
	
	Riadok č.	&Útok / Problém	&Mitigácia / Nastavenie	&\rot{Plane} 	&\rot{Severity}	&Facility layer	&Zdroj\\
	\endhead
	
	\rowcolor[rgb]{ .933,  .933,  .933} 1	&Nemožná identifikácia zariadenia	&Vytvoriť hostname	&M	&L	&A	&CIS Cisco IOS 15 Benchmark \cite{CIS_DrTLsgXv24lxeIIM}\\
	2	&Nemožnosť vzdialeného prístupu	&Vytvoriť doménové meno	&M	&L	&A	&CIS Cisco IOS 15 Benchmark \cite{CIS_DrTLsgXv24lxeIIM}\\
	\rowcolor[rgb]{ .933,  .933,  .933} 3	&Identifikácia pravidla v ACL	&Popis každého pravidla v ACL pre lepšiu identifikáciu	&M	&L	&A	&Cisco Guide to Harden Cisco IOS Devices \cite{Singh2018}\\
	4	&Indentifikácia rozhrania	&Popis každého rozhrania	&M	&L	&A	&CCNA Routing and Switching Study Guide \cite{Lammle2013}\\
	\rowcolor[rgb]{ .933,  .933,  .933} 5	&Nemožnosť identifikácie účelu VLAN	&Pridanie mena k VLAN	&C	&L	&DIST
	COLDISTACC
	ACC
	
	COLALL	&CCNA Routing and Switching Study Guide \cite{Lammle2013}\\
	
	\hline
	\caption{Odporúčania na identifikáciu zariadení a nastavení}
	\label{tab:identification}%
\end{longtable}%

\begin{longtable}[!htbp]{|C{2.5em}L{9em}L{9em}C{2em}C{2.5em}C{7.5em}L{10em}|}
	
	\hline
	\centering
	
	Riadok č.	&Útok / Problém	&Mitigácia / Nastavenie	&\rot{Plane} 	&\rot{Severity}	&Facility layer	&Zdroj\\
	\endhead

	\rowcolor[rgb]{ .933,  .933,  .933} 1	&Nekonzistencia časov v logoch a problém pričlenenia logov k relevantným incidentom	&Nastavenie NTP serveru pre aktuálny čas v logoch	&M	&H	&A	&Network Security Auditing \cite{Jackson2010}
	
	Cisco Guide to Harden Cisco IOS Devices \cite{Singh2018}
	
	CIS Cisco IOS 15 Benchmark \cite{CIS_DrTLsgXv24lxeIIM}\\
	2	&Pripojenie servera s rovnakou IP adresou, ale falošným časom	&Nastavenie NTP autentizácie	&M	&H	&A	&Network Security Auditing \cite{Jackson2010}
	
	Cisco Guide to Harden Cisco IOS Devices \cite{Singh2018}
	
	CIS Cisco IOS 15 Benchmark \cite{CIS_DrTLsgXv24lxeIIM}\\
	\rowcolor[rgb]{ .933,  .933,  .933} 3	&NTP zdrojové rozhranie nie je rovnaké pri každom reštarte	& Definovanie loopback zdrojového rozhrania pre NTP	&M	&M	&A	&Network Security Auditing \cite{Jackson2010}
	
	Cisco Guide to Harden Cisco IOS Devices \cite{Singh2018}
	
	CIS Cisco IOS 15 Benchmark \cite{CIS_DrTLsgXv24lxeIIM}\\
	4	&Väčšia bezpečnosť (pub/priv key) NTP a podpora IPv6	&Použitie NTP verzie 4	&M	&M	&A	&The NTP FAQ and HOWTO \cite{s0goWNnWp5OjqREE}\\
	\rowcolor[rgb]{ .933,  .933,  .933} 5	&Falošný čas od podvrhnutého NTP zdroja	&Nastavenie NTP peer s inými sieťovými zariadeniami na krížovú validáciu času a záložný zdroj času	&M	&M	&A	&Hardening Cisco Routers \cite{Akin2002}\\
	
	\hline
	\caption{Odporúčania k protokolu NTP}
	\label{tab:ntp}%
\end{longtable}%

\begin{longtable}[!htbp]{|C{2.5em}L{9em}L{9em}C{2em}C{2.5em}C{7.5em}L{10em}|}
	
	\hline
	\centering
	
	Riadok č.	&Útok / Problém	&Mitigácia / Nastavenie	&\rot{Plane} 	&\rot{Severity}	&Facility layer	&Zdroj\\
	\endhead
	
	\rowcolor[rgb]{ .933,  .933,  .933} 1	&Odpočúvanie SNMP verzie 1 a 2c	&Použitie SNMP verie 3 pokiaľ je SNMP používané	&M	&C	&A	&CIS Cisco IOS 15 Benchmark \cite{CIS_DrTLsgXv24lxeIIM}
	
	Cisco Router Hardening \cite{Graesser2001}\\
	2	&Modifikovanie konfigurácie pomocou SNMP	&Obmedzenie SNMP iba na čítanie	&M	&C	&A	&CIS Cisco IOS 15 Benchmark \cite{CIS_DrTLsgXv24lxeIIM}
	
	Cisco Router Hardening \cite{Graesser2001}
	
	Hardening Cisco Routers \cite{Akin2002}\\
	\rowcolor[rgb]{ .933,  .933,  .933} 3	&Neoprávnený prístup k SNMP informáciám	&Obmedzenie SNMP iba pre vybrané IP adresy	&M	&H	&A	&CIS Cisco IOS 15 Benchmark \cite{CIS_DrTLsgXv24lxeIIM}
	
	Cisco Router Hardening \cite{Graesser2001}\\
	4	&Administrátor nemá povedomie o problémoch na zariadení	&Povolenie asynchrónnych správ SNMP TRAP	&M	&M	&A	&CIS Cisco IOS 15 Benchmark \cite{CIS_DrTLsgXv24lxeIIM}
	
	Cisco Router Hardening \cite{Graesser2001}
	
	Hardening Cisco Routers \cite{Akin2002}\\
	\rowcolor[rgb]{ .933,  .933,  .933} 5	&Odpočúvanie SNMP sedenie z dôvodu slabého šifrovania a hashovacej  funkcie	&Vytvorenie SNMP verzie 3 užívateľa s minimálnym šifrovaním AES 128 bit a hashovacou funkciou SHA	&M	&C	&A	&Transitioning the Use of Cryptographic Algorithms and Key Lengths \cite{Barker2019}
	
	CIS Cisco IOS 15 Benchmark \cite{CIS_DrTLsgXv24lxeIIM}
	
	Hardening Cisco Routers \cite{Akin2002}\\
	6	&Hard identification of SNMP messages from many IPs/ Sťažená identifikácia SNMP správ z rôznych IP	&Definovanie lokácie SNMP serveru	&M	&L	&A	&Cisco IOS Cookbook \cite{Dooley2007}\\
	\rowcolor[rgb]{ .933,  .933,  .933} 7	&SNMP zdrojové rozhranie nie je rovnaké pri každom reštarte	& Definovanie loopback zdrojového rozhrania pre SNMP	&M	&M	&A	&CIS Cisco IOS 15 Benchmark \cite{CIS_DrTLsgXv24lxeIIM}\\
	8	&Zmeny názvov rozhraní medzi reštartami a nemožnosť monitorovania pomocou SNMP	&SNMP statické nemenné meno rozhrania aj po reštarte zariadenia	&M	&H	&A	&Cisco IOS Cookbook \cite{Dooley2007}\\
	
	\hline
	\caption{Odporúčanie pre protokol SNMP}
	\label{tab:snmp}%
\end{longtable}%

\begin{longtable}[!htbp]{|C{2.5em}L{9em}L{9em}C{2em}C{2.5em}C{7.5em}L{10em}|}
	
	\hline
	\centering
	
	Riadok č.	&Útok / Problém	&Mitigácia / Nastavenie	&\rot{Plane} 	&\rot{Severity}	&Facility layer	&Zdroj\\
	\endhead
	
	\rowcolor[rgb]{ .933,  .933,  .933} 1	&Administrátor nemá povedomie o problémoch na zariadení	&Povolenie logovania protokolom SYSLOG a špecifikovanie IP adresy SYSLOG serveru	&M	&H	&A	&CIS Cisco IOS 15 Benchmark \cite{CIS_DrTLsgXv24lxeIIM}
	
	Cisco Router Hardening \cite{Graesser2001}\\
	2	&Neprijímanie všetkých dôležitých incidentov na zariadení z protokolu SYSLOG	&Špecifikovanie dôležitosti oznámenií SYSLOG na INFORMATIONAL	&M	&M	&A	&CIS Cisco IOS 15 Benchmark \cite{CIS_DrTLsgXv24lxeIIM}\\
	\rowcolor[rgb]{ .933,  .933,  .933} 3	&SYSLOG zdrojové rozhranie nie je rovnaké pri každom reštarte	& Definovanie loopback zdrojového rozhrania pre SYSLOG	&M	&M	&A	&Cisco Guide to Harden Cisco IOS Devices \cite{Singh2018}
	
	CIS Cisco IOS 15 Benchmark \cite{CIS_DrTLsgXv24lxeIIM}\\
	4	&Insufficient and non-standard  time format in logging messages/ Nedostatočné a neštandardné formáty času pri logovacích správach	&Definovanie formátu času pre logovacie a ladiace výstupy	&M	&M	&A	&CIS Cisco IOS 15 Benchmark \cite{CIS_DrTLsgXv24lxeIIM}
	
	Cisco Router Hardening \cite{Graesser2001}\\
	\rowcolor[rgb]{ .933,  .933,  .933} 5	&Administrátor nevidí dôležité incidenty pri prihlásení a konfigurovaní cez konzolu	&Vypisovanie SYSLOG správ CRITICAL a dôležitejších do terminálu	&M	&M	&A	&Cisco Guide to Harden Cisco IOS Devices \cite{Singh2018}
	
	CIS Cisco IOS 15 Benchmark \cite{CIS_DrTLsgXv24lxeIIM}\\
	6	&Malá vyrovnávacia pamäť pre SYSLOG je dôvodom zahadzovanie správ	&Definovanie veľkosti SYSLOG buffera dôležitosti oznámení na INFORMATIONAL	&M	&H	&A	&Cisco Guide to Harden Cisco IOS Devices \cite{Singh2018}\\
	\rowcolor[rgb]{ .933,  .933,  .933} 7	&Neprístupný SYSLOG server spôsobuje zahadzovanie dôležitých syslog správ	&Definovanie dočasného úložiska SYSLOG správ v prípade nedostupnosti servera	&M	&H	&A	&Cisco Guide to Harden Cisco IOS Devices \cite{Singh2018}\\
	8	&Problém identifikácie SYSLOG správ s rovnakou časovou značkou	&Pridanie sekvenčného čísla ku každej syslog správe	&M	&L	&A	&Hardening Cisco Routers \cite{Akin2002}\\
	
	\hline
	\caption{Odporúčania pre protokol Syslog}
	\label{tab:syslog}%
\end{longtable}%

\begin{longtable}[!htbp]{|C{2.5em}L{9em}L{9em}C{2em}C{2.5em}C{7.5em}L{10em}|}
	
	\hline
	\centering
	
	Riadok č.	&Útok / Problém	&Mitigácia / Nastavenie	&\rot{Plane} 	&\rot{Severity}	&Facility layer	&Zdroj\\
	\endhead
	\rowcolor[rgb]{ .933,  .933,  .933} 1	&MAC Spoofing, MAC Flooding 	&Definovanie maximálne 1 MAC adresy na port, priradenie MAC adresy na port	&C	&C	&DIST
	COLDISTACC
	ACC	&CCNA Routing and Switching Study Guide \cite{Lammle2013}\\
	2	&MAC Spoofing, MAC Flooding 	&Nastavenie režimu narušenia, ktorý vypne port alebo informuje správcu o pripojení nepovoleného zariadenia	&C	&H	&DIST
	COLDISTACC
	ACC	&CCNA Routing and Switching Study Guide \cite{Lammle2013}\\
	\rowcolor[rgb]{ .933,  .933,  .933} 3	&Využívanie siete nepovolenými používateľmi	&Zapnutie 802.1x 	&C	&H	&DIST
	COLDISTACC
	ACC	&Lan Switch Security \cite{Vyncke2008}
	
	Cisco IOS 11 - IEEE 802.1x, autentizace k portu, MS IAS \cite{Bouska20071}
	
	Cisco IOS 12 - IEEE 802.1x a pokročilejší funkce  \cite{Bouska2007} \\
	4	&Útok hrubou silou hádaním prístupových údajov pre 802.1x 	&Limitovanie maximálneho počtu neúspešných pokusov o autentizáciu 802.1x	&C	&H	&DIST
	COLDISTACC
	ACC	&Lan Switch Security \cite{Vyncke2008}
	
	Cisco IOS 11 - IEEE 802.1x, autentizace k portu, MS IAS \cite{Bouska20071}
	
	Cisco IOS 12 - IEEE 802.1x a pokročilejší funkce  \cite{Bouska2007} \\
	\rowcolor[rgb]{ .933,  .933,  .933} 5	&DHCP spoofing	&DHCP snooping, IPv6 Snooping, DHCPv6 Guard	&C	&C	&DIST
	COLDISTACC
	ACC	&Lan Switch Security \cite{Vyncke2008}
	
	Cisco Guide to Harden Cisco IOS Devices \cite{Singh2018} 
	
	IPv6 First-Hop Security Configuration Guide \cite{zXCpMaLbN1J7D1z2}\\
	6	&Příliš veľa DHCP paketov, zaplavenie DHCP paketmi	&Odmedziť počet DHCP paketov na nedôverihodných rozhraniach	&C	&M	&DIST
	COLDISTACC
	ACC	&Lan Switch Security \cite{Vyncke2008}
	
	Cisco Guide to Harden Cisco IOS Devices \cite{Singh2018}
	
	IPv6 First-Hop Security Configuration Guide \cite{zXCpMaLbN1J7D1z2}\\
	\rowcolor[rgb]{ .933,  .933,  .933} 7	&ARP Spoofing	&Dynamic ARP Inspection	&C	&C	&DIST
	COLDISTACC
	ACC	&Cisco CCNA Security Study Guide \cite{McMillan2018}\\
	8	&IP spoofing	&IPv4/IPv6 Source Guard	&C	&C	&DIST
	COLDISTACC
	ACC	&Cisco Guide to Harden Cisco IOS Devices \cite{Singh2018}
	
	IPv6 First-Hop Security Configuration Guide \cite{zXCpMaLbN1J7D1z2}\\
	\rowcolor[rgb]{ .933,  .933,  .933} 9	&IPv6 ND Spoofing	&IPv6 ND Inspection	&C	&C	&DIST
	COLDISTACC
	ACC	&Bezpečné IPv6: zkrocení zlých směrovačů \cite{Podermanski1222015}
	
	Bezpečné IPv6 : směrovač se hlásí \cite{Gregr522015}
	
	IPv6 First-Hop Security Configuration Guide \cite{zXCpMaLbN1J7D1z2}\\
	10	&Rogue RA,
	
	RA Flood,
	
	RouterLifeTime=0
	&RA Guard	&C	&C	&DIST
	COLDISTACC
	ACC	&Bezpečné IPv6: zkrocení zlých směrovačů \cite{Podermanski1222015}
	
	Bezpečné IPv6 : směrovač se hlásí \cite{Gregr522015}
	
	IPv6 First-Hop Security Configuration Guide \cite{zXCpMaLbN1J7D1z2}\\
	\rowcolor[rgb]{ .933,  .933,  .933} 11	&Mobilné zariadenia pripojené bezdôtovo spotrebovávajú veľa energie kvôli častým RA správam	&RA Throttling	&C	&L	&DIST
	COLDISTACC
	ACC	&Bezpečné IPv6: trable s multicastem \cite{Podermanski532015}
	
	ND on wireless links and/or with sleeping nodes \cite{o31nYG4kn98wWNRS}\\
	12	&Vyčerpanie cache susedov	&Statický záznam pre kritické zariadenia (servery) spájajúce IP a MAC adresu a VLAN
	&C	&C	&DIST
	COLDISTACC
	ACC	&Bezpečné IPv6: když dojde keš \cite{Podermanski1232015}
	
	Bezpečné IPv6: když dojde keš – obrana \cite{Podermanski1932015}
	\\
	\rowcolor[rgb]{ .933,  .933,  .933} 13	&Vyčerpanie cache susedov	&Na zabránenie vzdialeného útoku na cache susedov cez internet je potreba nastaviť ACL, kde povolujeme iba komunikáciu s cieľovými IPv6 adresami, ktoré sa nachádzajú v našej sieti	&C	&C	&CORE/EDGE
	COLCOREDIST
	COLALL	&Bezpečné IPv6: když dojde keš \cite{Podermanski1232015}
	
	Bezpečné IPv6: když dojde keš – obrana \cite{Podermanski1932015}
	\\
	14	&Vyčerpanie cache susedov	&IP destination Guard (First Hop Security)
	
	
	&C	&C	&DIST
	COLDISTACC
	ACC	&Bezpečné IPv6: když dojde keš \cite{Podermanski1232015}
	
	Bezpečné IPv6: když dojde keš – obrana \cite{Podermanski1932015}
	\\
	\rowcolor[rgb]{ .933,  .933,  .933} 15	&Vyčerpanie cache susedov	&Limitovanie času IPv6 adresy v cache susedov	&C	&C	&DIST
	COLDISTACC
	ACC	&Bezpečné IPv6: když dojde keš \cite{Podermanski1232015}
	
	Bezpečné IPv6: když dojde keš – obrana \cite{Podermanski1932015}\\ 
	\hline
	\caption{First Hop Security útoky a odporúčania}
	\label{tab:fhs}%
\end{longtable}%

\newpage
\begin{longtable}[!htbp]{|C{2.5em}L{9em}L{9em}C{2em}C{2.5em}C{7.5em}L{10em}|}
	
	\hline
	\centering
	
	Riadok č.	&Útok / Problém	&Mitigácia / Nastavenie	&\rot{Plane} 	&\rot{Severity}	&Facility layer	&Zdroj\\
	\endhead
	
	\rowcolor[rgb]{ .933,  .933,  .933} 1	&Rogue root bridge protection (root guard)	&Rogue root bridge 	&C	&C	&DIST
	COLDISTACC
	ACC	&Lan Switch Security \cite{Vyncke2008}\\
	2	&BPDU protection (BPDU guard)	&Pripojenie pripínaču na koncový prístupový port	&C	&C	&DIST
	COLDISTACC
	ACC	&Lan Switch Security \cite{Vyncke2008}\\
	\rowcolor[rgb]{ .933,  .933,  .933} 3	&Prístupové porty by sa nemali podielať na STP procese	&Rýchlosť konvergencie	&C	&H	&DIST
	COLDISTACC
	ACC	&Lan Switch Security \cite{Vyncke2008}\\
	4	&Špeciálne konfigurácie zaisťujúce bezslučkovú topológiu pomocou STP keď nastane jednosmerná komunikácia (Loop Guard)	&Jednosmerná komunikácia medzi prepínačmi môźe viesť k topoógií so slučkami	&C	&C	&DIST
	COLDISTACC
	ACC	&Designing Cisco network service architectures \cite{Tiso2012}\\
	
	\hline
	\caption{Odporúčania pre Spanning Tree Protokol}
	\label{tab:stp}%
\end{longtable}%

\begin{longtable}[!htbp]{|C{2.5em}L{9em}L{9em}C{2em}C{2.5em}C{7.5em}L{10em}|}
	
	\hline
	\centering
	
	Riadok č.	&Útok / Problém	&Mitigácia / Nastavenie	&\rot{Plane} 	&\rot{Severity}	&Facility layer	&Zdroj\\
	\endhead
	
	\rowcolor[rgb]{ .933,  .933,  .933}  1	&Špeciálna VLAN pre manažment na obmedzenie prístupu iba pre administrátorov	&Vytvorenie separátnej VLAN pre manažment	&C	&M	&DIST, COLDISTACC, ACC	&CCNA Routing and Switching Study Guide \cite{Lammle2013}\\
	2	&Útočníkovi s fyzickým prístupom k portu môže byť pridelený prístup do časti siete, ktorá zodpovedá príslušnej VLAN 	&Vytvorenie špeciálnej black hole VLAN pre nevyužité porty	&C	&C	&DIST, COLDISTACC, ACC	&Cisco SAFE Reference Guide \cite{uYLsMtQInofenpV3}\\
	\rowcolor[rgb]{ .933,  .933,  .933}  3	&Predvolenej VLAN je povolené prepnute na akýkoľvek port, VLAN hopping, double tagging	&Odobrať všetky porty z predvolenej VLAN	&C	&C	&DIST, COLDISTACC, ACC	&Cisco SAFE Reference Guide \cite{uYLsMtQInofenpV3}\\
	4	&Predvolenej VLAN je povolené byť prepnutá na akýkoľvek port, VLAN hopping, double tagging	&Vytvorenie natívnej VLAN rozdielnej ako predvolená, priradeni k trunk portu a povolenie iba potrebných portov	&C	&C	&DIST, COLDISTACC, ACC	&Cisco SAFE Reference Guide \cite{uYLsMtQInofenpV3}\\
	\rowcolor[rgb]{ .933,  .933,  .933}  5	&DTP útok, Switch spoofing útok	&Vypnutie dynamického trunkovacieho protokolu a explicitne určiť porty ako prístupové a trunk	&C	&C	&DIST, COLDISTACC, ACC	&Cisco SAFE Reference Guide \cite{uYLsMtQInofenpV3}\\
	6	&Nový prepínač s vyšším číslom revízie, ale s nesprávnou VLAN databázou môže šíriť falošné VLAN identifikátory a spôsobiť nefunkčnosť siete, veľa možnćh VTP útokov kvǒli zraniteľnostiam 	&Vypnutie MVRP. MRP, GARP, VTP po úspešnej propagácií VLAN	&C	&C	&DIST
	COLDISTACC
	ACC	&Lan Switch Security \cite{Vyncke2008}\\
	\rowcolor[rgb]{ .933,  .933,  .933}  7	&VTP musí byť používané	&Uprednostniť VTP verzie 3, špecifikovať skryté heslo a zapnúť VTP prunning pokiaľ musí byť VTP zapnuté	&C	&C	&DIST
	COLDISTACC
	ACC	&Lan Switch Security \cite{Vyncke2008}\\
	8	&Vysoké zaťaženie linky	&Poslanie notifikácie pri prekročení prahovej hodnoty zaťaženia linky	&C	&M	&A	&Cisco SAFE Reference Guide \cite{uYLsMtQInofenpV3}\\
	
	\hline
	\caption{Odporúčania pre VLAN}
	\label{tab:vlan}%
\end{longtable}%


\begin{longtable}[!htbp]{|C{2.5em}L{9em}L{9em}C{2em}C{2.5em}C{7.5em}L{10em}|}
	
	\hline
	\centering
	
	Riadok č.	&Útok / Problém	&Mitigácia / Nastavenie	&\rot{Plane} 	&\rot{Severity}	&Facility layer	&Zdroj\\
	\endhead
	
	\rowcolor[rgb]{ .933,  .933,  .933} 1	&Zlyhanie zariadenia alebo linky môže viest k nefunkčnosti siete 	&Povolenie FHRP s autentizáciou a aktuálnou verziou	&C	&M	&CORE/EDGE
	COLCOREDIST
	COLALL	&CCNA Routing and Switching Study Guide \cite{Lammle2013}\\
	2	&Nepoužívané, staré a nezabezpečené služby môžu byť použité na škodlivé účely	&Vypnutie nepoužívaných služieb z bezpečnostných dôvodov a na šetrenie CPU a pamäte 	& \footnotemark &H	&\footnotemark&Cisco IOS features that you should disable or restrict \cite{yDzYjF1hoACahpg1}\\
	\rowcolor[rgb]{ .933,  .933,  .933} 3	&Odpočúvanie komunikácie  cez nezabezpečené tunely	&Vypnúť tunely ktoré nie sú zabezpečené alebo zabezpečiť tunely	&D	&C	&CORE/EDGE
	DIST
	COLCOREDIST
	COLDISTACC
	COLALL	&CIS Cisco IOS 15 Benchmark \cite{CIS_DrTLsgXv24lxeIIM}
	
	Cisco router configuration handbook \cite{Hucaby2010}\\
	4	&Môže byť zneužité odpočúvanie pokiaľ sa používa monitorovanie prevádzky a monitorovanie prevádzky kvôli legislatívnym potrebám	&Monitorovanie výkonnosti siete a zber sieťového prenosu kvôli legislatívnym potrebám	&C	&N	&A	&Cisco Guide to Harden Cisco IOS Devices \cite{Singh2018}\\
	\rowcolor[rgb]{ .933,  .933,  .933} 5	&Útočník s fyzickým prístupom k zariadeniu alebo portu môže odpočúvať alebo posielať škodlivý obsah	&Explicitne zakázať nepoužívané porty	&D	&C	&A	&Cisco Router Hardening \cite{Graesser2001}
	
	Network Security Auditing \cite{Jackson2010}\\
	6	&Zdrojové rozhranie pre management a control protokoly	&Vytvorť Loopback rozhranie s IP adresou	&M/C	&M	&A	&
	Cisco Guide to Harden Cisco IOS Devices \cite{Singh2018}
	
	CIS Cisco IOS 15 Benchmark \cite{CIS_DrTLsgXv24lxeIIM}\\
	\rowcolor[rgb]{ .933,  .933,  .933} 7	&Pretečenie pamäte	&Povoliť mechanizmy na detekciu pretečenia pamäte	&M	&M	&A	&Cisco Guide to Harden Cisco IOS Devices \cite{Singh2018}\\
	8	&Načítanie škodlivej konfigurácie zo siete počas bootovania	&Vypnutie načítania operačného systému alebo konfigurácie zo siete pokiaľ to nie je nutné	&M	&M	&A	&Hardening Cisco Routers \cite{Akin2002}\\
	\rowcolor[rgb]{ .933,  .933,  .933} 9	&Odpočuvanie konfigurácií zariadení pri zálohe	&Zapnutie zabezpečenej zálohy na server (SFTP, SCP)	&M	&H	&A	&Cisco Guide to Harden Cisco IOS Devices \cite{Singh2018}\\
	10	&Vymazanie konfigurácie	&Zapnutie ochrany pred výmazom konfigurácie	&M	&H	&A	&Cisco CCNA Security Study Guide \cite{McMillan2018}\\
	\rowcolor[rgb]{ .933,  .933,  .933} 11	&Možnosť urobiť diff zmien konfigurácií a jej návrat	&Periodické zálohovanie konfigurácie a logovanie jej zmien	&M	&M	&A	&Cisco CCNA Security Study Guide \cite{McMillan2018}
	
	Cisco Guide to Harden Cisco IOS Devices \cite{Singh2018}\\
	12	&Nemožnosť prihlásenia pri zaseknutom TCP spojení	&Terminovanie zaseknutého TCP spojenia	&M	&M	&A	&Cisco Guide to Harden Cisco IOS Devices \cite{Singh2018}\\
	\rowcolor[rgb]{ .933,  .933,  .933} 13	&Možnosť prečítať heslá z uniknutých konfigurácií	&Zašifrovanie hesiel v otvorenej podobe	&M	&C	&A	&CIS Cisco IOS 15 Benchmark \cite{CIS_DrTLsgXv24lxeIIM}\\
	
	\hline
	\caption{Ostatné nezatriedené odporúčania}
	\label{tab:other}%
\end{longtable}%
\footnotetext{Záleží na výrobcovi, operačnom systéme a verzii}






\restoregeometry 


