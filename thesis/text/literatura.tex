% Pro sazbu seznamu literatury použijte jednu z následujících možností

%%%%%%%%%%%%%%%%%%%%%%%%%%%%%%%%%%%%%%%%%%%%%%%%%%%%%%%%%%%%%%%%%%%%%%%%%
%1) Seznam citací definovaný přímo pomocí prostředí literatura / thebibliography

\begin{literatura}{99}
\bibitem{Milkovich3122018}
MILKOVICH, Devon. 13 Alarming Cyber Security Facts and Stats. In: \textit{Cybint} [online]. 3.12.2018 [cit. 2019-11-08]. Dostupné z: https://www.cybintsolutions.com/cyber-security-facts-stats/

\bibitem{Vyncke2008}
VYNCKE, Eric a Christopher PAGGEN. \textit{LAN switch security: What hackers know about your switches}. Indianapolis, IN: Cisco Press, 2008. ISBN :978-1-58705-256-9.	
	
\bibitem{McMillan2018}
MCMILLAN, Troy. \textit{CCNA security study guide: exam 210-260}. Indianapolis, Indiana: Sybex, a Wiley Brand, 2018. ISBN 978-111-9409-939.
	
\bibitem{Stallings2011}
STALLINGS, William. \textit{Network security essentials: applications and standards}. 4th ed. Boston: Prentice Hall, 2011. ISBN 978-0-13-610805-4.

\bibitem{Jackson2010}
JACKSON, Chris. \textit{Network security auditing}. Indianapolis, IN: Cisco Press, 2010. Cisco Press networking technology series. ISBN 978-1-58705-352-8.

\bibitem{7TVhmfuQFbsOANAz}
Guide for Conducting Risk Assessments: NIST Special Publication 800-30. In: \textit{NIST} [online]. 2012 [cit. 2019-11-08]. Dostupné z: https://nvlpubs.nist.g\
ov/nistpubs/Legacy/SP/nistspecialpublication800-30r1.pdf

\bibitem{Singh2018}
SINGH, Shashank. Cisco Guide to Harden Cisco IOS Devices. In: \textit{Cisco} [online]. 2018 [cit. 2019-11-02]. Dostupné z: https://www.cisco.com/c/en/us/support/docs/ip/access-lists/13608-21.html	

\bibitem{Lammle2013}
LAMMLE, Todd. \textit{CCNA: routing and switching : study guide}. Indianapolis,\
Indiana: SYBEX, [2013]. ISBN 978-1-118-74961-6.
	
\bibitem{Pepelnjak2013}
PEPELNJAK, Ivan. Management, Control and Data Planes in Network Devices and Systems. In: \textit{IpSpace} [online]. 2013 [cit. 2019-11-17]. Dostupné z: https://blog.ipspace.net/2013/08/management-control-and-data-planes-in.html
	






	
	
	
\bibitem{Alsadeh1252015}
ALSADEH, Ahmad. Augmented SEND: Aligning Security, Privacy, and Usability. In: \textit{RIPE NCC} [online]. 12.5.2015 [cit. 2019-11-02]. Dostupné z: https://ripe70.ripe.net/presentations/67-RIPE70-SEND.pdf
\bibitem{Podermanski1222015}
PODERMAŃSKI, Tomáš a Matěj GRÉGR. Bezpečné IPv6: zkrocení zlých směrovačů. In: \textit{ROOT.CZ} [online]. 12.2.2015 [cit. 2019-11-02]. Dostupné z: https://www.root.cz/clanky/bezpecne-ipv6-zkroceni-zlych-smerovacu/
\bibitem{Khandelwal2016}
KHANDELWAL, Manjul. OSPF Security: Attacks and Defenses. In: \textit{SANOG} [online]. 2016 [cit. 2019-11-04]. Dostupné z: https://www.sanog.org/resources/sanog28/SANOG28-Tutorial\_OSPF-Security-Attacks-and-Defences-Manjul.pdf
\bibitem{Podermanski1932015}
PODERMAŃSKI, Tomáš a Matěj GRÉGR. Bezpečné IPv6: když dojde keš --- obrana. In: \textit{ROOT.CZ} [online]. 19.3.2015 [cit. 2019-11-02]. Dostupné z: https://www.root.cz/clanky/bezpecne-ipv6-kdyz-dojde-kes-obrana/
\bibitem{Podermanski1232015}
PODERMAŃSKI, Tomáš a Matěj GRÉGR. Bezpečné IPv6: když dojde keš. In: \textit{ROOT.CZ} [online]. 12.3.2015 [cit. 2019-11-02]. Dostupné z: https://www.root.cz/clanky/bezpecne-ipv6-kdyz-dojde-kes/
\bibitem{Podermanski532015}
PODERMAŃSKI, Tomáš a Matěj GRÉGR. Bezpečné IPv6: trable s multicastem. In: \textit{ROOT.CZ} [online]. 5.3.2015 [cit. 2019-11-02]. Dostupné z: https://www.root.cz/clanky/bezpecne-ipv6-trable-s-multicastem/
\bibitem{Gregr2622015}
GRÉGR, Matěj a Tomáš PODERMAŃSKI. Bezpečné IPv6: vícehlavý útočník. In: \textit{ROOT.CZ} [online]. 26.2.2015 [cit. 2019-11-02]. Dostupné z: https://www.root.cz/clanky/bezpecne-ipv6-vicehlavy-utocnik/
\bibitem{Podermanski1922015}
PODERMAŃSKI, Tomáš a Matěj GRÉGR. Bezpečné IPv6: trable s hlavičkami. In: \textit{ROOT.CZ} [online]. 19.2.2015 [cit. 2019-11-02]. Dostupné z: https://www.root.cz/clanky/bezpecne-ipv6-trable-s-hlavickami/
\bibitem{Gregr522015}
GRÉGR, Matěj a Tomáš PODERMAŃSKI. Bezpečné IPv6 : směrovač se hlásí. In: \textit{ROOT.CZ} [online]. 5.2.2015 [cit. 2019-11-02]. Dostupné z: https://www.root.cz/clanky/bezpecne-ipv6-smerovac-se-hlasi/
\bibitem{zXCpMaLbN1J7D1z2}
IPv6 First-Hop Security Configuration Guide. In: \textit{Cisco} [online]. San Jose [cit. 2019-11-02]. Dostupné z: https://www.cisco.com/c/en/us/td/docs/ios-xml/ios/ipv6\_fhsec/configuration/15-1sg/ip6f-15-1sg-book.pdf
\bibitem{Bouska2007}
BOUŠKA, Petr. \textit{Cisco IOS 12 - IEEE 802.1x a pokročilejší funkce} [online]. In: . 2007 [cit. 2019-11-02]. Dostupné z: https://www.samuraj-cz.com/clanek/cisco-ios-12-ieee-802-1x-a-pokrocilejsi-funkce/
\bibitem{yDzYjF1hoACahpg1}
MOLENAAR, René. Cisco IOS features that you should disable or restrict. In: \textit{NetworkLessons.com} [online]. [cit. 2019-11-02]. Dostupné z: https://networklessons.com/uncategorized/cisco-ios-features-that-you-should-disable-or-restrict
\bibitem{Bouska2009}
BOUŠKA, Petr. Cisco IOS 23 - Autentizace uživatele na switchi vůči Active Directory. In: \textit{SAMURAJ-cz} [online]. 2009 [cit. 2019-11-02]. Dostupné z: https://www.samuraj-cz.com/clanek/cisco-ios-23-autentizace-uzivatele-na-switchi-vuci-active-directory/
\bibitem{Barker2019}
BARKER, Elaine a Allen ROGINSKY. Transitioning the Use of Cryptographic Algorithms and Key Lengths. In: \textit{NIST} [online]. 2019 [cit. 2019-11-02]. Dostupné z: https://nvlpubs.nist.gov/nistpubs/SpecialPublications/NIST.SP.800-131Ar2.pdf
\bibitem{o31nYG4kn98wWNRS}
VYNCKE, Erik. ND on wireless links and/or with sleeping nodes. In: \textit{IETF} [online]. [cit. 2019-11-02]. Dostupné z: https://www.ietf.org/proceedings/89/slides/slides-89-v6ops-3.pdf
\bibitem{DrTLsgXv24lxeIIM}
CIS Cisco IOS 15 Benchmark. In: \textit{Center For Internet Security} [online]. 2015 [cit. 2019-11-02]. Dostupné z: https://www.cisecurity.org/benchmark/cisco/

\bibitem{Graesser2001}
GRAESSER, Dana. Cisco Router Hardening Step-by-Step. In: \textit{SANS Institute} [online]. 2001 [cit. 2019-11-02]. Dostupné z: https://www.sans.org/reading-room/whitepapers/firewalls/paper/794
\bibitem{Pilihanto2012}
PILIHANTO, Atik. A Complete Guide on IPv6 Attack and Defense. In: \textit{SANS Institute} [online]. SANS Institute, 2012 [cit. 2019-11-02]. Dostupné z: https://www.sans.org/reading-room/whitepapers/detection/paper/33904
\bibitem{Rey2016}
REY, Enno, Antonios ATLASIS a Jayson SALAZAR. MLD Considered Harmful. In: \textit{RIPE NCC} [online]. 2016 [cit. 2019-11-02]. Dostupné z: https://ripe72.ripe.net/presentations/74-ERNW\_RIPE72\_MLD\_Considered\_Harmful\_v1\_light\_web.pdf
\bibitem{Vyncke2012}
VYNCKE, Erik. IPv6 First Hop Security: the IPv6 version of DHCP snooping and dynamic ARP inspection. In: \textit{Slidde Share} [online]. 2012 [cit. 2019-11-02]. Dostupné z: https://www.slideshare.net/IKTNorge/eric-vyncke-layer2-security-ipv6-norway
\bibitem{77Eg8gGc0CKWfGBi}
IPv6 First-Hop Security Configuration Guide. In: \textit{Cisco} [online]. 2012 [cit. 2019-11-02]. Dostupné z: https://www.cisco.com/c/en/us/td/docs/ios-xml/ios/ipv6\_fhsec/configuration/15-s/ip6f-15-s-book/ip6-snooping.html
\bibitem{Gregr2011}
GREGR, Matej, Petr MATOUSEK, Miroslav SVEDA a Tomas PODERMANSKI. Practical IPv6 monitoring-challenges and techniques. In: \textit{12th IFIP/IEEE International Symposium on Integrated Network Management (IM 2011) and Workshops}. IEEE, 2011, 2011, s.~650-653. DOI: 10.1109/INM.2011.5990647. ISBN 978-1-4244-9219-0. Dostupné také z: http://ieeexplore.ieee.org/document/5990647/
\bibitem{1xYhFLUJF9lmHMC0}
PODERMAŃSKI, Tomáš a Matějj GRÉGR. \textit{Deploying IPv6 - practical problems from the campus perspective} [online]. In: . [cit. 2019-11-02].
\bibitem{Martin2016}
MARTIN, Tim. IPv6 Sys Admin Style. In: \textit{SlideShare} [online]. 2016 [cit. 2019-11-02]. Dostupné z: https://www.slideshare.net/tjmartin2020/ipv6-sysadmins-63071235
\bibitem{uYLsMtQInofenpV3}
Cisco SAFE Reference Guide. In: \textit{CIsco} [online]. San Jose, CA, 8 Júl 2018 [cit. 2019-11-02]. Dostupné z: https://www.cisco.com/c/en/us/td/docs/solutions/Enterprise/Security/SAFE\_RG/SAFE\_rg.pdf
\bibitem{JnCqiekTXFe2KIyx}
SAFE Overview Guide: Threats, Capabilities, and the Security Reference Architecture. In: \textit{Cisco} [online]. Január 2018 [cit. 2019-11-02]. Dostupné z: https://www.cisco.com/c/dam/en/us/solutions/collateral/enterprise/design-zone-security/safe-overview-guide.pdf

\bibitem{Akin2002}
AKIN, Thomas. \textit{Hardening Cisco routers}. Sebastopol: O'Reilly, 2002. ISBN 05-960-0166-5.

\bibitem{Hucaby2010}
HUCABY, Dave, Steve MCQUERRY, Andrew WHITAKER a Dave HUCABY. \textit{Cisco router configuration handbook}. 2nd ed. Indianapolis, IN: Cisco Press, 2010. ISBN 978-1-58714-116-4.
\bibitem{Satrapa2019}
SATRAPA, Pavel. \textit{IPv6: internetový protokol verze 6}. 4. aktualizované a rozšířené vydání. Praha: CZ.NIC, z.s.p.o., 2019. CZ.NIC. ISBN 978-808-8168-430.



\end{literatura}


%%%%%%%%%%%%%%%%%%%%%%%%%%%%%%%%%%%%%%%%%%%%%%%%%%%%%%%%%%%%%%%%%%%%%%%%%
%%2) Seznam citací pomocí BibTeXu
%% Při použití je nutné v TeXnicCenter ve výstupním profilu aktivovat spouštění BibTeXu po překladu.
%% Definice stylu seznamu
%\bibliographystyle{unsrturl}
%% Pro českou sazbu lze použít styl czechiso.bst ze stránek
%% http://www.fit.vutbr.cz/~martinek/latex/czechiso.tar.gz
%%\bibliographystyle{czechiso}
%% Vložení souboru se seznamem citací
%\bibliography{text/literatura}
%
%% Následující příkaz je pouze pro ukázku sazby literatury při použití BibTeXu.
%% Způsobí citaci všech zdrojů v souboru odkazy.bib, i když nejsou citovány v textu.
%\nocite{*}