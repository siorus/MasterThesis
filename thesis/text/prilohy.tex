\newgeometry{left=2.5cm,bottom=3.4cm, top=2.5cm}
\chapter{Kontrolný zoznam odporúčaní pre zariadenia CISCO}
\label{apendix:cisco_cmd}

\scriptsize

\begin{longtable}[!htbp]{|L{10em}L{10em}>{\fontfamily{qcr}\selectfont}L{34em}|}
	\caption{Rozpracovaná tabuľka s príkazmi na konfiguráciu zariadení od spoločnosti Cisco}
	\label{tab:cisco_table}\\ \hline
	\centering 
	
	\centering{\hspace{-3em}\vspace{1em}Útok / Problém} & Mitigácia / Nastavenie&{\fontfamily{lmr}\selectfont Príkazy}\\ \hhline{===}
	\endfirsthead
	
	
	\hline
	\centering 
	\centering{\hspace{-3em}Útok / Problém} & Mitigácia / Nastavenie&{\fontfamily{lmr}\selectfont Príkazy}\\ \hhline{===}
	\endhead
	
	
	
	\rowcolor[rgb]{ .94,  .94,  .94} Nemožná identifikácia zariadenia	&Vytvoriť hostname	&hostname $<$hostname$>$\\
	Nemožnosť vzdialeného prístupu	&Vytvoriť doménové meno	&ip domain-name $<$domain$>$\\
	
	\rowcolor[rgb]{ .94,  .94,  .94} Nepovolený prístup k manažovaniu zariadenia	&Vytvoriť a aplikovať ACL pre Telnet, SSH a pod. a zaznamenať v logu prístupy	&ip access-list standard $<$acl name$>$
	
	\hspace{0.5em}remark permit specifi ip and log
	
	\hspace{0.5em}permit $<$ip address$>$ $<$mask$>$ log-input
	
	\hspace{0.5em}remark deny other and log
	
	\hspace{0.5em}deny any log-input
	
	\vspace{0.5em}
	
	ipv6 access-list $<$acl name$>$
	
	\hspace{0.5em}remark permit specifi ip and log
	
	\hspace{0.5em}permit $<$ipv6 address$>$/$<$prefix$>$ any log-input
	
	\hspace{0.5em}remark deny other and log
	
	\hspace{0.5em}deny any any log-input
	
	\vspace{0.5em}
	{\fontfamily{lmr}\selectfont alebo v global config }
	\vspace{0.5em}
	
	login on-failure log-input
	
	login on-failure trap
	
	login on-failure
	
	login on-success log-input
	
	login on-success trap
	
	login on-success\\
	
	
	
	
	Nepovolený prístup k manažovaniu zariadenia	&Vytvoriť a aplikovať ACL pre Telnet, SSH a pod. a zaznamenať v logu prístupy	&line vty $<$num$>$ $<$num$>$
	
	\hspace{0.5em}ip access-class $<$acl name$>$ in
	
	\hspace{0.5em}ipv6 access-class $<$acl name$>$ in
	
	\vspace{0.5em}line tty $<$num$>$ $<$num$>$
	
	\hspace{0.5em}ip access-class $<$acl name$>$ in
	
	\hspace{0.5em}ipv6 access-class $<$acl name$>$ in\\
	
	
	
	
	
	\rowcolor[rgb]{ .94,  .94,  .94}Neautorizovaný prístup cez nepoužívané a nezabezpečené protokoly na manažment zariadení	&Vypnúť nepoužívané protokoly na prístup k manažovaniu zariadení (telnet a pod.)	&line aux 0
	
	\hspace{0.5em}no exec
	
	\hspace{0.5em}transport input none\\
	
	
	
	
	Prístup bez požadovaných prístupových údajov	&Nakonfigurovanie protokolov na manažment zariadení, aby požadovali prístupové údaje (telnet a pod.)	&line vty $<$num$>$ $<$num$>$
	\hspace{0.5em}password
	
	\hspace{0.5em}login | login local
	
	\vspace{0.5em}line tty $<$num$>$ $<$num$>$
	
	\hspace{0.5em}password
	
	\hspace{0.5em}login | login local
	
	\vspace{0.5em}
	line con $<$num$>$
	
	\hspace{0.5em}password
	
	\hspace{0.5em}login | login local
	
	\vspace{0.5em}line aux $<$num$>$
	
	\hspace{0.5em}password
	
	\hspace{0.5em}login | login local\\
	
	
	
	
	\rowcolor[rgb]{ .94,  .94,  .94}Nepoužívanie zabezpečeného protokolu na manažment zariadení môže viesť k odposluchu	&Zapnutie SSH	&line vty $<$num$>$ $<$num$>$
	
	\hspace{0.5em}login local
	
	\hspace{0.5em}transport input ssh\\
	
	
	
	Nebezpečná verzia 1 protokolu SSH	&SSH verzia 2	&ip ssh version 2\\
	
	
	
	
	\rowcolor[rgb]{ .94,  .94,  .94}Dlhé neaktívne sedenie môže byť zneužité alebo aj fyzický prístup útočníka k aktívnemu sedeniu môže viesť k zmene konfigurácie	&SSH čas vypršania sedenia	&ip ssh timeout $<$timeout seconds$>$\\
	
	
	
	
	Útok na krátky RSA kľúč	&Dĺžka RSA kľúča minimálne 2048 bitov	&crypto key generate rsa modulus 2048\\
	
	
	
	
	\rowcolor[rgb]{ .94,  .94,  .94}Hádanie hesla k RSA kľúču	&SSH maximálny počet neúspešných pokusov	&ip 
	ssh authentication-retries $<$max num$>$\\
	
	
	
	
	Útok hrubou silou na zistenie prihlasovacích údajov	&Špecifikovať čas po ktorý nie je možné po N pokusoch sa prihlásiť	&login block-for 60 attempts 3 within 30\\
	
	
	
	
	\rowcolor[rgb]{ .94,  .94,  .94} Prihlásenie na zariadenie nie je možné kvôli zablokovaniu pre príliš veľa neúspešných pokusov	&Povolenie prístupu administrátorovi na základe IP adresy, keď je protokol na manažovanie zariadení nedostupný kvôli DOS útoku	&login quiet-mode access-class $<$acl name$>$\\
	
	
	
	Možné prihlásenie do zariadenia cez telnet keď je prítomné SSH	&Zakázať telnet ak je SSH aktívne	&line vty $<$num$>$ $<$num$>$
	
	\hspace{0.5em}no transport input all
	
	\hspace{0.5em}no transport input telnet
	\vspace{0.5em}
	
	line tty $<$num$>$ $<$num$>$
	
	\hspace{0.5em}no transport input all
	
	\hspace{0.5em}no transport input telnet
	\vspace{0.5em}
	
	line con $<$num$>$
	
	\hspace{0.5em}no transport input all
	
	\hspace{0.5em}no transport input telnet
	\vspace{0.5em}
	
	line aux $<$num$>$
	
	\hspace{0.5em}no transport input all
	
	\hspace{0.5em}no transport input telnet\\
	
	
	
	
	\rowcolor[rgb]{ .94,  .94,  .94} Útočník nie je informovaný o právnych následkoch	&Právne upozornenie pri prístupe k zariadeniu	&banner motd
	
	banner login
	
	banner exec\\
	
	
	
	Dlhé neaktívne sedenie môže byť zneužité alebo aj fyzický prístup útočníka k aktívnemu sedeniu môže viesť k zmene konfigurácie	&Čas vypršania sedenia pre protokol na manažovanie zariadení	&line vty $<$num$>$ $<$num$>$
	
	exec-timeout 5
	\vspace{0.5em}
	
	line tty $<$num$>$ $<$num$>$
	
	\hspace{0.5em}exec-timeout 5
	\vspace{0.5em}
	
	line con $<$num$>$
	
	\hspace{0.5em}exec-timeout 5
	\vspace{0.5em}
	
	line aux $<$num$>$
	
	\hspace{0.5em}exec-timeout 5\\
	
	
	
	
	\rowcolor[rgb]{ .94,  .94,  .94} Možnosť prečítať heslá z uniknutých konfigurácií	&Zašifrovanie hesiel v otvorenej podobe	&service password-encryption\\
	
	
	
	
	Nepovolená zmena konfigurácie zariadenia	&Vytvorenie hesla na editovanie konfigurácie zariadenia	&enable secret $<$secret password$>$\\
	
	
	
	
	\rowcolor[rgb]{ .94,  .94,  .94} Nepovolená zmena konfigurácie zariadenia	&Vytvorenie hesla na editovanie konfigurácie zariadenia	&no enable password $<$password$>$\\
	
	
	
	Nepovolený prístup k manažmentu konfigurácie zariadenia	&Lokálne zabezpečené účty	&username secret $<$username$>$ $<$secret password$>$\\
	
	
	
	\rowcolor[rgb]{ .94,  .94,  .94} Nepovolený prístup k manažmentu konfigurácie zariadenia	&Lokálne zabezpečené účty	&no username password  $<$username$>$ $<$password$>$\\
	
	
	
	Centrálna správa prihlásení a dohľadateľnosť zmien v konfigurácií	&Definovanie a povolenie AAA serveru na prihlásenie a definovanie záložného prihlásenia	&aaa new-model
	
	radius server $<$radius server name$>$
	
	\hspace{0.5em}address ipv4 $<$ip adddress$>$ / address ipv6 $<$ipv6 adddress$>$
	
	\hspace{0.5em}key $<$password$>$
	\vspace{0.5em}
	
	{\fontfamily{lmr}\selectfont alebo}
	
	\vspace{0.5em}
	radius-server host $<$ip adddress$>$
	
	radius-server key $<$password$>$
	
	\vspace{0.5em}
	
	aaa group server radius $<$radius group$>$
	
	\hspace{0.5em}server name $<$radius server name$>$
	
	aaa authentication login default / $<$radius login$>$ group 
	
	\hspace{0.5em}$<$radius group$>$ local enable
	
	line tty $<$num$>$ $<$num$>$
	
	\hspace{0.5em}login authentication default / $<$radius login$>$
	
	line vty $<$num$>$ $<$num$>$
	
	\hspace{0.5em}login authentication default / $<$radius login$>$
	
	line con $<$num$>$
	
	\hspace{0.5em}login authentication default / $<$radius login$>$
	
	line aux $<$num$>$
	
	
	\hspace{0.5em}login authentication default / $<$radius login$>$\\
	
	
	
	
	\rowcolor[rgb]{ .94,  .94,  .94} Centrálna správa prihlásení a dohľadateľnosť zmien v konfigurácií	&Definovanie a povolenie AAA serveru na prihlásenie a definovanie záložného prihlásenia	&aaa new-model
	
	tacacs server $<$tacacs server name$>$
	
	\hspace{0.5em}address ipv4 $<$ip adddress$>$ / address ipv6 $<$ipv6 adddress$>$
	
	\hspace{0.5em}key $<$password$>$
	\vspace{0.5em}
	
	{\fontfamily{lmr}\selectfont alebo}
	
	\vspace{0.5em}
	tacacs-server host $<$ip adddress$>$
	
	tacacs-server key $<$password$>$
	
	\vspace{0.5em}
	
	aaa group server tacacs $<$tacacs group$>$
	
	\hspace{0.5em}server name $<$tacacs server name$>$
	
	aaa authentication login default / $<$tacacs login$>$ group 
	
	\hspace{0.5em}$<$tacacs group$>$ local enable
	
	line tty $<$num$>$ $<$num$>$
	
	\hspace{0.5em}login authentication default / $<$tacacs login$>$
	
	line vty $<$num$>$ $<$num$>$
	
	\hspace{0.5em}login authentication default / $<$tacacs login$>$
	
	line con $<$num$>$
	
	\hspace{0.5em}login authentication default / $<$tacacs login$>$
	
	line aux $<$num$>$
	
	
	\hspace{0.5em}login authentication default / $<$tacacs login$>$\\
	
	
	
	
	Centrálna správa prihlásení a dohľadateľnosť zmien v konfigurácií	&Definovanie a povolenie AAA serveru na editáciu konfigurácií a definovanie záložného prihlásenia	&aaa authentication enable default group 
	
	\hspace{0.5em}$<$radius group$>$ enable\\
	
	
	
	
	\rowcolor[rgb]{ .94,  .94,  .94} Centrálna správa prihlásení a dohľadateľnosť zmien v konfigurácií	&Definovanie a povolenie AAA serveru na editáciu konfigurácií a definovanie záložného prihlásenia	&no aaa authentication enable default enable\\
	
	
	
	
	Hádanie prístupových údajov	&Definovanie maximálneho počtu neúspešných pokusov o prihlásenie a následné zablokovanie účtu	&aaa authentication attempts login 3\\
	
	
	
	
	\rowcolor[rgb]{ .94,  .94,  .94} Prihlásenie bez prihlasovacích údajov	&Zakázať záložné prihlásenie bez poskytnutia autentizačných prostriedkov	&{\fontfamily{lmr}\selectfont vyhnúť sa} aaa authentication login.*none.*\\
	
	
	
	
	AAA používa primárne lokálne účty namiesto centralizovaných na serveri	&AAA nesmie používať ako prvú možnosť prihlásenia lokálny účet 	&{\fontfamily{lmr}\selectfont vyhnúť sa} authentication login default local\\
	
	
	
	
	\rowcolor[rgb]{ .94,  .94,  .94} Používateľ prihlásený do zariadenia môže spúšťať akékoľvek príkazy	&Nastavenie AAA autorizácie pre spúšťanie príkazov. V prípade výpadku AAA serveru, bude užívateľ odhlásený a následne prihlásený podľa  záložného prihlásenia, aby mu nebolo pridelené vysoké oprávnenie umožňujúce vykonávať príkazy, na ktoré nemá právo	&aaa authorization exec $<$radius login$>$ group $<$radius group$>$ 
	
	\hspace{0.5em}local if-authenticated\\
	
	
	
	
	Používateľ prihlásený do zariadenia môže spúšťať akékoľvek príkazy	&Nastavenie AAA autorizácie pre spúšťanie príkazov. V prípade výpadku AAA serveru, bude užívateľ odhlásený a následne prihlásený podľa  záložného prihlásenia, aby mu nebolo pridelené vysoké oprávnenie umožňujúce vykonávať príkazy, na ktoré nemá právo	&aaa authorization commands 15 $<$radius login$>$ group
	
	\hspace{0.5em} $<$radius group$>$ local if-authenticated \\
	
	
	
	
	\rowcolor[rgb]{ .94,  .94,  .94} Administrátor vloží zlý príkaz a po čase je ho nemožné dohľadať a zjednať nápravu	&Nastavenie AAA účtovania respektíve logovania pripojení a vykonaných príkazov	&aaa accounting connection
	
	aaa accounting commands
	
	aaa accounting exec\\
	
	
	
	
	Odpočúvanie SNMP verzie 1 a 2c	&Použitie SNMP verzie 3 pokiaľ je SNMP používané	&no snmp-server community
	no snmp-server host  version 1/2c
	
	snmp-server group $<$group name$>$ v3 priv \\
	
	
	
	
	\rowcolor[rgb]{ .94,  .94,  .94} AAA zdrojové rozhranie nie je rovnaké pri každom reštarte	&Definovanie loopback zdrojového rozhrania pre AAA	&ip radius source interface loopback $<$id$>$
	
	ip tacacs source interface loopback $<$id$>$\\
	
	
	
	
	Modifikovanie konfigurácie pomocou SNMP	&Obmedzenie SNMP iba na čítanie	&snmp-server view $<$view name$>$ iso included
	
	snmp-server group $<$group name$>$ v3 priv read $<$view name$>$\\
	
	
	
	
	\rowcolor[rgb]{ .94,  .94,  .94} Neoprávnený prístup k SNMP informáciám	&Obmedzenie SNMP iba pre vybrané IP adresy	&ip access-list standard $<$acl name$>$
	
	\hspace{0.5em}remark permit only this IP 
	
	\hspace{0.5em}permit $<$ip address$>$ $<$wildcard mask$>$
	
	\hspace{0.5em}deny any log-input
	\vspace{0.5em}
	
	ipv6 access-list $<$acl name$>$
	
	\hspace{0.5em}remark permit only this IP 
	
	\hspace{0.5em}permit $<$ipv6 address$>$/$<$prefix$>$ any
	
	\hspace{0.5em}remark deny other
	
	\hspace{0.5em}deny any any log-input
	
	snmp-server group $<$group name$>$ v3 priv read $<$view name$>$  access $<$acl name$>$\\
	
	
	
	
	Administrátor nemá povedomie o problémoch na zariadení	&Povolenie asynchrónnych správ SNMP TRAP	&snmp-server host $<$ip adddress$>$ traps version 3 priv $<$user$>$
	
	snmp-server host $<$ip adddress$>$ version 3 priv $<$user$>$\\
	
	
	
	
	\rowcolor[rgb]{ .94,  .94,  .94} Odpočúvanie SNMP sedenie z dôvodu slabého šifrovania a hashovacej  funkcie	&Vytvorenie SNMP verzie 3 užívateľa s minimálnym šifrovaním AES 128 bit a hashovacou funkciou SHA	&snmp-server user $<$user$>$ $<$group name$>$ v3 auth sha 
	
	\hspace{0.5em}$<$password$>$ pri aes 128 $<$password$>$\\
	
	
	
	
	Sťažená identifikácia SNMP správ z rôznych IP	&Definovanie lokácie SNMP serveru	&snmp-server location $<$location$>$\\
	
	
	
	
	\rowcolor[rgb]{ .94,  .94,  .94} SNMP zdrojové rozhranie nie je rovnaké pri každom reštarte	& Definovanie loopback zdrojového rozhrania pre SNMP	&snmp-server trap-source loopback $<$id$>$\\
	
	
	
	
	Zmeny názvov rozhraní medzi reštartami a nemožnosť monitorovanie pomocou SNMP	&SNMP statické nemenné meno rozhrania aj po reštarte zariadenia	&snmp-server ifindex persist\\
	
	
	
	
	\rowcolor[rgb]{ .94,  .94,  .94} Administrátor nemá povedomie o problémoch na zariadení	&Povolenie logovania protokolom SYSLOG a špecifikovanie IP adresy SYSLOG serveru	&logging on
	logging host $<$ip adddress$>$\\
	
	
	
	
	Neprijímanie všetkých dôležitých incidentov na zariadení z protokolu SYSLOG	&Špecifikovanie dôležitosti oznámení SYSLOG na INFORMATIONAL	&logging trap informational\\
	
	
	
	
	\rowcolor[rgb]{ .94,  .94,  .94} SYSLOG zdrojové rozhranie nie je rovnaké pri každom reštarte	& Definovanie loopback zdrojového rozhrania pre SYSLOG	&logging source-interface loopback $<$id$>$\\
	
	
	
	Nedostatočné a neštandardné formáty času pri logovacích správach	&Definovanie formátu času pre logovacie a ladiace výstupy	&service timestamp log datetime
	
	service timestamp debug datetime\\
	
	
	
	
	\rowcolor[rgb]{ .94,  .94,  .94} Administrátor nevidí dôležité incidenty pri prihlásení a konfigurovaní cez konzolu	&Vypisovanie SYSLOG správ CRITICAL a dôležitejších do terminálu	&logging console critical\\
	
	
	
	
	Malá vyrovnávacia pamäť pre SYSLOG je dôvodom zahadzovanie správ	&Definovanie veľkosti SYSLOG buffera dôležitosti oznámení na INFORMATIONAL	&logging buffered 64000 6\\
	
	
	
	
	\rowcolor[rgb]{ .94,  .94,  .94} Neprístupný SYSLOG server spôsobuje zahadzovanie dôležitých syslog správ	&Definovanie dočasného úložiska SYSLOG správ v prípade nedostupnosti servera	&logging persistent url flash:/syslog\\
	
	
	
	
	Skenovanie a zistenie informácií o sieti za pomoci protokolu CDP a využitie bezpečnostných chýb	&Zakázanie protokolu CDP	&no cdp run
	
	
	interface $<$interface name$>$ $<$interface id$>$ 
	
	\hspace{0.5em}no cdp enable\\
	
	
	
	
	\rowcolor[rgb]{ .94,  .94,  .94} Skenovanie a zistenie informácií o sieti za pomoci protokolu LLDP a využitie bezpečnostných chýb	&Zakázanie protokolu LLDP	&no lldp run
	
	interface $<$interface name$>$ $<$interface id$>$	
	
	\hspace{0.5em}no lldp receive 
	
	\hspace{0.5em}no lldp transmit\\
	
	
	
	
	Nekonzistencia časov v logoch a problém pričlenenia logov k relevantným incidentom	&Nastavenie NTP serveru pre aktuálny čas v logoch	&ntp server $<$ip adddress$>$\\
	
	
	
	
	\rowcolor[rgb]{ .94,  .94,  .94} Pripojenie servera s rovnakou IP adresou, ale falošným časom	&Nastavenie NTP autentizácie	&ntp authenticate
	
	ntp authentication-key 1 md5 $<$password$>$
	
	trusted-key 1\\
	
	
	
	
	NTP zdrojové rozhranie nie je rovnaké pri každom reštarte	& Definovanie loopback zdrojového rozhrania pre NTP	&ntp source loopback $<$id$>$\\
	
	
	
	
	\rowcolor[rgb]{ .94,  .94,  .94}Väčšia bezpečnosť (pub/priv key) NTP a podpora IPv6	&Použitie NTP verzie 4	&ntp server $<$ip adddress$>$ version 4\\
	
	
	
	Falošný čas od podvrhnutého NTP zdroja	&Nastavenie NTP peer s inými sieťovými zariadeniami na krížovú validáciu času a záložný zdroj času	&ntp peer $<$ip adddress$>$
	
	ip access-list standard $<$acl name$>$
	
	\hspace{0.5em}remark permit only this IP 
	
	\hspace{0.5em}permit $<$ip adddress$>$ $<$wildcard mask$>$
	
	\hspace{0.5em}remark deny other 
	
	\hspace{0.5em}deny any log-input
	
	ntp access-group serve-only $<$acl name$>$
	
	interface $<$interface name$>$ $<$interface id$>$
	
	\hspace{0.5em}ntp disable\\
	
	
	
	
	\rowcolor[rgb]{ .94,  .94,  .94} Útočník s fyzickým prístupom k zariadeniu alebo portu môže odpočúvať alebo posielať škodlivý obsah	&Explicitne zakázať nepoužívané porty	&interface $<$interface name$>$ $<$interface id$>$
	
	\hspace{0.5em}shutdown\\
	
	
	
	
	Zdrojové rozhranie pre management a control protokoly	&Vytvoriť Loopback rozhranie s IP adresou	&interface loopback $<$id$>$
	
	\hspace{0.5em}ip address $<$ip paddress$>$\\
	
	
	
	
	\rowcolor[rgb]{ .94,  .94,  .94} Identifikácia pravidla v ACL	&Popis každého pravidla v ACL pre lepšiu identifikáciu	&ip access-list standard $<$acl name$>$
	
	\hspace{0.5em}remark Deny SNMP from VLAN 20
	
	\hspace{0.5em}deny ip $<$ip address$>$ $<$wildcard mask$>$
	\vspace{0.5em}
	
	ipv6 access-list $<$acl name$>$
	
	\hspace{0.5em}remark Deny SNMP from VLAN 20
	
	\hspace{0.5em}deny $<$ipv6 address$>$ $<$prefix$>$ any\\
	
	
	
	
	Identifikácia rozhrania	&Popis každého rozhrania	&interface $<$interface name$>$ $<$interface id$>$
	
	\hspace{0.5em}description PRODUCTION\_SERVER\\
	
	
	
	
	\rowcolor[rgb]{ .94,  .94,  .94}SSH zdrojové rozhranie nie je rovnaké pri každom reštarte	& Definovanie loopback zdrojového rozhrania pre SSH	&ip ssh source-interface loopback $<$id$>$\\
	
	
	
	
	DOS útok na štandardný SSH port 22	&Špecifikovanie iného portu pre SSH ako štandardného alebo aplikovanie port knocking	&ip ssh port 2223
	
	\vspace{0.5em}
	{\fontfamily{lmr}\selectfont alebo}
	\vspace{0.5em}
	
	ip access-list extended $<$acl name$>$
	
	\hspace{0.5em}remark *** KNOCK ***
	
	\hspace{0.5em}permit udp any any eq 65535 log-input
	
	\hspace{0.5em}remark *** TRUSTED ***
	
	\hspace{0.5em}permit tcp any any established
	
	\hspace{0.5em}remark *** DENIED ***
	
	\hspace{0.5em}deny   tcp any any log input
	
	\hspace{0.5em}remark *** PERMITED ***
	
	\hspace{0.5em}permit ip any any
	
	
	interface $<$interface name$>$ $<$interface id$>$
	
	\hspace{0.5em}ip access-group $<$acl name$>$
	
	\hspace{0.5em}ipv6 traffic-filter $<$acl name$>$
	
	event manager environment $<$env name$>$ $<$acl name$>$
	
	event manager applet KNOCK
	
	\hspace{0.5em}event syslog pattern "\%SEC-6-IPACCESSLOGP: list \$KNOCK\_ACL
	
	\hspace{1em}permitted *"
	
	\hspace{0.5em}action 1.0 regexp "[0-9]+$\backslash$.[0-9]+$\backslash$.[0-9]+$\backslash$.[0-9]+" \$\_syslog\_msg ADDR
	
	\hspace{0.5em}action 1.1 regexp "$\backslash$([0-9]+$\backslash$)," "\$\_syslog\_msg" PORT
	
	\hspace{0.5em}action 1.2 regexp "[0-9]+" "\$PORT" PORT 
	
	\hspace{0.5em}action 2.0 syslog msg "Received a knock from \$ADDR on port \$PORT..."
	
	\hspace{0.5em}action 2.1 syslog msg "Adding \$ADDR to the \$KNOCK\_ACL ACL"
	
	\hspace{0.5em}action 3.0 cli command "enable"
	
	\hspace{0.5em}action 3.1 cli command "configure terminal"
	
	\hspace{0.5em}action 3.2 cli command "ip access-list extended \$KNOCK\_ACL"
	
	\hspace{0.5em}action 3.3 cli command "1 permit tcp host \$ADDR any eq 22"
	
	\hspace{0.5em}action 4.0 WAIT 15
	
	\hspace{0.5em}action 5.0 syslog msg "Removing \$ADDR to the \$KNOCK\_ACL ACL"
	
	\hspace{0.5em}action 6.0 cli command "no permit tcp host \$ADDR any eq 22"
	
	\hspace{0.5em}action 6.1 cli command "exit"\\
	
	
	
	\rowcolor[rgb]{ .94,  .94,  .94}Nepovolený prístup k manažmentu konfigurácie zariadenia	&Vypnutie odchádzajúcich spojení pre protokoly na manažment zariadení pokiaľ sa nepoužívajú (telnet a pod.)	&line vty $<$num$>$ $<$num$>$
	
	\hspace{0.5em}transport output none
	
	line tty $<$num$>$ $<$num$>$
	
	\hspace{0.5em}transport output none
	
	line con $<$num$>$
	
	\hspace{0.5em}transport output none
	
	line aux $<$num$>$
	
	\hspace{0.5em}transport output none\\
	
	
	
	
	
	Odpočúvanie konfigurácií zariadení pri zálohe	&Zapnutie zabezpečenej zálohy na server (SFTP, SCP)	&ip scp server enable
	
	copy startup-config scp://$<$username$>$@$<$ip address$>$/backup\\
	
	
	
	
	\rowcolor[rgb]{ .94,  .94,  .94}Vymazanie konfigurácie	&Zapnutie ochrany pred výmazom konfigurácie	&secure boot config\\
	
	
	
	Možnosť urobiť diff zmien konfigurácií a jej návrat	&Periodické zálohovanie konfigurácie a logovanie jej zmien	&archive
	write-memory
	
	time-period $<$num$>$
	
	log changes
	
	log config
	
	logging enable
	
	logging size $<$num$>$
	
	
	hidekeys
	
	notify syslog
	
	maximum $<$num$>$\\
	
	
	
	
	\rowcolor[rgb]{ .94,  .94,  .94} DOS útok alebo pokus o prístup k tomu, čo nie je povolené	&Logovanie pravidiel zahodenia paketov v ACL	&ip access-list standard $<$acl name$>$
	
	\hspace{0.5em}deny any log-input
	
	ipv6 access-list $<$acl name$>$
	
	\hspace{0.5em}deny any any log-input\\
	
	
	
	
	Nízky stav voľnej pamäte	&Nastavenie notifikácie pri dochádzaní pamäte	&memory free low-watermark processor $<$threshold$>$
	
	memory free low-watermark io $<$threshold$>$\\
	
	
	
	
	\rowcolor[rgb]{ .94,  .94,  .94}Logovacie správy nemôžu byť zaznamenané kvôli nedostatku pamäte	&Rezervovanie pamäte pre kritické notifikácie pri nedostatku pamäte	&memory reserve critical $<$value$>$ \\
	
	
	
	
	Vysoké zaťaženie CPU	&Nastavenie notifikácie vysokom zaťažení CPU	&snmp-server enable traps cpu threshold
	
	snmp-server host $<$ip adddress$>$ version 3 priv $<$user$>$ cpu
	
	process cpu threshold type $<$type$>$ rising 
	
	\hspace{0.5em}$<$percentage$>$ interval $<$seconds$>$
	
	process cpu statistics limit entry-percentage\\
	
	
	
	\rowcolor[rgb]{ .94,  .94,  .94} Vysoké zaťaženie zariadenia spôsobilo nemožnosť prihlásenia k nemu	&Rezervovanie pamäte pre protokoly na manažment zariadení pri nedostatku pamäte	&memory reserve console 4096\\
	
	
	
	Pretečenie pamäte	&Povoliť mechanizmy na detekciu pretečenia pamäte	&exception memory ignore overflow io
	
	exception memory ignore overflow processor
	
	exception crashinfo maximum files $<$number-of-files$>$\\
	
	
	
	
	\rowcolor[rgb]{ .94,  .94,  .94}Načítanie škodlivej konfigurácie zo siete počas bootovania	&Vypnutie načítania operačného systému alebo konfigurácie zo siete pokiaľ to nie je nutné	&no boot network
	no service config\\
	
	
	
	
	Proxy ARP môže viesť k obídeniu PVLAN a rozširuje broadcast doménu	&Vypnutie Proxy ARP	&no proxy-arp\\
	
	
	
	\rowcolor[rgb]{ .94,  .94,  .94}DOS útok na stanicu, cez ktorú bola špecifikovaná cesta a teda nemožnosť komunikácie s koncovým bodom. Alebo zosnovanie MITM útoku	&Vypnutie IP source routing	&no ip source-route\\
	
	
	
	
	DOS útok pomocou podvrhnutej IP adresy alebo vzdialený útok na smerovací protokol	&Zapnutie reverse path forwarding strict/loose mode	&ip verify unicast source reachable-via rx\\
	
	
	
	\rowcolor[rgb]{ .94,  .94,  .94} Nepoužívané, staré a nezabezpečené služby môžu byť použité na škodlivé účely	&Vypnutie nepoužívaných služieb z bezpečnostných dôvodov a na šetrenie CPU a pamäte 	&no ip bootp server\\
	
	
	
	
	Nepoužívané, staré a nezabezpečené služby môžu byť použité na škodlivé účely	&Vypnutie nepoužívaných služieb z bezpečnostných dôvodov a na šetrenie CPU a pamäte 	&no service pad\\
	
	
	
	
	\rowcolor[rgb]{ .94,  .94,  .94} Nepoužívané, staré a nezabezpečené služby môžu byť použité na škodlivé účely	&Vypnutie nepoužívaných služieb z bezpečnostných dôvodov a na šetrenie CPU a pamäte 	&no ip identd\\
	
	
	
	Nepoužívané, staré a nezabezpečené služby môžu byť použité na škodlivé účely	&Vypnutie nepoužívaných služieb z bezpečnostných dôvodov a na šetrenie CPU a pamäte 	&no vstack\\
	
	
	
	\rowcolor[rgb]{ .94,  .94,  .94} Nepoužívané, staré a nezabezpečené služby môžu byť použité na škodlivé účely	&Vypnutie nepoužívaných služieb z bezpečnostných dôvodov a na šetrenie CPU a pamäte 	&no ip http server
	
	no ip http secure server\\
	
	
	
	Nepoužívané, staré a nezabezpečené služby môžu byť použité na škodlivé účely	&Vypnutie nepoužívaných služieb z bezpečnostných dôvodov a na šetrenie CPU a pamäte 	&no service tcp-small-server\\
	
	
	
	
	\rowcolor[rgb]{ .94,  .94,  .94} Nepoužívané, staré a nezabezpečené služby môžu byť použité na škodlivé účely	&Vypnutie nepoužívaných služieb z bezpečnostných dôvodov a na šetrenie CPU a pamäte 	&no service udp-small-server\\
	
	
	
	Nepoužívané, staré a nezabezpečené služby môžu byť použité na škodlivé účely	&Vypnutie nepoužívaných služieb z bezpečnostných dôvodov a na šetrenie CPU a pamäte 	&no service finger\\
	
	
	
	
	\rowcolor[rgb]{ .94,  .94,  .94} Nepoužívané, staré a nezabezpečené služby môžu byť použité na škodlivé účely	&Vypnutie nepoužívaných služieb z bezpečnostných dôvodov a na šetrenie CPU a pamäte 	&interface $<$interface name$>$ $<$interface id$>$
	
	\hspace{0.5em} no mop enabled\\
	
	
	
	Nepoužívané, staré a nezabezpečené služby môžu byť použité na škodlivé účely	&Vypnutie nepoužívaných služieb z bezpečnostných dôvodov a na šetrenie CPU a pamäte 	&no ip domain lookup\\
	
	
	
	\rowcolor[rgb]{ .94,  .94,  .94} Útočník môže zistiť, že IP adresa, na ktorú skúšal ping je nesprávna	&Vypnutie správ ICMP Unreachable	&interface $<$interface name$>$ $<$interface id$>$
	
	\hspace{0.5em}no ip unreachables\\
	
	
	
	
	Útočník môže zistiť masku podsiete pomocou ICMP Mask reply	&Vypnutie správ ICMP Mask reply	&interface $<$interface name$>$ $<$interface id$>$
	
	\hspace{0.5em}no ip mask-reply\\
	
	
	
	
	\rowcolor[rgb]{ .94,  .94,  .94} Umožňuje DOS Smurf útok, mapovanie siete pomocou ping na broadcast adresu vzdialenej siete	&Vypnutie ICMP echo správ na broadcast adresu, vypnutie directed broadcasts	&interface $<$interface name$>$ $<$interface id$>$
	
	\hspace{0.5em}no ip directed-broadcast\\
	
	
	
	
	Útočník môže zistiť smerovacie informácie alebo vyťažiť CPU	&Vypnutie správ ICMP Redirects	&interface $<$interface name$>$ $<$interface id$>$
	
	\hspace{0.5em}no ip redirects\\
	
	
	
	\rowcolor[rgb]{ .94,  .94,  .94} Nekonzistencia konfiguračných súborov pri zmenách konfigurácie viac ako jedným administrátorom	&Povoliť súčasne iba jednému administrátorovi vykonávanie zmien v konfigurácii	&configuration mode exclusive auto\\
	
	
	
	
	Problém identifikácie SYSLOG správ s rovnakou časovou značkou	&Pridanie sekvenčného čísla ku každej syslog správe	&service sequence-numbers\\
	
	
	
	
	\rowcolor[rgb]{ .94,  .94,  .94} Nemožnosť prihlásenia pri zaseknutom TCP spojení	&Terminovanie zaseknutého TCP spojenia	&service tcp-keepalives-in
	
	service tcp-keepalives-out\\
	
	
	
	
	Vloženie a manipulácia so smerovacími informáciami	&Autentizácia smerovacích protokolov (nie heslá v otvorenej podobe)	&router bgp $<$as number$>$
	
	\hspace{0.5em}neighbor $<$ip address$>$ password $<$password$>$  \\
	
	
	
	
	\rowcolor[rgb]{ .94,  .94,  .94} Vloženie a manipulácia so smerovacími informáciami	&Autentizácia smerovacích protokolov (nie heslá v otvorenej podobe)	&key chain $<$chain name$>$
	
	\hspace{0.5em}key$<$id$>$
	
	\hspace{1em}key-string $<$password$>$
	
	interface $<$interface name$>$ $<$interface id$>$
	
	\hspace{0.5em}ip authentication mode eigrp $<$as$>$ md5
	
	\hspace{0.5em}ip authentication keyc-chain eigrp $<$as$>$ $<$chain name$>$
	
	\hspace{0.5em}ipv6 authentication mode eigrp $<$as$>$ md5
	
	\hspace{0.5em}ipv6 authentication keyc-chain eigrp $<$as$>$ $<$chain name$>$\\
	
	
	
	
	Vloženie a manipulácia so smerovacími informáciami	&Autentizácia smerovacích protokolov (nie heslá v otvorenej podobe)	&key chain $<$chain name$>$
	
	\hspace{0.5em}key$<$id$>$
	\hspace{1em}key-string $<$password$>$
	
	router ospf $<$process id$>$
	
	\hspace{0.5em}area $<$ared id$>$ authentication message-digest
	
	\hspace{0.5em}area $<$area id$>$ authentication key-chain $<$chain name$>$
	
	ipv6 router ospf $<$process id$>$
	
	\hspace{0.5em}area $<$area id$>$ authentication message-digest
	
	interface $<$interface name$>$ $<$interface id$>$
	
	\hspace{0.5em}ip ospf message-digest-key $<$key id$>$ md5|sha $<$password$>$
	
	\hspace{0.5em}ip ospf authentication message-digest
	
	\hspace{0.5em}ospfv3 authentication md5 0 2757613409476813242031209727
	
	\hspace{0.5em}no ip ospf authentication-key OPENKEY\\
	
	
	
	
	\rowcolor[rgb]{ .94,  .94,  .94} Vloženie a manipulácia so smerovacími informáciami	&Autentizácia smerovacích protokolov (nie heslá v otvorenej podobe)	&key chain $<$chain name$>$
	
	\hspace{0.5em}key $<$id$>$
	
	\hspace{1em}key-string $<$password$>$
	
	interface $<$interface name$>$ $<$interface id$>$
	
	\hspace{0.5em}ip rip authentication key-chain $<$chain name$>$
	
	\hspace{0.5em}ip rip authentication mode md5\\
	
	
	
	
	OSPF virtuálne linky degradujú výkon	&Vypnutie virtuálnych liniek pre OSPF	&no area $<$area id$>$ virtual-link $<$ip address$>$\\
	
	
	
	\rowcolor[rgb]{ .94,  .94,  .94} Koncové zariadenie, užívateľ a útočník môžu vidieť smerovacie správy a topológiu siete alebo pripojenie škodlivého zariadenia, ktoré vysielať a prijímať smerovacie správy	&Špecifikovanie rozhraní, ktoré nebudú prijímať smerovacie informácie	&router rip
	
	\hspace{0.5em}passive-interface default
	
	\hspace{0.5em}no passive-interface $<$interface name$>$ $<$interface id$>$
	
	router ospf $<$process$>$
	
	\hspace{0.5em}passive-interface default
	
	\hspace{0.5em}no passive-interface $<$interface name$>$ $<$interface id$>$
	
	router eigrp $<$as$>$
	
	\hspace{0.5em}passive-interface default
	
	\hspace{0.5em}no passive-interface $<$interface name$>$ $<$interface id$>$\\
	
	
	
	
	Nemožnosť sprevádzkovať procesy smerovacích protokolov v určitých prípadoch pri použití IPv6	&Špecifikovanie identifikátorov smerovacích protokolov pre každý router (router ID)	&router ospf $<$process id$>$|eigrp1$<$as number$>$|bgp$<$as$>$
	
	\hspace{0.5em}router-id $<$ip-address$>$ 
	
	ipv6 router ospf  $<$process-id$>$
	
	\hspace{0.5em} router-id $<$ip-address$>$ \\
	
	
	
	
	\rowcolor[rgb]{ .94,  .94,  .94} Vysledovateľnosť nefunkčnosti smerovacieho protokolu a nesprávneho nastavenia	&Zaznamenanie zmeny v logu pri zmenách v smerovaní	&router eigrp $<$as$>$|ospf $<$process id$>$|bgp $<$as$>$
	
	\hspace{0.5em}log-neighbor-changes\\
	
	
	
	Škodlivé vloženie smerovacích informácií informácií, vzdialený útok	&TTL security	&hostname $<$hostname$>$\\
	
	
	
	\rowcolor[rgb]{ .94,  .94,  .94} Nesprávne smerovanie kvôli sumarizácií	&Vypnutie automatickej sumarizácie smerovacích protokolov	&router rip
	
	\hspace{0.5em}no auto-summary
	
	router eigrp $<$as$>$
	
	\hspace{0.5em}no auto-summary
	
	router bgp $<$as$>$
	
	\hspace{0.5em}no auto-summary\\
	
	
	
	
	Pakety budú spracovávané v CPU, ktoré môže byť preťažené a môže byť zmenené smerovanie na obídenie bezpečnostnej kontroly	&Zahadzovanie IPv4 paketov s rozšírenou hlavičkou (IP Options filtering)	&ip options drop\\
	
	
	
	
	\rowcolor[rgb]{ .94,  .94,  .94} Odpočúvanie komunikácie  cez nezabezpečené tunely	&Vypnúť tunely ktoré nie sú zabezpečené alebo zabezpečiť tunely	&crypto isakmp policy $<$policy id$>$
	
	\hspace{0.5em}encryption aes
	
	\hspace{0.5em}authentication pre-shared
	
	\hspace{0.5em}group $<$group id$>$
	
	crypto isakmp key $<$key$>$ address $<$ip address$>$ 
	
	crypto ipsec transform-set $<$set name$>$ esp-aes esp-sha-hmac
	
	\hspace{0.5em}crypto map $<$map name$>$ 10 ipsec-isakmp
	
	\hspace{1em}set peer $<$peer ip$>$
	
	\hspace{1em}set transform $<$set name$>$ 
	
	\hspace{1em}match address $<$acl name$>$
	
	ip access-list extended $<$acl name$>$
	
	\hspace{0.5em}permit ip $<$source ip$>$ $<$wildcard mask$>$ $<$destination ip$>$ 
	
	\hspace{1em}$<$ wildcard mask$>$
	
	interface $<$interface name$>$
	
	\hspace{0.5em}crypto map $<$map name$>$
	
	interface tunnel $<$number$>$
	
	\hspace{0.5em}ip address $<$ip address$>$ $<$mask$>$
	
	\hspace{0.5em}tunnel source $<$ip address$>$ $<$mask$>$
	
	\hspace{0.5em}tunnel destination $<$ip address$>$ $<$mask$>$\\
	
	
	
	
	Môže byť zneužité odpočúvanie pokiaľ sa používa monitorovanie prevádzky kvôli legislatívnym potrebám	(zrkadlenie portov, záznam tokov) &Monitorovanie výkonnosti siete a zber sieťového prenosu kvôli legislatívnym potrebám	&ip flow-export version 9
	
	
	ip flow-export destination $<$ip address$>$ $<$port$>$
	
	interface $<$interface name$>$ $<$interface id$>$
	
	\hspace{0.5em}ip flow ingress
	
	\hspace{0.5em}ip flow egress\\
	
	
	
	
	\rowcolor[rgb]{ .94,  .94,  .94} Môže byť zneužité odpočúvanie pokiaľ sa používa monitorovanie prevádzky kvôli legislatívnym potrebám (zrkadlenie portov, záznam tokov)&Monitorovanie výkonnosti siete a zber sieťového prenosu kvôli legislatívnym potrebám	&monitor session $<$session id$>$ source $<$interface name$>$ 
	
	\hspace{0.5em}$<$interface id$>$
	
	monitor session $<$session id$>$ destination $<$interface name$>$ 
	
	\hspace{0.5em}$<$interface id$>$\\
	
	
	
	
	IP spoofing	&Špecifikácia ACL na zakázanie a logovanie privátnych a špeciálnych IP adries z RFC 6890, RFC 8190	&ip access-list standard $<$acl name$>$
	
	\hspace{0.5em}remark BLOCK\_ADDRESSES RFC 1918, 6890, 8190
	
	\hspace{0.5em}deny $<$ip address$>$ $<$wildcard mask$>$ log-input
	
	interface $<$interface name$>$ $<$interface id$>$
	
	\hspace{0.5em}ip access-group $<$acl name$>$ in\\
	
	
	
	
	\rowcolor[rgb]{ .94,  .94,  .94} IP spoofing	&Špecifikácia ACL na zakázanie a logovanie špeciálnych IPv6 adries z RFC 6890, RFC 8190, RFC 5156	&
	ipv6 access-list $<$acl name$>$
	
	\hspace{0.5em}remark BLOCK\_ADDRESSES RFC 5156,6890,8190
	
	\hspace{0.5em}deny $<$ipv6 address$>$ $<$prefix$>$ any log-input
	
	interface $<$interface name$>$ $<$interface id$>$
	
	\hspace{0.5em}ipv6 traffic-filter $<$acl name$>$ in\\
	
	
	
	Rogue root bridge 	&Rogue root bridge protection (root guard)	&interface $<$interface name$>$ $<$interface id$>$
	
	\hspace{0.5em}spanning-tree rootguard
	
	\vspace{0.5em}
	{\fontfamily{lmr}\selectfont alebo}
	\vspace{0.5em}
	
	\hspace{0.5em}spanning-tree guard root\\
	
	
	
	
	\rowcolor[rgb]{ .94,  .94,  .94} Pripojenie prepínaču na koncový prístupový port	&BPDU protection (BPDU guard)	&spanning-tree portfast bpduguard default
	
	\vspace{0.5em}
	{\fontfamily{lmr}\selectfont alebo}
	\vspace{0.5em}
	
	interface $<$interface name$>$ $<$interface id$>$
	
	\hspace{0.5em}spanning-tree bpduguard enable
	
	{\fontfamily{lmr}\selectfont vyhnúť sa} spanning-tree portfast bpdufilter enable
	
	
	interface $<$interface name$>$ $<$interface id$>$
	
	{\fontfamily{lmr}\selectfont \hspace{0.5em}vyhnúť sa} spanning-tree bpdufilter enable\\
	
	
	
	
	Rýchlosť konvergencie	&Prístupové porty by sa nemali podieľať na STP procese	&spanning-tree portfast default
	
	\vspace{0.5em}
	{\fontfamily{lmr}\selectfont alebo}
	\vspace{0.5em}
	
	interface $<$interface name$>$ $<$interface id$>$
	
	\hspace{0.5em}spanning-tree portfast\\
	
	
	
	
	\rowcolor[rgb]{ .94,  .94,  .94} Jednosmerná komunikácia medzi prepínačmi môže viesť k topológii so slučkami	& Špeciálne konfigurácie zaisťujúce bezslučkovú topológiu pomocou STP keď nastane jednosmerná komunikácia (Loop Guard)	&spanning-tree loopguard default
	
	\vspace{0.5em}
	{\fontfamily{lmr}\selectfont alebo }
	\vspace{0.5em}
	
	interface $<$interface name$>$ $<$interface id$>$
	
	\hspace{0.5em}spanning-tree guard loop\\
	
	
	
	
	Nemožnosť identifikácie účelu VLAN	&Pridanie mena k VLAN	&vlan $<$id$>$
	name $<$name$>$\\
	
	
	
	
	\rowcolor[rgb]{ .94,  .94,  .94} Špeciálna VLAN pre manažment na obmedzenie prístupu iba pre administrátorov	&Vytvorenie separátnej VLAN pre manažment	&vlan $<$id$>$
	
	\hspace{0.5em}name MANAGEMENT\_VLAN\\
	
	
	
	
	Útočníkovi s fyzickým prístupom k portu môže byť pridelený prístup do časti siete, ktorá zodpovedá príslušnej VLAN 	&Vytvorenie špeciálnej black hole VLAN pre nevyužité porty	&vlan $<$id$>$ 
	
	\hspace{0.5em}name BLACKHOLE\_VLAN\\
	
	
	
	
	\rowcolor[rgb]{ .94,  .94,  .94} Predvolenej VLAN je povolené prepnuté na akýkoľvek port, VLAN hopping, double tagging	&Odobrať všetky porty z predvolenej VLAN	&interface $<$interface name$>$ $<$interface id$>$
	
	\hspace{0.5em} switchport mode access
	
	\hspace{0.5em} switchport access vlan $<$id$>$\\
	
	
	
	
	Predvolenej VLAN je povolené byť prepnutá na akýkoľvek port, VLAN hopping, double tagging	&Vytvorenie natívnej VLAN rozdielnej ako predvolená, priradenie k trunk portu a povolenie iba potrebných portov	&vlan $<$id$>$  
	name NATIVE\_VLAN
	
	interface $<$interface name$>$ $<$interface id$>$
	
	\hspace{0.5em}switchport mode trunk
	
	\hspace{0.5em}switchport trunk native vlan $<$id$>$
	
	\hspace{0.5em}switchport trunk allowed vlan $<$id$>$\\
	
	
	
	
	\rowcolor[rgb]{ .94,  .94,  .94} DTP útok, Switch spoofing útok	&Vypnutie dynamického trunkovacieho protokolu a explicitne určiť porty ako prístupové a trunk	&
	interface $<$interface name$>$ $<$interface id$>$
	
	\hspace{0.5em}switchport mode trunk
	
	\hspace{0.5em}no switchport mode dynamic desirable
	
	\hspace{0.5em}no switchport mode dynamic auto
	
	\hspace{0.5em}switchport nonegotiate\\
	
	
	
	MAC Spoofing, MAC Flooding 	&Definovanie maximálne 1 MAC adresy na port, priradenie MAC adresy na port	&
	interface $<$interface name$>$ $<$interface id$>$
	
	\hspace{0.5em}switchport port-security maximum 1
	
	\hspace{0.5em}switchport port-security mac-address sticky
	
	\vspace{0.5em}
	{\fontfamily{lmr}\selectfont alebo}
	\vspace{0.5em}
	
	\hspace{0.5em}switchport port-security mac-address static $<$mac address$>$
	
	\hspace{0.5em}switchport port-security\\
	
	
	
	\rowcolor[rgb]{ .94,  .94,  .94} MAC Spoofing, MAC Flooding 	&Nastavenie režimu narušenia, ktorý vypne port alebo informuje správcu o pripojení nepovoleného zariadenia	&interface $<$interface name$>$ $<$interface id$>$
	
	\hspace{0.5em}switchport port-security violation mode shutdown
	
	\hspace{0.5em}switchport port-security violation mode restrict
	
	\hspace{0.5em}no switchport port-security violation mode protect\\
	
	
	
	
	Nový prepínač s vyšším číslom revízie, ale s nesprávnou VLAN databázou môže šíriť falošné VLAN identifikátory a spôsobiť nefunkčnosť siete, veľa možných VTP útokov kvôli zraniteľnostiam 	&Vypnutie MVRP. MRP, GARP, VTP, GVRP po úspešnej propagácií VLAN	&vtp mode transparent
	
	\vspace{0.5em}
	{\fontfamily{lmr}\selectfont alebo}
	\vspace{0.5em}
	
	vtp off\\
	
	
	
	
	\rowcolor[rgb]{ .94,  .94,  .94}VTP musí byť používané	&Uprednostniť VTP verzie 3, špecifikovať skryté heslo a zapnúť VTP prunning pokiaľ musí byť VTP zapnuté	&vtp version 3
	
	vtp password $<$password$>$ hidden
	
	vtp prunning\\
	
	
	
	
	Vysoké zaťaženie linky	&Poslanie notifikácie pri prekročení prahovej hodnoty zaťaženia linky	&interface $<$interface name$>$ $<$interface id$>$
	
	\hspace{0.5em}storm-control unicast level $<$top level$>$ $<$down level$>$
	
	\hspace{0.5em}storm-control broadcast level $<$top level$>$ $<$down level$>$
	
	\hspace{0.5em}storm-control multicast level $<$top level$>$ $<$down level$>$
	
	\hspace{0.5em}storm-control action trap\\
	
	
	
	
	\rowcolor[rgb]{ .94,  .94,  .94} Využívanie siete nepovolenými používateľmi	&Zapnutie 802.1x 	&dot1x system-auth-control
	identity profile default
	
	
	interface $<$interface name$>$ $<$interface id$>$
	
	\hspace{0.5em}dot1x port-control auto
	
	\vspace{0.5em}
	{\fontfamily{lmr}\selectfont alebo}
	\hspace{0.5em}
	
	\hspace{0.5em}access-session port-control auto
	
	\vspace{0.5em}
	{\fontfamily{lmr}\selectfont alebo}
	\vspace{0.5em}
	
	\hspace{0.5em}authentication port-control auto
	
	\hspace{0.5em}dot1x pae authenticator|supplicant 
	
	\hspace{0.5em}no dot1x port-control force-authorized
	
	\vspace{0.5em}
	{\fontfamily{lmr}\selectfont alebo}
	\vspace{0.5em}
	
	\hspace{0.5em}no access-session port-control force-authorized
	
	\vspace{0.5em}
	{\fontfamily{lmr}\selectfont alebo}
	\vspace{0.5em}
	
	\hspace{0.5em}no authentication port-control force-authorized\\
	
	
	
	Útok hrubou silou hádaním prístupových údajov pre 802.1x 	&Limitovanie maximálneho počtu neúspešných pokusov o autentizáciu 802.1x	&dot1x auth-fail max-attempts $<$number$>$\\
	
	
	
	
	\rowcolor[rgb]{ .94,  .94,  .94}IPv6 ND Spoofing	&IPv6 ND Inspection	&ipv6 nd inspection policy $<$policy name$>$
	
	\hspace{0.5em}drop unsecure
	
	\hspace{0.5em}device-role monitor
	
	\hspace{0.5em}tracking disable stale-lifetime infinite
	
	\hspace{0.5em}trusted-port
	
	interface $<$interface name$>$ $<$interface id$>$
	
	\hspace{0.5em}ipv6 nd inspection attach-policy  $<$policy name$>$\\
	
	
	
	Rogue RA
	RA Flood
	Route Information Option injection
	RA RouterLifeTime=0
	&RA Guard	&ipv6 nd raguard policy $<$polic name$>$
	device-role host|router
	
	\hspace{0.5em}hop-limit maximu $<$number$>$
	
	\hspace{0.5em}managed-config-flag on|off
	
	\hspace{0.5em}other-config-flag on|off
	
	\hspace{0.5em}match ipv6 access-list $<$acl name$>$
	
	\hspace{0.5em}match ra prefix-list $<$prefix list name$>$
	
	\hspace{0.5em}trusted-port
	
	interface $<$interface name$>$ $<$interface id$>$
	
	\hspace{0.5em}ipv6 nd raguard attach-policy $<$policy name$>$\\
	
	
	
	
	\rowcolor[rgb]{ .94,  .94,  .94}DHCP spoofing	&DHCP snooping, IPv6 Snooping, DHCPv6 Guard	&ip dhcp snooping
	ip dhcp snooping vlan $<$vlan-id$>$ 
	
	interface $<$interface name$>$ $<$interface id$>$
	
	\hspace{0.5em}ip dhcp snooping trust
	
	\hspace{0.5em}no ip dhcp snooping trust\\
	
	
	
	
	DHCP spoofing	&DHCP snooping, IPv6 Snooping, DHCPv6 Guard	&ipv6 snooping policy $<$policy name$>$
	
	\hspace{0.5em}ipv6 snooping attach-policy $<$policy name$>$
	\hspace{0.5em}prefix-glean\\
	
	
	
	\rowcolor[rgb]{ .94,  .94,  .94} DHCP spoofing	&DHCP snooping, IPv6 Snooping, DHCPv6 Guard	&ipv6 access-list $<$acl name$>$
	
	\hspace{0.5em}permit host $<$ipv6 address$>$ any
	
	ipv6 prefix-list $<$prefix list name$>$ permit $<$ipv6 address$>$  
	
	\hspace{0.5em}le 128
	
	ipv6 dhcp guard policy $<$policy name$>$
	
	\hspace{0.5em}device-role server|client
	
	\hspace{0.5em}match server access-list $<$acl name$>$
	
	\hspace{0.5em}match reply prefix-list $<$prefix list name$>$
	
	\hspace{0.5em}trusted-port
	
	interface $<$interface name$>$ $<$interface id$>$
	
	\hspace{0.5em}ipv6 dhcp guard attach-policy $<$policy name$>$\\
	
	
	
	
	Príliš veľa DHCP paketov, zaplavenie DHCP paketmi	&Obmedziť počet DHCP paketov na nedôveryhodných rozhraniach	&ip dhcp snooping limit rate 100 \\
	
	
	
	\rowcolor[rgb]{ .94,  .94,  .94} ARP Spoofing	&Dynamic ARP Inspection	&ip arp inspection vlan $<$vlan id$>$
	 
	ip arp inspection validate src-mac dst-mac 
	
	\hspace{0.5em}
	
	{\fontfamily{lmr}\selectfont na uplink}
	
	interface $<$interface name$>$ $<$interface id$>$
	
	\hspace{0.5em}ip arp inspection trust\\
	
	
	
	
	IP spoofing	&IPv4/IPv6 Source Guard	&ip verify source port-security
	
	ip verify source
	
	ip verify source vlan dhcp-snooping
	
	ip verify source vlan dhcp-snooping port-security
	
	ipv6 source-guard policy $<$policy name$>$
	
	\hspace{0.5em}permit link-local
	
	\hspace{0.5em}deny global-autoconf
	
	\hspace{0.5em}trusted
	
	interface $<$interface name$>$ $<$interface id$>$
	
	\hspace{0.5em}ipv6 source-guard attach-policy $<$policy name$>$\\
	
	
	
	
	\rowcolor[rgb]{ .94,  .94,  .94}IPv6 Next Header  a IPv6 Fragmentation útok	&ACL blokujúce nerozpoznateľné rozšírené hlavičky	&ipv6 access-list $<$acl name$>$
	
	\hspace{0.5em}remark deny undetermined next headers
	
	\hspace{0.5em}deny any any undetermined-transport log-input\\
	
	
	
	
	Mapovanie siete pomocou pingu na multicast adresu všetkých uzlov a MLD/IGMP Query Overload a Smurf útok	& ACL blokujúce ICMP echo request na multicast adresu všetkých uzlov a MLD/IGMP Query na prístupových portoch	&ip access-list extended $<$acl name$>$
	
	\hspace{0.5em}remark deny all node ipv4 address
	
	\hspace{0.5em}deny icmp any host 224.0.0.1 echo log-input
	
	
	ipv6 access-list $<$acl name$>$
	
	\hspace{0.5em}remark deny all node ipv6 address
	
	\hspace{0.5em}deny icmp any host ff02::1 echo-request log-input
	
	\hspace{0.5em}remark deny mld query
	
	\hspace{0.5em}deny icmp any any mld-query\\
	
	
	
	
	\rowcolor[rgb]{ .94,  .94,  .94}Mobilné zariadenia pripojené bezdrôtovo spotrebovávajú veľa energie kvôli častým RA správam	&RA Throttling	&ipv6 nd ra-throttle policy $<$policy name$>$
	
	\hspace{0.5em}allow at-least $<$value$>$ at-most $<$value$>$
	
	\hspace{0.5em}interval-option inherit
	
	\hspace{0.5em}max-through $<$value$>$
	
	\hspace{0.5em}media-type wired|access-point|wire|wifi
	
	\hspace{0.5em}throttle-period $<$value$>$
	
	vlan configuration $<$vlan id$>$
	
	\hspace{0.5em}ipv6 nd ra-throttle attach-policy $<$policy name$>$
	
	\vspace{0.5em}
	{\fontfamily{lmr}\selectfont alebo }
	\vspace{0.5em}
	
	interface $<$interface name$>$ $<$interface id$>$
	
	\hspace{0.5em}ipv6 nd ra-throttle policy $<$policy name$>$\\
	
	
	
	
	Zlyhanie zariadenia alebo linky môže viest k nefunkčnosti siete 	&Povolenie FHRP s autentizáciou a aktuálnou verziou	&key chain $<$key chain$>$
	
	\hspace{0.5em}key $<$id$>$
	
	\hspace{1em}key-string $<$key string$>$
	
	track $<$value$>$ interface $<$interface name$>$ $<$interface id$>$ 
	
	\hspace{0.5em}line-protocol
	
	fhrp version vrrp 3
	
	interface $<$interface name$>$ $<$interface id$>$
	
	\hspace{0.5em}vrrp $<$group id$>$ ip $<$ip  address$>$
	
	\hspace{0.5em}vrrp priority $<$value$>$
	
	\hspace{0.5em}vrrp $<$group id$>$ track $<$value$>$ decrement $<$value$>$
	
	\hspace{0.5em}vrrp $<$group id$>$ authentication md5 key-string $<$key$>$
	
	\vspace{0.5em}
	{\fontfamily{lmr}\selectfont alebo}
	\vspace{0.5em}
	
	\hspace{0.5em}vrrp $<$group id$>$ authentication md5 key-chain $<$key chain$>$\\
	
	
	\rowcolor[rgb]{ .94,  .94,  .94} Zlyhanie zariadenia alebo linky môže viest k nefunkčnosti siete 	&Povolenie FHRP s autentizáciou a aktuálnou verziou	&key chain $<$key chain$>$
	
	\hspace{0.5em}key $<$id$>$
	
	\hspace{0.5em}key-string $<$key string$>$
	
	track $<$id$>$  interface $<$interface name$>$ $<$interface id$>$
	
	interface $<$interface name$>$ $<$interface id$>$
	
	\hspace{0.5em}standby $<$group id$>$ ip $<$ip address$>$
	
	\hspace{0.5em}standby $<$group id$>$ priority $<$value$>$
	
	\hspace{0.5em}standby $<$group id$>$ preempt
	
	\hspace{0.5em}standby version 2
	
	\hspace{0.5em}standby $<$group id$>$ track $<$id$>$ decrement $<$value$>$
	
	\hspace{0.5em}standby $<$group id$>$ authentication md5 key-string $<$key$>$
	
	\vspace{0.5em}
	{\fontfamily{lmr}\selectfont alebo}
	\vspace{0.5em}
	
	\hspace{0.5em}standby $<$group id$>$ authentication md5 key-chain $<$key 
	
	\hspace{1em}chain$>$\\	
	
	
	
	Zlyhanie zariadenia alebo linky môže viest k nefunkčnosti siete 	&Povolenie FHRP s autentizáciou a aktuálnou verziou	&key chain $<$key chain$>$
	
	\hspace{0.5em}key $<$id$>$
	
	\hspace{1em}key-string $<$key string$>$
	
	track $<$value$>$ interface $<$interface name$>$ $<$interface id$>$
	
	\hspace{0.5em}line-protocol
	
	interface $<$interface name$>$ $<$interface id$>$
	
	\hspace{0.5em}glbp $<$group id$>$ ip $<$ip  address$>$
	
	\hspace{0.5em}glbp $<$group id$>$ priority $<$value$>$
	
	\hspace{0.5em}glbp $<$group id$>$ preempt
	
	\hspace{0.5em}glbp $<$group id$>$ weighting $<$value$>$ lower $<$value$>$ upper $<$value$>$ 
	
	\hspace{0.5em}glbp $<$group id$>$ weighting track $<$value$>$ decrement $<$value$>$
	
	\hspace{0.5em}glbp $<$group id$>$ authentication md5 key-string $<$key$>$
	
	\vspace{0.5em}
	{\fontfamily{lmr}\selectfont alebo}
	\vspace{0.5em}
	
	\hspace{0.5em}glbp $<$group id$>$ authentication md5 key-chain $<$key chain$>$\\
	
	
	
	
	\rowcolor[rgb]{ .94,  .94,  .94}Vyčerpanie cache susedov	&Statický záznam pre kritické zariadenia (servery) spájajúce IP a MAC adresu a VLAN
	&ipv6 neighbor $<$ipv6 address$>$ vlan $<$vlan id$>$ $<$mac address$>$\\
	
	
	
	
	Vyčerpanie cache susedov	&Na zabránenie vzdialeného útoku na cache susedov cez internet je potreba nastaviť ACL, kde povoľujeme iba komunikáciu s cieľovými IPv6 adresami, ktoré sa nachádzajú v našej sieti	&ipv6 access-list $<$acl name$>$
	
	\hspace{0.5em}remark permit only this ip  
	
	\hspace{0.5em}permit any $<$ipv6$>$/$<$prefix$>$
	
	\hspace{0.5em}remark deny other
	
	\hspace{0.5em}deny ipv6 any any 
	
	interface $<$interface name$>$ $<$interface id$>$
	
	\hspace{0.5em}ipv6 traffic-filter $<$acl name$>$ in\\
	
	
	
	
	\rowcolor[rgb]{ .94,  .94,  .94} Vyčerpanie cache susedov	&IP destination Guard (First Hop Security)
	
	
	&ipv6 destination-guard policy $<$policy name$>$
	
	\hspace{0.5em}enforcement always
	
	interface $<$interface name$>$ $<$interface id$>$
	
	\hspace{0.5em}ipv6 destination-guard attach-policy $<$policy name$>$\\
	
	

	
	
	
	\rowcolor[rgb]{ .94,  .94,  .94} Vyčerpanie cache susedov	&Limitovanie času IPv6 adresy v cache susedov	&ipv6 nd cache expire $<$time in seconds$>$\\
	
	
	
	Komplexné bezpečnostné hrozby a narušenie bezpečnosti	&Nastavenie IDS/IPS	&ip ips sdf location $<$signature location$>$
	
	ip ips fail  open|close
	
	ip ips $<$signature name$>$ list $<$alc name$>$
	
	ip ips $<$signature name$>$ in|out\\
	
	
	\hline
	
\end{longtable}%



\restoregeometry
\normalsize

