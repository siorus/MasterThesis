\chapter{Implementácia}
\phantomsection

\section{Použité technológie}
 \subsection{Python}
 Python \cite{B4mfUgNUpPnXbiEr} je objektovo orientovaný, interpretovaný programovací jazyk vytvorený holanďanom Guidom van Rossom. Radí sa k~vysokoúrovňovým programovacím jazykom a umožňuje automatickú správu pamäte, teda programátor nie je nútený explicitne potrebnú pamäť alokovať a uvolňovať. Jeho výhodou je práve fakt, že je interpretovaný a teda programy napísané v~ňom nemusia byť preložené pre danú platformu, ale postačuje spustenie zdrojového kódu pomocou interpretu nainštalovaného na danom systéme. Interpret jazyka Python je dostupný pre Microsoft Windows, GNU/Linux, macOS a mnoho ďalších vstavaných a exotickejších systémov. Syntax jazyku Python je založená na syntaxi jazyku C, no zdrojový kód je čitateľnejší a na vymedzenie funkčných blokov využíva odsadenie, vďaka čomu je kód čitateľnejší. V~súčasnosti sa využíva syntax a interpret dvoch verzií, a to verzie 2.x a 3.x, ktoré sú vzájomne nekompatibilné, z~dôvodu rozdielnych syntaktických konštrukcií. Prvá zo zmienených verzií však čoskoro prestane byť podporovaná. Z~tohto dôvodu je vhodnejšie použiť na novo vyvíjané programy verziu 3.x. 
  
 Práve pre vyššie zmienené výhody, a to najmä dobrú čitateľnosť kódu, rozšíriteľnosť  medzi programátormi, ako aj spustenie na rôznych platformách bol Python zvolený za programovací jazyk pre túto diplomovú prácu.
 \subsection{YAML}
 YAML \cite{Jd4UTaVyTULvXDoN} je jazyk na serializáciu dát vo forme veľmi dobre čitateľnej pre človeka. Bol inšpirovaný konceptami a syntaxou jazykov C a Python. Dovoľuje definovať primitíva z~týchto programovacích jazykov ako zoznamy, asociatívne polia, reťazce a zároveň v~jednom súbore dovoľuje definovať viac dokumentov.
 
 Pre konfiguračné súbory na túto diplomovú prácu boli spočiatku uvažované tri metódy. Prvou možnosťou na serializáciu dát bol jazyk XML, tento formát však nie je tak dobre čitateľný pre človeka a je tu väčšia pravdepodobnosť zanesenia chýb z~dôvodu uzatvárania značiek. Druhou možnosťou bolo využitie syntaxe jazyka JSON, no problémom je, že nepodporuje vkladanie komentárov, preto nie je vhodný na konfiguračné súbory. Poslednou možnosťou bolo vytvorenie vlastnej syntaxe a vlastného analyzátoru, to je však pre potreby diplomovej práce zbytočné a hotové riešenie v~podobe YAML je dostatočné a eliminuje všetky nedostatky vyššie zmienených jazykov určených na serializáciu dát. Použitý jazyk Python naviac disponuje viacerými knižnicami na prácu s~jazykom YAML.  
 \subsection{Regulárne výrazy}
 %nejaký obkec okolo (krátko), prečo sú vhodné, ako budú použité
 Regulárnym výrazom sa rozumie sekvencia znakov, ktorá definuje určitý vzor \cite{sBBUt3Q3bPUfAMue}. Regulárne výrazy sa používajú na vyhľadávanie alebo výmenu reťazcov v~texte pričom na to využívajú znaky s~vopred definovanou sémantikou. V~diplomovej práci budú použité na vyhľadávanie prítomnosti alebo absencii nastavení v~konfigurácií zariadení. 
 \newpage
\section{Konfiguračné súbory}
%možno do implementácie, automaticke zistovaine niektorych atributov
\subsection{Súbor popisujúci zariadenie}
\begin{lstlisting}[frame=single,numbers=right,caption={Konfiguračný súbor device.yaml, ktorý popisuje základné informácie o~jednom konkrétnom zariadení},label=lst:lldp,basicstyle=\ttfamily\small, keywordstyle=\color{black},language=python,breaklines=true]
---
# Hostname of device
hostname: "sw1-access"

# Path to configuration file of device
config: "sw1_access-config.txt"

# Version of running operating system
version: "12.2(55)SE12"

# L3 protocols which are used 
# not only available but literally used and enabled
l3_protocols:
  - "ipv4"
  - "ipv6"

# Manufacturer of device
# Same directory name has to be created inside 
# directory "Modules", where all modules 
# for this vendor are stored
vendor: "cisco"

# Operating system
os: "ios"

# Type of device
# Types: [r(router), l3sw(L3 switch), l2sw(L2 switch)]
facility: "l3sw"

# Type of layer where facility is installed
# Types: [core/edge, distribution, access, collapsed all, 
# collapsed distribution access, 
# collapsed core distribution]
facility-layer: "access"

# Exclude modules which are specified in file 
# "modules_by_facility.yaml" for specific 
# "facility-layer" you do not want to be used
excluded-modules:
  - "dot1x.py"

# Include modules which are not 
# part of  specific "facility-layer"
# in file "modules_by_facility.yaml" 
# and you want to use them.
include-modules: 
  - "cdp.py"
  - "portsec-max-mac.py"

# All available interfaces, roles of interfaces
# can be specified, roles such as "access" or "trunk"
# are assigned automatically according to config 
# and more than one type can be assigned to port
# Roles: [access, trunk, wan, toinet, access-datacenter,
# wlan, to_distribution_layer, to_core_layer, unused, none]   
interfaces:
- FastEthernet 0/1: "access"
- FastEthernet 0/2: "access-datacenter"
- FastEthernet 0/3: "access"
- FastEthernet 0/4: 
  - "trunk"
  - "to-core-layer"

# SHA1 hash of input configuration of device
input-config-hash: "12975910C3E6352B5B2BDEE81FA2FC4653A5BD59"

# SHA1 hash of current fix configuration 
fix-hash: "86F7E437FAA5A7FCE15D1DDCB9EAEAEA377667B8"

\end{lstlisting}
 \newpage
\subsection{Súbor popisujúci modul}

\begin{lstlisting}[frame=single,numbers=right,caption={Konfiguračný súbor cdp\_module.yaml, ktorý obsahuje informácie na nájdenie problému, jeho odstránenie a informácie do správy},label=lst:lldp,basicstyle=\ttfamily\small, keywordstyle=\color{black},language=python,breaklines=true]
---
# Module name, it will be seen in report output
name: "CDP disabled"

# Instance name or identifier, e.g. OSPF processes are used
instance-name: ""

# Type of configuration command
# Types: [o, i, m, b]
# o~- one time in config e.g. "ip ssh version 2"
# i - applied on interface (onetime/manytime) e.g. portsecurity, ACL
# m - multiple time e.g. password for telnet, password for eigrp processes
# b - both interface and general e.g. CDP, root guard
type: "b"

# Type of device
# Types: [r(router), l3sw(L3 switch), l2sw(L2 switch), all]
default-facility: "all"

# Type of layer where facility is installed
# Types: [core/edge, distribution, access, collapsed all, collapsed distribution access, collapsed core distribution, all]
default-facility-layer: "all"

# Run only when module(s) below have found an error
# means security problem or missing configuration
run-if-error-returned: 
  - "none"

# Run only when module(s) below have not find an error
# means configuration which module(s) looking for is present
run-if-error-returned: 
  - "none"

# Variables needed for generating configuration fix
# e.g. - snmp-user: "administrator1"
instance-public-vars:
  - "none"

# Variable that holds secret until configuration fix is generated
# after that it is cleared due to security
# e.g. - password: "" 
instance-secret-vars:
  - "none"

#------------------------------------------------------------
# GENERAL CMD CONFIG
#------------------------------------------------------------
name-cmd-general: "CDP Globally DISABLED"

# Severity which defines importance of found problem 
# Types: [critical, high, medium, low, notice]
default-cmd-general-severity: "critical"

# Severity which defines importance of found problem 
# Types: [critical, high, medium, low, notice]
# Default: none
user-cmd-general-severity: "none"

#Regex to match and find occurrence
regex-cmd-general: "no cdp run"

# Specifies whether module should look for regex occurrence
# or non-occurrence match
# Types:
# occurrence - set when regex occurrence in variable "regex-cmd-general" signalize no issue
# nonoccurrence - set when regex non-occurrence in variable "regex-cmd-general" signalize no issue
regex-cmd-general-occurrence: "occurrence"

# Boolean variable to store whether regex matches or not 
regex-cmd-general-match-status: "False"

# Command to resolve problem
# String when one line command, for multiple command setup a list
fix-cmd-general: "no cdp run"

# Notice seen in report when fix will be applied
# e.g. notice about something can stop working after applying fix
fix-cmd-general-notice: "Fix may cause CISCO IP telephony malfunction"

# Boolean which indicates whether a fix will be ignored or applied
fix-cmd-general-ignore: "True"

# Comment to specify reason why command for fix is ignored,
# reason or comment about accepting the risk
fix-cmd-general-ignore-comment: "Enabled due to CISCO IP Telephony"

# Boolean which indicates finding an issue is false positive
fix-cmd-general-false-positive: "False"

# Comment to specify reason why is finding marked as false positive
fix-cmd-general-false-positive-comment: "none"

#------------------------------------------------------------
# INTERFACE CMD CONFIG
#------------------------------------------------------------
name-cmd-affected-ports: "CDP on interface DISABLED"

# Severity which defines importance of found problem on affected interface
# Types: [critical, high, medium, low, notice]
default-cmd-affected-ports-severity: "critical"

# Severity which defines importance of found problem on affected interface
# Types: [critical, high, medium, low, notice]
# Default: none
user-cmd-affected-ports-severity: "none"

#Regex to match and find occurrence
regex-cmd-affected-ports: "no cdp enable"

# Specifies whether module should look for regex occurrence
# or non-occurrence match
# Types:
# occurrence - set when regex occurrence in variable "regex-cmd-general" signalize no issue
# nonoccurrence - set when regex non-occurrence in variable "regex-cmd-general" signalize no issue
regex-cmd-affected-ports-occurrence: "occurrence"

# Boolean variable to store whether regex matches or not on interfaces 
regex-cmd-affected-ports-match-status: "False"

# List of ports where issue is found when variable "type" is "i" or "b"
affected-ports:
  - FastEthernet 0/1
  - FastEthernet 0/2

# Command to resolve problem on interfaces
# String when one line command, for multiple command setup a list
fix-cmd-affected-ports: "no cdp enable"

# Notice seen in report when fix will be applied on affected interface
# e.g. notice about something can stop working after applying fix
fix-cmd-affected-ports-notice: "Fix may cause CISCO IP telephony malfunction"

# Boolean which indicates whether a fix will be ignored or applied
fix-cmd-affected-ports-ignore: "True"

# Comment to specify reason why command for fix is ignored
fix-cmd-affected-ports-ignore-comment: "Enabled due to CISCO IP Telephony"

# Boolean which indicates finding an issue is false positive
fix-cmd-affected-ports-false-positive: "False"

# Comment to specify reason why is finding marked as false positive
fix-cmd-affected-ports-positive-comment: "none"


#------------------------------------------------------------
# INTERFACE CMD IGNORE CONFIG
#------------------------------------------------------------

name-cmd-explicit-ignored-ports: "CDP running on interfaces IGNORED"

# Severity which defines importance of found problem on affected interface
# Types: [critical, high, medium, low, notice]
default-cmd-explicit-ignored-ports-severity: "critical"

# Severity which defines importance of found problem on affected interface
# Types: [critical, high, medium, low, notice]
# Default: none
user-cmd-explicit-ignored-ports-severity: "none"

# List of ports which should be ignored when "type" is "i" or "b"
explicit-ignored-ports:
  - FastEthernet 0/3
  - FastEthernet 0/4

# Command to resolve problem on interfaces which are ignored
# Can be blank when you want just ignore ports, some commands
# like CDP can have on ignored ports command "cdp enable" when
# globally is disabled
fix-cmd-explicit-ignored-ports: "cdp enable"

# Notice seen in report when fix will be applied on ignored interface
# e.g. notice about something can stop working after applying fix
fix-cmd-explicit-ignored-ports-notice: "Enabling CDP on interface(s) can lead to serious attacks"

# Comment to specify reason why ignore command is applied
fix-cmd-explicit-ignored-ports-comment: "Enabled due to CISCO IP Telephony"

\end{lstlisting}

%\section{Moduly}


