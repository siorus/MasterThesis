\chapter*{Úvod}
\phantomsection
\addcontentsline{toc}{chapter}{Úvod}

Kybernetická bezpečnosť je bezpochyby jednou z~hlavných tém 21. storočia. Útoky na infraštruktúru a systémy naberajú nielen na frekvencii, ale čo je ešte horšie na sofistikovanosti. Napriek častému zdôrazňovaniu odborníkov o~kladenie čoraz väčšieho dôrazu na bezpečnosť pri návrhu, implementácii a nasadeniu, sa stále stretávame s~fatálnymi dôsledkami, ktoré boli spôsobené nedostatočným venovaním pozornosti bezpečnosti. 

Problém nedostatočného zabezpečenia nie je ani tak nevedomosť základných bezpečnostných praktík administrátorov alebo programátorov, ale potreba rýchleho nasadenia systému a infraštruktúry s~odložením implementácie bezpečnostných praktík na neskôr. Tieto problémy vznikajú aj pri dodatočnej implementácií nových modulov a pridaní novej infraštruktúry, kedy sa nemení celok, ale pridanie jednej časti môže výrazne ovplyvniť a zmeniť stav bezpečnosti celého systému. Z~tohto dôvodu je priam žiadúce disponovať nejakým procesom alebo nástrojom na dodatočné zistenie nedostatkov a ich následnú elimináciu. Veľmi silnou motiváciou by malo byť aj to, že dôsledkom bezpečnostných nedostatkov sú globálne miliardové škody a straty reputácií firiem. 

Jednou z~hlavných častí infraštruktúry, kde dochádza k~významným bezpečnostným incidentom je počítačová sieť, bez ktorej by dnes informačné technológie nevedeli fungovať. Preto sa táto práca bude zaoberať práve ňou, keďže je vstupnou bránou do systémov a jej vyradením alebo zneužitím prichádzajú organizácie o~finančné prostriedky, citlivé dáta a dôveru užívateľov.

Výsledkom tejto práce bude aplikácia overujúca nastavenia sieťových zariadení prevažne v~lokálnej sieti, ktorá umožňuje zjednať nápravu na základe nájdených nedostatkov. Výhodou oproti existujúcim riešeniam bude otvorenosť kódu a modularita, ktorá umožní rozšírenie aplikácie na sieťové zariadenia rôznych výrobcov. Dôležitým výstupom bude taktiež zoznam bezpečnostných a prevádzkových odporučaní vychádzajúcich z~rôznych štandardov a odporučaní, ktoré môžu byť v~budúcnosti použité ďalšími užívateľmi aplikácie pri zostavovaní modulov pre zariadenia rôznych výrobcov. Vytvorený zoznam odporúčaní bude obsahovať zatriedenie odporúčaní podľa závažnosti, čo súčasné riešenia neponúkajú. Odporúčania ako aj výsledná aplikácia počíta s~rozdelením odporúčaní a nálezov podľa umiestnenia zariadenia v~hierarchickom modely siete, aby nedochádzalo ku falošne pozitívnym správam. Jednou z~kľúčových vlastností je bezplatnosť, keďže podľa zistení takmer polovica útokov smeruje na malé firmy, ktoré bezpečnosť často neriešia z~finančnej náročnosti programov na detekciu bezpečnostných nedostatkov.    
