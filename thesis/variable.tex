%% Nastavení jazyka při sazbě.
% Pro sazbu češtiny je použit mezinárodní balíček 'babel', použití
% národního balíčku 'czech', ve spojení s programy 'cslatex' a
% 'pdfcslatex' není od verze 3.0 podporován a nedoporučujeme ho.
\usepackage[
%%Nastavení balíčku babel (!!! pri zmene jazyka je potreba zkompilovat dvakrat !!!)
  main=slovak,english       % originální jazyk je čeština (výchozí), překlad je anglicky
  %main=slovak,english      % originální jazyk je slovenčina, překlad je anglicky
  %main=english,czech       % originální jazyk je angličtina, překlad je česky
]{babel}    					% Balíček pro sazbu různojazyčných dokumentů; kompilovat (pdf)latexem!

\usepackage{lmodern}	% vektorové fonty Latin Modern, nástupce půvoních Knuthových Computern Modern fontů
\usepackage{textcomp} % Dodatečné symboly
\usepackage[LGR,T1]{fontenc}  % Kódování fontu -- mj. kvůli správným vzorům pro dělení slov

\usepackage[
%% Z následujících voleb lze použít pouze jednu
  %semestral,					%	sazba semestrálního práce (nesází se abstrakty, prohlášení, poděkování)
  %bachelor,					%	sazba bakalářské práce
  diploma,						% sazba diplomové práce
  %treatise,          % sazba pojednání o dizertační práci
  %phd,               % sazba dizertační práce
%% Z následujících voleb lze použít pouze jednu
% left,               % Rovnice a popisky plovoucich objektů budou %zarovnány vlevo
  center,             % Rovnice a popisky plovoucich objektů budou zarovnány na střed (vychozi)
]{thesis}   % Balíček pro sazbu studentských prací
                      % Musí být vložen až jako poslední, aby
                      % ostatní balíčky nepřepisovaly jeho příkazy


%% Jméno a příjmení autora ve tvaru
%  [tituly před jménem]{Křestní}{Příjmení}[tituly za jménem]
% Pokud osoba nemá titul před/za jménem, smažte celý řetězec '[...]'
\autor[Bc.]{Juraj}{Korček}


%% Pohlaví autora/autorky
% Číselná hodnota: 1...žena, 0...muž
\autorpohlavi{0}

%% Jméno a příjmení vedoucího/školitele včetně titulů
%  [tituly před jménem]{Křestní}{Příjmení}[tituly za jménem]
% Pokud osoba nemá titul před/za jménem, smažte celý řetězec '[...]'
\vedouci[doc.\ Ing.]{Jan}{Jeřábek}[PhD.]

%% Jméno a příjmení oponenta včetně titulů
%  [tituly před jménem]{Křestní}{Příjmení}[tituly za jménem]
% Pokud osoba nemá titul před/za jménem, smažte celý řetězec '[...]'
% Uplatní se pouze v prezentaci k obhajobě;
% v případě, že nechcete, aby se na titulním snímku prezentace zobrazoval oponent, pouze příkaz zakomentujte;
% u obhajoby semestrální práce se oponent nezobrazuje
\oponent[doc.\ Mgr.]{Křestní}{Příjmení}[Ph.D.]

%% Název práce:
%  První parametr je název v originálním jazyce,
%  druhý je překlad v angličtině nebo češtině (pokud je originální jazyk angličtina)
\nazev{Aplikace pro generování a ověřování konfigurací síťových zařízení}{Application generating and verifying configurations of network devices}

%% Označení oboru studia
% První parametr je obor v originálním jazyce,
% druhý parametr je překlad v angličtině nebo češtině
\oborstudia{Informační bezpečnost}{Information Security}

%% Označení ústavu
% První parametr je název ústavu v originálním jazyce,
% druhý parametr je překlad v angličtině nebo češtině
%\ustav{Ústav automatizace a měřicí techniky}{Department of Control and Instrumentation}
%\ustav{Ústav biomedicínského inženýrství}{Department of Biomedical Engineering}
%\ustav{Ústav elektroenergetiky}{Department of Electrical Power Engineering}
%\ustav{Ústav elektrotechnologie}{Department of Electrical and Electronic Technology}
%\ustav{Ústav fyziky}{Department of Physics}
%\ustav{Ústav jazyků}{Department of Foreign Languages}
%\ustav{Ústav matematiky}{Department of Mathematics}
%\ustav{Ústav mikroelektroniky}{Department of Microelectronics}
%\ustav{Ústav radioelektroniky}{Department of Radio Electronics}
%\ustav{Ústav teoretické a experimentální elektrotechniky}{Department of Theoretical and Experimental Electrical Engineering}
\ustav{Ústav telekomunikací}{Department of Telecommunications}
%\ustav{Ústav výkonové elektrotechniky a elektroniky}{Department of Power Electrical and Electronic Engineering}

%% Označení fakulty
% První parametr je název fakulty v originálním jazyce,
% druhý parametr je překlad v angličtině nebo v češtině
%\fakulta{Fakulta architektury}{Faculty of Architecture}
\fakulta{Fakulta elektrotechniky a~komunikačních technologií}{Faculty of Electrical Engineering and~Communication}
%\fakulta{Fakulta chemická}{Faculty of Chemistry}
%\fakulta{Fakulta informačních technologií}{Faculty of Information Technology}
%\fakulta{Fakulta podnikatelská}{Faculty of Business and Management}
%\fakulta{Fakulta stavební}{Faculty of Civil Engineering}
%\fakulta{Fakulta strojního inženýrství}{Faculty of Mechanical Engineering}
%\fakulta{Fakulta výtvarných umění}{Faculty of Fine Arts}

\logofakulta[loga/FEKT_zkratka_barevne_PANTONE_CZ]{loga/UTKO_color_PANTONE_CZ}


%% Rok obhajoby
\rok{2020}
\datum{6.\,1.\,2020} % Datum se uplatní pouze v prezentaci k obhajobě

%% Místo obhajoby
% Na titulních stránkách bude automaticky vysázeno VELKÝMI písmeny
\misto{Brno}

%% Abstrakt
\abstrakt{%
Cieľom tejto diplomovej práce je návrh a následná implementácia programu na nájdenie bezpečnostných a prevádzkových nedostatkov v sieťových zariadeniach, ako aj ich náprava pomocou generovania opravnej konfigurácie. Z dôvodu nedostatočného zabezpečenia a nesprávnej konfigurácie sú mnohé zariadenia v sieti často nevedome vystavené riziku bezpečnostného incidentu. Z tohto dôvodu program porovnáva ich nastavenia s rôznymi štandardmi, odporúčaniami a osvedčenými postupmi a vytvára správu s nálezmi, aby bolo možné tieto nedostatky odstrániť pomocou automaticky vygenerovanej nápravy alebo manuálne, pokiaľ automatická náprav nie je možná. Program využíva na nájdenie problémových nastavení regulárne výrazy, pomocou ktorých hľadá nedostatky vo vyexportovaných konfiguráciách. Jeho implementácia je v jazyku Python a využíva sa aj značkovací jazyk YAML. Vedľajším produktom práce je aj kontrolný zoznam, ktorým sa dá riadiť pri zostavovaní modulov pre podporu ďalších výrobcov, a tým rozšíriť program.
}{%
The aim of this master's thesis is a design and implementation of a program for finding security and operational deficiencies of network devices and afterwards, resolving them by generating corrective configuration. Due to a lack of security and misconfiguration, there are a lot of devices exposed to the risk of a security incident. Therefore, the program compares settings with various standards, recommendations, and best practices and generates a report with findings. Afterwards, deficiencies can be eliminated by automatic resolution or manually if automatic resolving is not possible. The program uses regular expressions to find problem settings in previously exported configurations. Implementation is written in Python, and YAML markup language is used too. Another output of this thesis is a checklist, which can be used for the creation of future modules for support of other network device vendors and thus extend the program.
}

%% Klíčová slova
\klicovaslova{%
sieť, zariadenie, smerovač, prepínač, bezpečnosť, overenie, kontrola, audit, generovanie, konfigurácia, nastavenie, python, yaml 
}{%
network, device, security, router, switch, verification, check, audit, generation, configuration, setting, python, yaml
}

%% Poděkování
\podekovanitext{%
Rád by som poďakoval vedúcemu diplomovej práce pánovi doc. Ing. Janovi Jeřábkovi Ph.D.\ za odborné vedenie, konzultácie, trpezlivosť a podnetné návrhy k~práci.
}%

% Zrušení sazby poděkování projektu SIX, pokud není nutné
%\renewcommand\vytvorpodekovaniSIX\relax