% Soubory musí být v kódování, které je nastaveno v příkazu \usepackage[...]{inputenc}

\documentclass[%
%  draft,    				  % Testovací překlad
  12pt,       				% Velikost základního písma je 12 bodů
  %a4paper,    				% Formát papíru je A4
	t,                  % obsah slajdů nebude centrovaný, nýbrž budou začínat od hora
	aspectratio=1610,   % poměr stran bude 16:10 (všechny projektory v učebnách na Technické 12),
	                    % další volby jsou 43, 149, 169, 54, 32.
%% Z následujicich voleb lze použít maximálně jednu:
%	dvipdfm  						% výstup bude zpracován programem 'dvipdfm' do PDF
%	dvips	  						% výstup bude zpracován programem 'dvips' do PS
%	pdftex							% překlad bude proveden programem 'pdftex' do PDF (výchozí)
	unicode,						% Záložky a informace budou v kódování unicode
]{beamer}				    	% Dokument třídy 'zpráva'
%\usepackage{etex}

\usepackage[utf8]		%	Kódování zdrojových souborů je v UTF-8
	{inputenc}					% Balíček pro nastavení kódování zdrojových souborů
\usepackage{graphicx} % Balíček 'graphicx' pro vkládání obrázků
											% Nutné pro vložení log školy a fakulty

\usepackage[
	nohyperlinks				% Nebudou tvořeny hypertextové odkazy do seznamu zkratek
]{acronym}						% Balíček 'acronym' pro sazby zkratek a symbolů
											% Nutné pro použití prostředí 'seznamzkratek' balíčku 'thesis'

%% Balíček hyperref je volán třídou beamer automaticky
%\usepackage[
%	breaklinks=true,		% Hypertextové odkazy mohou obsahovat zalomení řádku
%	hypertexnames=false % Názvy hypertextových odkazů budou tvořeny
%											% nezávisle na názvech TeXu
%]{hyperref}						% Balíček 'hyperref' pro sazbu hypertextových odkazů
%											% Nutné pro použití příkazu 'nastavenipdf' balíčku 'thesis'

%\usepackage{pdfpages} % Balíček umožňující vkládat stránky z PDF souborů
                      %% Nutné při vkládání titulních listů a zadání přímo
                      %% ve formátu PDF z informačního systému

\usepackage{cmap} 		% Balíček cmap zajišťuje, že PDF vytvořené `pdflatexem' je
											% plně "prohledávatelné" a "kopírovatelné"

\usepackage{upgreek}	% Balíček pro sazbu stojatých řeckých písmem
											% např. stojaté pí: \uppi
											% např. stojaté mí: \upmu (použitelné třeba v mikrometrech)
											% pozor, grafická nekompatibilita s fonty typu Computer Modern!

%\usepackage{amsmath}

\usepackage{booktabs} % Balíček, který umožňuje v tabulce používat
                      % příkazy \toprule, \midrule, \bottomrule

\usepackage{float}
\usepackage{csvsimple}
\usepackage{longtable,array,ragged2e}
\usepackage{caption}
\newcommand*{\headentry}[2]{\multicolumn{1}{#1}{\centering\arraybackslash\bfseries #2}}
\newcolumntype{P}[1]{>{\RaggedRight\arraybackslash}p{#1}}
\newcolumntype{L}[1]{>{\raggedright\let\newline\\\arraybackslash\hspace{0pt}}m{#1}}

\newcolumntype{C}[1]{>{\centering\let\newline\\\arraybackslash\hspace{0pt}}m{#1}}
\usepackage{booktabs}
\usepackage{colortbl}
\usepackage{adjustbox}
\usepackage{changepage}
\usepackage{multirow}
\usepackage{hhline}

%%%%%%%%%%%%%%%%%%%%%%%%%%%%%%%%%%%%%%%%%%%%%%%%%%%%%%%%%%%%%%%%%
%%%%%%      Definice informací o dokumentu             %%%%%%%%%%
%%%%%%%%%%%%%%%%%%%%%%%%%%%%%%%%%%%%%%%%%%%%%%%%%%%%%%%%%%%%%%%%%


%% Nastavení jazyka při sazbě.
% Pro sazbu češtiny je použit mezinárodní balíček 'babel', použití
% národního balíčku 'czech', ve spojení s programy 'cslatex' a
% 'pdfcslatex' není od verze 3.0 podporován a nedoporučujeme ho.
\usepackage[
%%Nastavení balíčku babel (!!! pri zmene jazyka je potreba zkompilovat dvakrat !!!)
  main=slovak,english       % originální jazyk je čeština (výchozí), překlad je anglicky
  %main=slovak,english      % originální jazyk je slovenčina, překlad je anglicky
  %main=english,czech       % originální jazyk je angličtina, překlad je česky
]{babel}    					% Balíček pro sazbu různojazyčných dokumentů; kompilovat (pdf)latexem!

\usepackage{lmodern}	% vektorové fonty Latin Modern, nástupce půvoních Knuthových Computern Modern fontů
\usepackage{textcomp} % Dodatečné symboly
\usepackage[LGR,T1]{fontenc}  % Kódování fontu -- mj. kvůli správným vzorům pro dělení slov

\usepackage[
%% Z následujících voleb lze použít pouze jednu
  %semestral,					%	sazba semestrálního práce (nesází se abstrakty, prohlášení, poděkování)
  %bachelor,					%	sazba bakalářské práce
  diploma,						% sazba diplomové práce
  %treatise,          % sazba pojednání o dizertační práci
  %phd,               % sazba dizertační práce
%% Z následujících voleb lze použít pouze jednu
% left,               % Rovnice a popisky plovoucich objektů budou %zarovnány vlevo
  center,             % Rovnice a popisky plovoucich objektů budou zarovnány na střed (vychozi)
]{thesis}   % Balíček pro sazbu studentských prací
                      % Musí být vložen až jako poslední, aby
                      % ostatní balíčky nepřepisovaly jeho příkazy


%% Jméno a příjmení autora ve tvaru
%  [tituly před jménem]{Křestní}{Příjmení}[tituly za jménem]
% Pokud osoba nemá titul před/za jménem, smažte celý řetězec '[...]'
\autor[Bc.]{Juraj}{Korček}


%% Pohlaví autora/autorky
% Číselná hodnota: 1...žena, 0...muž
\autorpohlavi{0}

%% Jméno a příjmení vedoucího/školitele včetně titulů
%  [tituly před jménem]{Křestní}{Příjmení}[tituly za jménem]
% Pokud osoba nemá titul před/za jménem, smažte celý řetězec '[...]'
\vedouci[doc.\ Ing.]{Jan}{Jeřábek}[PhD.]

%% Jméno a příjmení oponenta včetně titulů
%  [tituly před jménem]{Křestní}{Příjmení}[tituly za jménem]
% Pokud osoba nemá titul před/za jménem, smažte celý řetězec '[...]'
% Uplatní se pouze v prezentaci k obhajobě;
% v případě, že nechcete, aby se na titulním snímku prezentace zobrazoval oponent, pouze příkaz zakomentujte;
% u obhajoby semestrální práce se oponent nezobrazuje
\oponent[doc.\ Mgr.]{Křestní}{Příjmení}[Ph.D.]

%% Název práce:
%  První parametr je název v originálním jazyce,
%  druhý je překlad v angličtině nebo češtině (pokud je originální jazyk angličtina)
\nazev{Aplikace pro generování a ověřování konfigurací síťových zařízení}{Application generating and verifying configurations of network devices}

%% Označení oboru studia
% První parametr je obor v originálním jazyce,
% druhý parametr je překlad v angličtině nebo češtině
\oborstudia{Informační bezpečnost}{Information Security}

%% Označení ústavu
% První parametr je název ústavu v originálním jazyce,
% druhý parametr je překlad v angličtině nebo češtině
%\ustav{Ústav automatizace a měřicí techniky}{Department of Control and Instrumentation}
%\ustav{Ústav biomedicínského inženýrství}{Department of Biomedical Engineering}
%\ustav{Ústav elektroenergetiky}{Department of Electrical Power Engineering}
%\ustav{Ústav elektrotechnologie}{Department of Electrical and Electronic Technology}
%\ustav{Ústav fyziky}{Department of Physics}
%\ustav{Ústav jazyků}{Department of Foreign Languages}
%\ustav{Ústav matematiky}{Department of Mathematics}
%\ustav{Ústav mikroelektroniky}{Department of Microelectronics}
%\ustav{Ústav radioelektroniky}{Department of Radio Electronics}
%\ustav{Ústav teoretické a experimentální elektrotechniky}{Department of Theoretical and Experimental Electrical Engineering}
\ustav{Ústav telekomunikací}{Department of Telecommunications}
%\ustav{Ústav výkonové elektrotechniky a elektroniky}{Department of Power Electrical and Electronic Engineering}

%% Označení fakulty
% První parametr je název fakulty v originálním jazyce,
% druhý parametr je překlad v angličtině nebo v češtině
%\fakulta{Fakulta architektury}{Faculty of Architecture}
\fakulta{Fakulta elektrotechniky a~komunikačních technologií}{Faculty of Electrical Engineering and~Communication}
%\fakulta{Fakulta chemická}{Faculty of Chemistry}
%\fakulta{Fakulta informačních technologií}{Faculty of Information Technology}
%\fakulta{Fakulta podnikatelská}{Faculty of Business and Management}
%\fakulta{Fakulta stavební}{Faculty of Civil Engineering}
%\fakulta{Fakulta strojního inženýrství}{Faculty of Mechanical Engineering}
%\fakulta{Fakulta výtvarných umění}{Faculty of Fine Arts}

\logofakulta[loga/FEKT_zkratka_barevne_PANTONE_CZ]{loga/UTKO_color_PANTONE_CZ}


%% Rok obhajoby
\rok{2020}
\datum{6.\,1.\,2020} % Datum se uplatní pouze v prezentaci k obhajobě

%% Místo obhajoby
% Na titulních stránkách bude automaticky vysázeno VELKÝMI písmeny
\misto{Brno}

%% Abstrakt
\abstrakt{%
Cieľom tejto diplomovej práce je návrh a následná implementácia programu na nájdenie bezpečnostných a prevádzkových nedostatkov v sieťových zariadeniach, ako aj ich náprava pomocou generovania opravnej konfigurácie. Z dôvodu nedostatočného zabezpečenia a nesprávnej konfigurácie sú mnohé zariadenia v sieti často nevedome vystavené riziku bezpečnostného incidentu. Z tohto dôvodu program porovnáva ich nastavenia s rôznymi štandardmi, odporúčaniami a osvedčenými postupmi a vytvára správu s nálezmi, aby bolo možné tieto nedostatky odstrániť pomocou automaticky vygenerovanej nápravy alebo manuálne, pokiaľ automatická náprav nie je možná. Program využíva na nájdenie problémových nastavení regulárne výrazy, pomocou ktorých hľadá nedostatky vo vyexportovaných konfiguráciách. Jeho implementácia je v jazyku Python a využíva sa aj značkovací jazyk YAML. Vedľajším produktom práce je aj kontrolný zoznam, ktorým sa dá riadiť pri zostavovaní modulov pre podporu ďalších výrobcov, a tým rozšíriť program.
}{%
The aim of this master's thesis is a design and implementation of a program for finding security and operational deficiencies of network devices and afterwards, resolving them by generating corrective configuration. Due to a lack of security and misconfiguration, there are a lot of devices exposed to the risk of a security incident. Therefore, the program compares settings with various standards, recommendations, and best practices and generates a report with findings. Afterwards, deficiencies can be eliminated by automatic resolution or manually if automatic resolving is not possible. The program uses regular expressions to find problem settings in previously exported configurations. Implementation is written in Python, and YAML markup language is used too. Another output of this thesis is a checklist, which can be used for the creation of future modules for support of other network device vendors and thus extend the program.
}

%% Klíčová slova
\klicovaslova{%
sieť, zariadenie, smerovač, prepínač, bezpečnosť, overenie, kontrola, audit, generovanie, konfigurácia, nastavenie, python, yaml 
}{%
network, device, security, router, switch, verification, check, audit, generation, configuration, setting, python, yaml
}

%% Poděkování
\podekovanitext{%
Rád by som poďakoval vedúcemu diplomovej práce pánovi doc. Ing. Janovi Jeřábkovi Ph.D.\ za odborné vedenie, konzultácie, trpezlivosť a podnetné návrhy k~práci.
}%

% Zrušení sazby poděkování projektu SIX, pokud není nutné
%\renewcommand\vytvorpodekovaniSIX\relax      % v tomto souboru doplňte údaje o sobě, o názvu práce...

%%%%%%%%%%%%%%%%%%%%%%%%%%%%%%%%%%%%%%%%%%%%%%%%%%%%%%%%%%%%%%%%%%%%%%%%

%%%%%%%%%%%%%%%%%%%%%%%%%%%%%%%%%%%%%%%%%%%%%%%%%%%%%%%%%%%%%%%%%%%%%%%%
%%%%%%     Nastavení polí ve Vlastnostech dokumentu PDF      %%%%%%%%%%%
%%%%%%%%%%%%%%%%%%%%%%%%%%%%%%%%%%%%%%%%%%%%%%%%%%%%%%%%%%%%%%%%%%%%%%%%
%% Při vloženém balíčku 'hyperref' lze použít příkaz '\nastavenipdf'
\nastavenipdf
%  Nastavení polí je možné provést také ručně příkazem:
%\hypersetup{
%  pdftitle={Název studentské práce},    	% Pole 'Document Title'
%  pdfauthor={Autor studenstké práce},   	% Pole 'Author'
%  pdfsubject={Typ práce}, 						  	% Pole 'Subject'
%  pdfkeywords={Klíčová slova}           	% Pole 'Keywords'
%}
\hypersetup{pdfpagemode=FullScreen}       % otevření rovnou v režimu celé obrazovky
%%%%%%%%%%%%%%%%%%%%%%%%%%%%%%%%%%%%%%%%%%%%%%%%%%%%%%%%%%%%%%%%%%%%%%%

\usetheme{VUT} 				% barvy a rozložení prezentace odpovídající VUT FEKT
% alternativne lze pouzit jina berevna temata, napr. 
%\usetheme{Darmstadt} \usecolortheme{default2}
% ale bez zaruky
\logohlavicka					% vytvoření zkráceného loga VUT FEKT (FEEC) v hlavičce slajdu, nechte odkomentované


\begin{document}

% v pripade zakomentovani se zobrazi v pravem dolnim rohu slajdu klikatelne navigacni symboly 
\vypninavigacnisymboly

% snimek s titulni strankou vysazen bez hornich, dolnich a postranich list (volba plain),
% neni tak vysazen ani nadpis snimku
\vytvortitulku

%%%%%%%%%%%%%%%%%%%%%%%%%%%%%%%%%%%%%%%%%%%%%%%%%%%%%%%%%%%%%%%%%%%%%%%
% 1.snimek s cili (zadanim) prace
\begin{frame} 
	% nadpis snímku
	\frametitle{Ciele semestrálne práce}
	\begin{itemize}
			\vspace{1em}
			\item Zoznámenie s problematikou sieťových zariadení, spôsobom ich konfigurácie.\vspace{1em}
			\item Preštudovať problematiku osvedčených postupov konfigurácií s ohľadom na bezpečnosť fungovania zariadení v sieti.\vspace{1em}
			\item Vybrať vhodné programovacie prostredie.\vspace{1em}
			\item Vypracovať teoretickú časť.\vspace{1em}
			\item Navrhnúť a popísať štruktúru aplikácie a základný popis komponentov s predpokladanou funkcionalitou.\vspace{1em}
			\item Vlastné riešenie komponentov mierne rozpracovať.

	\end{itemize}


%Navrhněte systém či aplikaci, která bude umět pro vybranou množinu síťových zařízení vytvářet přednastavené parametry nastavení, které bude možné na dané síťové zařízení aplikovat. Dále musí daná aplikace umět verifikovat správnost existujících konfigurací, upozornit na případné nedostatky a i konfiguraci modifikovat tak, aby splňovala hlavní bezpečnostní a provozní standardy a doporučení. Fungování aplikace ověřte na testovacích vzorcích síťových konfigurací různých zařízení z několika různých sítí a případně i různých výrobců.



\end{frame}

%%%%%%%%%%%%%
\begin{frame} 
	\frametitle{Úvod do problematiky}
	\begin{itemize}
		\vspace{1em}
		\item Prístup pri ktorom sieťové zariadenia musia plniť v prvom rade základnú funkcionalitu, bezpečnosť a osvedčená konfigurácia druhoradá.
		
		\item Domnienka všetko funguje, konfigurácia robí čo má.
		
		\item Potreba nastaviť zrkadlo\,--\,audit, súlad s osvedčenými postupmi.
		
		\item Skontrolovať všetko\,--\,manuálne nemožné, reprezentatívna vzorka nedostatočná.
		
		\item Zjednanie nápravy pri nájdení nedostatkov\,--\,automaticky/semiautomaticky.
		
		\item Povedomie o verziách sieťových operačných systémov v topológii a ich chybách (bezpečnostných).
	\end{itemize}
	
\end{frame} 


%%%%%%%%%%%%%
\begin{frame} 
	\frametitle{Zoznam odporúčaní}
		\begin{itemize}
				\item 120 odporúčaní
				\item Zdroj viacero benchmarkov, odporúčaní a štandardov.
				\item Rozšírenie o severity (Critical, High, Medium, Low, Notice).
				\item Rozšírenie o facility layer\,--\,rozšírený hierarchický model.
				\item Vytvorené mapovanie na zariadenia Cisco.
		\end{itemize}

	\begin{table}
	\scriptsize
		\begin{tabular}[!htbp]{|L{11em}L{10em}C{3.1em}C{3.3em}C{13em}|}
		
		\hline	
			
		Útok / Problém	&Mitigácia / Nastavenie	&Plane	&Severity	&Facility layer\\ \hhline{=====} 

			
		Pripojenie prepínaču na koncový prístupový port & BPDU protection (BPDU guard)&Control	&Critical	&Distribution
		
		Collapsed Distribution Access
		
		Access \\ \hline
		\end{tabular}
	\end{table}
\end{frame}


%%%%%%%%%%%%%
\begin{frame} 
	\frametitle{Vlastnosti a rozdiel oproti existujúcim riešeniam}
	
	\begin{itemize}
		\item Modularita
		%každé overenie a náprava bude v~zvlášť súbore prihliadajúc na výrobcu a operačný systém pre ktoré je určené.
		\vspace{0.5em}
		\item Prispôsobenie na ďalších výrobcov
		%definovanie API pre moduly na budúcu podporu pre zariadenia od viacerých výrobcov a ich operačných systémov.
		\vspace{0.5em}
		\item Zjednanie nápravy
		%pri nájdení nedostatku vygenerovanie opravného nastavenia.
		\vspace{0.5em}
		\item Podpora IPv6
		%detekcia zlého alebo chýbajúceho nastavenia a následná náprava pre protokol IPv6.
		\vspace{0.5em}
		\item Hierarchický model
		%skenovanie nedostatkov typických pre jednotlivé vrstvy, v~ktorých sa zariadenia nachádzajú a tým zníženie falošne pozitívnych varovaní.
		\vspace{0.5em}
		\item Definovanie závažnosti
		%každý nájdený nedostatok je hodnotený na 4 stupňovej škále.
		\vspace{0.5em}
		\item Personalizácia
		%definovanie modulov, ktoré sa spustia pre jednotlivé vrstvy, zmena závažnosti nájdených nedostatkov v~konfiguračných súboroch.
		\vspace{0.5em}
		\item Zoznam útokov a problémov aktuálne bežiacej verzie operačného systému.
		\vspace{0.5em}
		\item Vygenerovanie správy s~nedostatkami.
	\end{itemize}
	
\end{frame}


%%%%%%%%%%%%%
\begin{frame} 
\frametitle{Programovacie prostredie}
	\begin{itemize}
		\vspace{2em}
		\item Python \vspace{2em}
		\item YAML \vspace{2em}
		\item Regex
	\end{itemize}
\end{frame}


%%%%%%%%%%%%%
\begin{frame} 
	\frametitle{Záver}
	
	\begin{itemize}
		\item Dosiahnuté ciele	
			\begin{itemize}
				\item Naštudovaná problematika a spracovaná teória.
				\item Vytvorený zoznam odporúčaní a namapovanie príkazov pre zariadenia Cisco.
				\item Vybrané programovacie prostredie.
				\item Vytvorený návrh.
				\item Vytvorené konfiguračné súbory YAML.
			\end{itemize}
		\item Budúce smerovanie
			\begin{itemize}
				\item Naprogramovanie modulov a aplikácie.
				\item Otestovanie na konfiguráciách z virtuálnych zariadení (GNS3) a na zariadeniach z reálnej prevádzky.
				\item Vytvorenie dokumentácie a definovanie API pre moduly.
			\end{itemize}
	\end{itemize}

\end{frame}


% podekovani
\begin{frame}[c] 
% bez nadpisu snimku
	\frametitle{\mbox{ }}
	\begin{center}
		{\Huge Ďakujem za pozornosť!}
	\end{center}
\end{frame}

\end{document}
