%%%%%%%%%%%%%%%%%%%%%%%%%%%%%%%%%%%%%%%%%%%%%%%%%%%%%%%%%%%%%%%%%%%%%%
%%                                                                  %%
%%  This is a LaTeX2e table fragment exported from Gnumeric.        %%
%%                                                                  %%
%%%%%%%%%%%%%%%%%%%%%%%%%%%%%%%%%%%%%%%%%%%%%%%%%%%%%%%%%%%%%%%%%%%%%%
Riadok č.	&Útok / Problém	&Mitigácia / Nastavenie	&Plane 	&Severity	&Facility layer	&Zdroj\\
1	&MAC Spoofing, MAC Flooding 	&Definovanie maximálne 1 MAC adresy na port, priradenie MAC adresy na port	&C	&C	&DIST
COLDISTACC
ACC	&CCNA Routing and Switching Study Guide \cite{Lammle2013}\\
2	&MAC Spoofing, MAC Flooding 	&Nastavenie režimu narušenia, ktorý vypne port alebo informuje správcu o pripojení nepovoleného zariadenia	&C	&H	&DIST
COLDISTACC
ACC	&CCNA Routing and Switching Study Guide \cite{Lammle2013}\\
3	&Využívanie siete nepovolenými používateľmi	&Zapnutie 802.1x 	&C	&H	&DIST
COLDISTACC
ACC	&Lan Switch Security \cite{Vyncke2008}
Cisco IOS 11 - IEEE 802.1x, autentizace k portu, MS IAS \cite{Bouska20071}
Cisco IOS 12 - IEEE 802.1x a pokročilejší funkce  \cite{Bouska2007} \\
4	&Útok hrubou silou hádaním prístupových údajov pre 802.1x 	&Limitovanie maximálneho počtu neúspešných pokusov o autentizáciu 802.1x	&C	&H	&DIST
COLDISTACC
ACC	&Lan Switch Security \cite{Vyncke2008}
Cisco IOS 11 - IEEE 802.1x, autentizace k portu, MS IAS \cite{Bouska20071}
Cisco IOS 12 - IEEE 802.1x a pokročilejší funkce  \cite{Bouska2007} \\
5	&DHCP spoofing	&DHCP snooping, IPv6 Snooping, DHCPv6 Guard	&C	&C	&DIST
COLDISTACC
ACC	&Lan Switch Security \cite{Vyncke2008}
Cisco Guide to Harden Cisco IOS Devices \cite{Singh2018} IPv6 First-Hop Security Configuration Guide \cite{zXCpMaLbN1J7D1z2}\\
6	&Příliš veľa DHCP paketov, zaplavenie DHCP paketmi	&Odmedziť počet DHCP paketov na nedôverihodných rozhraniach	&C	&M	&DIST
COLDISTACC
ACC	&Lan Switch Security \cite{Vyncke2008}
Cisco Guide to Harden Cisco IOS Devices \cite{Singh2018}
IPv6 First-Hop Security Configuration Guide \cite{zXCpMaLbN1J7D1z2}\\
7	&ARP Spoofing	&Dynamic ARP Inspection	&C	&C	&DIST
COLDISTACC
ACC	&Cisco CCNA Security Study Guide \cite{McMillan2018}\\
8	&IP spoofing	&IPv4/IPv6 Source Guard	&C	&C	&DIST
COLDISTACC
ACC	&Cisco Guide to Harden Cisco IOS Devices \cite{Singh2018}
IPv6 First-Hop Security Configuration Guide \cite{zXCpMaLbN1J7D1z2}\\
9	&IPv6 ND Spoofing	&IPv6 ND Inspection	&C	&C	&DIST
COLDISTACC
ACC	&Bezpečné IPv6: zkrocení zlých směrovačů \cite{Podermanski1222015}
Bezpečné IPv6 : směrovač se hlásí \cite{Gregr522015}
IPv6 First-Hop Security Configuration Guide \cite{zXCpMaLbN1J7D1z2}\\
10	&Rogue RA
RA Flood
Route Information Option injection
RA RouterLifeTime=0
	&RA Guard	&C	&C	&DIST
COLDISTACC
ACC	&Bezpečné IPv6: zkrocení zlých směrovačů \cite{Podermanski1222015}
Bezpečné IPv6 : směrovač se hlásí \cite{Gregr522015}
IPv6 First-Hop Security Configuration Guide \cite{zXCpMaLbN1J7D1z2}\\
11	&Mobilné zariadenia pripojené bezdôtovo spotrebovávajú veľa energie kvôli častým RA správam	&RA Throttling	&C	&L	&DIST
COLDISTACC
ACC	&Bezpečné IPv6: trable s multicastem \cite{Podermanski532015}
ND on wireless links and/or with sleeping nodes \cite{o31nYG4kn98wWNRS}\\
12	&Vyčerpanie cache susedov	&Statický záznam pre kritické zariadenia (servery) spájajúce IP a MAC adresu a VLAN
	&C	&C	&DIST
COLDISTACC
ACC	&Bezpečné IPv6: když dojde keš \cite{Podermanski1232015}
Bezpečné IPv6: když dojde keš – obrana \cite{Podermanski1932015}
\\
13	&Vyčerpanie cache susedov	&Na zabránenie vzdialeného útoku na cache susedov cez internet je potreba nastaviť ACL, kde povolujeme iba komunikáciu s cieľovými IPv6 adresami, ktoré sa nachádzajú v našej sieti	&C	&C	&CORE/EDGE
COLCOREDIST
COLALL	&Bezpečné IPv6: když dojde keš \cite{Podermanski1232015}
Bezpečné IPv6: když dojde keš – obrana \cite{Podermanski1932015}
\\
14	&Vyčerpanie cache susedov	&IP destination Guard (First Hop Security)


	&C	&C	&DIST
COLDISTACC
ACC	&Bezpečné IPv6: když dojde keš \cite{Podermanski1232015}
Bezpečné IPv6: když dojde keš – obrana \cite{Podermanski1932015}
\\
15	&Vyčerpanie cache susedov	&Limitovanie času IPv6 adresy v cache susedov	&C	&C	&DIST
COLDISTACC
ACC	&Bezpečné IPv6: když dojde keš \cite{Podermanski1232015}
Bezpečné IPv6: když dojde keš – obrana \cite{Podermanski1932015}
\\
%%%%%%%%%%%%%%%%%%%%%%%%%%%%%%%%%%%%%%%%%%%%%%%%%%%%%%%%%%%%%%%%%%%%%%
%%                                                                  %%
%%  This is a LaTeX2e table fragment exported from Gnumeric.        %%
%%                                                                  %%
%%%%%%%%%%%%%%%%%%%%%%%%%%%%%%%%%%%%%%%%%%%%%%%%%%%%%%%%%%%%%%%%%%%%%%
Riadok č.	&Útok / Problém	&Mitigácia / Nastavenie	&Plane 	&Severity	&Facility layer	&Zdroj\\
1	&IP spoofing	&Špecifikácia ACL na zakázanie a logovanie privátnych a špeciálnych IP adries z RFC 1918, RFC 3330	&C	&C	&CORE/EDGE
COLCOREDIST
COLALL	&Network Security Auditing \cite{Jackson2010}
Cisco Guide to Harden Cisco IOS Devices \cite{Singh2018}
CIS Cisco IOS 15 Benchmark \cite{CIS_DrTLsgXv24lxeIIM}\\
2	&IP spoofing	&Špecifikácia ACL na zakázanie a logovanie špeciálnych IPv6 adries z RFC 5156	&C	&C	&CORE/EDGE
COLCOREDIST
COLALL	&Network Security Auditing \cite{Jackson2010}
Cisco Guide to Harden Cisco IOS Devices \cite{Singh2018}
CIS Cisco IOS 15 Benchmark \cite{CIS_DrTLsgXv24lxeIIM}\\
3	&IPv6 Next Header  a IPv6 Fragmentation útok	&ACL blokujúce nerozpoznateľne rozšírené hlavičky	&C	&C	&A	&Bezpečné IPv6: trable s hlavičkami \cite{Podermanski1922015}
Bezpečné IPv6: vícehlavý útočník \cite{Gregr2622015}\\
4	&DOS útok alebo pokus o prístup k tomu, čo nie je povolené	&Logovanie pravidiel zahodenia paketov v ACL	&M	&M	&A	&Hardening Cisco Routers \cite{Akin2002}\\
5	&Packety budú spracovávané v CPU, ktoré môže byť preťažené a môže byť zmenené smerovanie na obídenie bezpečnostnej kontroly	&Zahadzovanie IPv4 paketov s rozšírenou hlavičkou (IP Options filtering)	&C	&C	&CORE/EDGE
DIST
COLCOREDIST
COLDISTACC
COLALL	&Cisco Guide to Harden Cisco IOS Devices \cite{Singh2018}\\
6	&Komplexné bezpečnostné hrozby a narušenie bezpečnosti	&Nastavenie IDS/IPS ak to zariadenie podporuje	&C	&H	&CORE/EDGE COLCOREDIST
COLALL	&Cisco router configuration handbook \cite{Hucaby2010}\\
%%%%%%%%%%%%%%%%%%%%%%%%%%%%%%%%%%%%%%%%%%%%%%%%%%%%%%%%%%%%%%%%%%%%%%
%%                                                                  %%
%%  This is a LaTeX2e table fragment exported from Gnumeric.        %%
%%                                                                  %%
%%%%%%%%%%%%%%%%%%%%%%%%%%%%%%%%%%%%%%%%%%%%%%%%%%%%%%%%%%%%%%%%%%%%%%
Riadok č.	&Útok / Problém	&Mitigácia / Nastavenie	&Plane 	&Severity	&Facility layer	&Zdroj\\
1	&Nízky stav voľnej pamäte	&Nastavenie notifikácie pri dochádzaní pamäte	&M	&M	&A	&Cisco Guide to Harden Cisco IOS Devices \cite{Singh2018}
Cisco SAFE Reference Guide \cite{uYLsMtQInofenpV3}\\
2	&Logovacie správy nemôžu byť zaznamenané kvôli nedostatku pamäte	&Rezervovanie pamäte pre kritické notifikácie pri nedostatku pamäte	&M	&H	&A	&Cisco Guide to Harden Cisco IOS Devices \cite{Singh2018}
Cisco SAFE Reference Guide \cite{uYLsMtQInofenpV3}\\
3	&Vysoké zaťaženie CPU	&Nastavenie notifikácie vysokom zaťažení CPU	&M	&M	&A	&Cisco Guide to Harden Cisco IOS Devices \cite{Singh2018}
Cisco SAFE Reference Guide \cite{uYLsMtQInofenpV3}\\
4	&Vysoké zaťaženie zariadenia spôsobilo nemožnosť prihlásenia k nemu	&Rezervovanie pamäte preprotokoly na manažment zariadení pri nedostatku pamäte	&M	&H	&A	&Cisco Guide to Harden Cisco IOS Devices \cite{Singh2018}\\
%%%%%%%%%%%%%%%%%%%%%%%%%%%%%%%%%%%%%%%%%%%%%%%%%%%%%%%%%%%%%%%%%%%%%%
%%                                                                  %%
%%  This is a LaTeX2e table fragment exported from Gnumeric.        %%
%%                                                                  %%
%%%%%%%%%%%%%%%%%%%%%%%%%%%%%%%%%%%%%%%%%%%%%%%%%%%%%%%%%%%%%%%%%%%%%%
Riadok č.	&Útok / Problém	&Mitigácia / Nastavenie	&Plane 	&Severity	&Facility layer	&Zdroj\\
1	&Nemožná identifikácia zariadenia	&Vytvoriť hostname	&M	&L	&A	&CIS Cisco IOS 15 Benchmark \cite{CIS_DrTLsgXv24lxeIIM}\\
2	&Nemožnosť vzdialeného prístupu	&Vytvoriť doménové meno	&M	&L	&A	&CIS Cisco IOS 15 Benchmark \cite{CIS_DrTLsgXv24lxeIIM}\\
3	&Identifikácia pravidla v ACL	&Popis každého pravidla v ACL pre lepšiu identifikáciu	&M	&L	&A	&Cisco Guide to Harden Cisco IOS Devices \cite{Singh2018}\\
4	&Indentifikácia rozhrania	&Popis každého rozhrania	&M	&L	&A	&CCNA Routing and Switching Study Guide \cite{Lammle2013}\\
5	&Nemožnosť identifikácie účelu VLAN	&Pridanie mena k VLAN	&C	&L	&DIST
COLDISTACC
ACC
COLALL	&CCNA Routing and Switching Study Guide \cite{Lammle2013}\\
%%%%%%%%%%%%%%%%%%%%%%%%%%%%%%%%%%%%%%%%%%%%%%%%%%%%%%%%%%%%%%%%%%%%%%
%%                                                                  %%
%%  This is a LaTeX2e table fragment exported from Gnumeric.        %%
%%                                                                  %%
%%%%%%%%%%%%%%%%%%%%%%%%%%%%%%%%%%%%%%%%%%%%%%%%%%%%%%%%%%%%%%%%%%%%%%
Riadok č.	&Útok / Problém	&Mitigácia / Nastavenie	&Plane 	&Severity	&Facility layer	&Zdroj	\\
1	&Nepovolený prístup k manažovaniu zariadenia	&Vytvoriť a aplikovať ACL pre OOB, Telnet, SSH a pod. a zaznamenať v logu prístupy	&M	&C	&A	&Hardening Cisco Routers \cite{Akin2002}
CIS Cisco IOS 15 Benchmark \cite{CIS_DrTLsgXv24lxeIIM}	\\
2	&Neautorizovaný prístup cez nepoužívané a nezabezpečené protokoly na manažment zariadení	&Vypnúť nepoužívané protokoly na prístup k manažovaniu zariadení (telnet a pod.)	&M	&H	&A	&CIS Cisco IOS 15 Benchmark \cite{CIS_DrTLsgXv24lxeIIM}
Cisco Guide to Harden Cisco IOS Devices \cite{Singh2018}
	\\
3	&Nepovolený prístup k manažmentu konfigurácie zariadenia	&Vypnutie odchádzajúcich spojení pre protokoly na manažment zariadení pokiaľ sa nepoužívajú (telnet a pod.)	&M	&H	&A	&Cisco Guide to Harden Cisco IOS Devices \cite{Singh2018}	\\
4	&Prítup bez požadovaných prístupových údajov	&Nakonfigruovanie protokolov na manažment zariadení, aby požadovali prístupové údaje (telnet a pod.)	&M	&C	&A	&CIS Cisco IOS 15 Benchmark \cite{CIS_DrTLsgXv24lxeIIM}	\\
5	&Nekonzistenia konfiguračných súborov pri zmenách konfigurácie viac ako jedným administrátorom	&Povolit súčasne iba jednému administrátorovi vykonávanie zmien v konfigurácii	&M	&H	&A	&Cisco Guide to Harden Cisco IOS Devices \cite{Singh2018}	\\
6	&Nepoužívanie zabezpečeného protokolu na manažment zariadení môže viesť k odposluchu	&Zapnutie SSH	&M	&C	&A	&CIS Cisco IOS 15 Benchmark \cite{CIS_DrTLsgXv24lxeIIM}
Cisco Router Hardening \cite{Graesser2001}	\\
7	&Nebezpečná verzia 1 protokolu SSH	&SSH verzia 2	&M	&C	&A	&Cisco CCNA Security Study Guide \cite{McMillan2018}	\\
8	&Útok na krátky RSA kĺúč	&Dĺžka RSA kľúča minimálne 2048 bitov	&M	&C	&A	&CIS Cisco IOS 15 Benchmark \cite{CIS_DrTLsgXv24lxeIIM}
Transitioning the Use of Cryptographic Algorithms and Key Lengths \cite{Barker2019} 
	\\
9	&Dlhé neaktívne sedenie môže byť zneužité alebo aj fyzický prístup útočníka k aktívnemu sedeniu môže viesť k zmene konfigurácie	&SSH čas vypršania sedenia	&M	&M	&A	&CIS Cisco IOS 15 Benchmark \cite{CIS_DrTLsgXv24lxeIIM}
Cisco Router Hardening \cite{Graesser2001}	\\
10	&Hádanie hesla k RSA kľúču	&SSH maximálny počet neúspešných pokusov	&M	&H	&A	&Cisco router configuration handbook \cite{Hucaby2010}
 Cisco IOS 23 - Autentizace uživatele na switchi vůči Active Directory \cite{Bouska2009}	\\
11	&Útok hrubou silou na zistenie prihlasovacích údajov	&Špecifikovať čas po ktorý nie je možné po N pokusoch sa prihlásiť	&M	&H	&A	&Cisco router configuration handbook \cite{Hucaby2010}
 Cisco IOS 23 - Autentizace uživatele na switchi vůči Active Directory \cite{Bouska2009}	\\
12	&Prihlásenie na zariadenie nie je možné kvôli zablokovaniu pre príliš veľa neúspešných pokusov	&Povolenie prístupu administrátorovi na základe IP adresy, keď je protokol na manažovanie zariadení nedostupný kvôli DOS útoku	&M	&M	&A	&Cisco router configuration handbook \cite{Hucaby2010}
 Cisco IOS 23 - Autentizace uživatele na switchi vůči Active Directory \cite{Bouska2009}	\\
13	&Dlhé neaktívne sedenie môže byť zneužité alebo aj fyzický prístup útočníka k aktívnneum sedeniu môže viesť k zmene konfigurácie	&Čas vypršania sedenia pre protokol na manažovanie zariadení	&M	&M	&A	&CIS Cisco IOS 15 Benchmark \cite{CIS_DrTLsgXv24lxeIIM}
Cisco Router Hardening \cite{Graesser2001}
Cisco SAFE Reference Guide \cite{uYLsMtQInofenpV3}
	\\
14	&Možné prihlásenie do zariadenia cez telnet keď je prítomné SSH	&Zakázať telnet ak je SSH aktívne	&M	&C	&A	&CIS Cisco IOS 15 Benchmark \cite{CIS_DrTLsgXv24lxeIIM}
Cisco Router Hardening \cite{Graesser2001}	\\
15	&Útočník nie je informovaný o právnych následkoch	&Právne upozornenie pri prístupe k zariadeniu	&M	&L	&A	&Cisco CCNA Security Study Guide \cite{McMillan2018}
CIS Cisco IOS 15 Benchmark \cite{CIS_DrTLsgXv24lxeIIM}
Cisco Router Hardening \cite{Graesser2001}	\\
16	&Nepovolená zmena konfigurácie zariadenia	&Vytvorenie hesla na editovanie konfigurácie zariadenia	&M	&C	&A	&CIS Cisco IOS 15 Benchmark \cite{CIS_DrTLsgXv24lxeIIM}
Cisco Router Hardening \cite{Graesser2001}	\\
17	&Nepovolený prístup k manažmentu konfigurácie zariadenia	&Lokálne zabezpečené účty	&M	&C	&A	&Cisco Guide to Harden Cisco IOS Devices \cite{Singh2018}
CIS Cisco IOS 15 Benchmark \cite{CIS_DrTLsgXv24lxeIIM}	\\
18	&Centrálna správa prihlásení a dohľadateľnosť zmien v konfigurácií	&Definovanie a povolenie AAA serveru na prihlásenie a definovanie záložného prihlásenia	&M	&H	&A	&Hardening Cisco Routers \cite{Akin2002}
CIS Cisco IOS 15 Benchmark \cite{CIS_DrTLsgXv24lxeIIM} 
Cisco CCNA Security Study Guide \cite{McMillan2018}
Cisco Router Hardening \cite{Graesser2001}	\\
19	&Centrálna správa prihlásení a dohľadateľnosť zmien v konfigurácií	&Definovanie a povolenie AAA serveru na editáciu konfigurácií a definovanie záložného prihlásenia	&M	&M	&A	&CIS Cisco IOS 15 Benchmark \cite{CIS_DrTLsgXv24lxeIIM}
Cisco Router Hardening \cite{Graesser2001}	\\
20	&Hádanie prístupových údajov	&Definovanie maximálneho počtu neúspešných pokusov o prihlásenie a následné zablokovanie účtu	&M	&H	&A	&CIS Cisco IOS 15 Benchmark \cite{CIS_DrTLsgXv24lxeIIM} 
Cisco Router Hardening \cite{Graesser2001}	\\
21	&Prihlásenie bez prihlasovacích údajov	&Zakázať záložné prihlásenie bez poskynutia autentizačných prostriedkov	&M	&C	&A	&Cisco Guide to Harden Cisco IOS Devices \cite{Singh2018}	\\
22	&AAA používa primárne lokálne účty namiesto centralizovaných na serveri	&AAA nesmie používať ako prvú možnosť prihlásenia lokálny účet 	&M	&H	&A	&CIS Cisco IOS 15 Benchmark \cite{CIS_DrTLsgXv24lxeIIM}
Cisco Router Hardening \cite{Graesser2001}	\\
23	&Používateľ prihlásený do zariadenia môže spúšťať akékoľvek príkazy	&Nastavenie AAA autorizácie pre spúštanie príkazov. V prípade výpadku AAA serveru, bude užívateľ odhlásený a následne prihlásený podľa  záložného prihlásenia, aby mu nebolo pridelené vysoké oprávnenie umožňujúce vykonávať príkazy, na ktoré nemá právo	&M	&H	&A	&Cisco Router Hardening \cite{Graesser2001}
Cisco Guide to Harden Cisco IOS Devices \cite{Singh2018}	\\
24	&Administrátor vloží zlý príkaz a po čase je ho nemožné dohľadať a zjednať nápravu	&Nastavenie AAA účtovania respektíve logovania pripojení a vykonaných príkazov	&M	&H	&A	&CIS Cisco IOS 15 Benchmark \cite{CIS_DrTLsgXv24lxeIIM}	\\
25	&AAA zdrojové rozhranie nie je rovnaké pri každom reštarte	&Definovanie loopback zdrojového rozhrania pre AAA	&M	&M	&A	&CIS Cisco IOS 15 Benchmark \cite{CIS_DrTLsgXv24lxeIIM}	\\
26	&SSH zdrojové rozhranie nie je rovnaké pri každom reštarte	& Definovanie loopback zdrojového rozhrania pre SSH	&M	&M	&A	&CIS Cisco IOS 15 Benchmark \cite{CIS_DrTLsgXv24lxeIIM}	\\
27	&DOS útok na štandardný SSH port 22	&Špecifikovanie iného portu pre SSH ako štandardného alebo aplikovanie Port Knocking \cite{MJVmQiKUgZl92S8u}	&M	&H	&A	&Port Knocking \cite{MJVmQiKUgZl92S8u}	\\
%%%%%%%%%%%%%%%%%%%%%%%%%%%%%%%%%%%%%%%%%%%%%%%%%%%%%%%%%%%%%%%%%%%%%%
%%                                                                  %%
%%  This is a LaTeX2e table fragment exported from Gnumeric.        %%
%%                                                                  %%
%%%%%%%%%%%%%%%%%%%%%%%%%%%%%%%%%%%%%%%%%%%%%%%%%%%%%%%%%%%%%%%%%%%%%%
Riadok č.	&Útok / Problém	&Mitigácia / Nastavenie	&Plane 	&Severity	&Facility layer	&Zdroj\\
1	&Skenovanie a zistenie informácií o sieti za pomoci protokolu CDP a využitie bezpečnostných chýb	&Zakázanie protokolu CDP	&M	&C	&A	&CIS Cisco IOS 15 Benchmark \cite{CIS_DrTLsgXv24lxeIIM}
Cisco Router Hardening \cite{Graesser2001}\\
2	&Skenovanie a zistenie informácií o sieti za pomoci protokolu LLDP a využitie bezpečnostných chýb	&Zakázanie protokolu LLDP	&M	&C	&A	&Cisco CCNA Security Study Guide \cite{McMillan2018}\\
3	&Proxy ARP môže viesť k obídeniu PVLAN a rozširuje broadcast doménu	&Vypnutie Proxy ARP	&C	&C	&CORE/EDGE
DIST
COLCOREDIST
COLDISTACC
COLALL	&CIS Cisco IOS 15 Benchmark \cite{CIS_DrTLsgXv24lxeIIM}
Cisco Router Hardening \cite{Graesser2001}\\
4	&Útočník môže zistiť, že IP adresa, na ktorú skušal ping je nesprávna	&Vypnutie spáv ICMP Unreachable	&D	&H	&CORE/EDGE
DIST
COLCOREDIST
COLDISTACC
COLALL	&Cisco Guide to Harden Cisco IOS Devices \cite{Singh2018}
\\
5	&Útočník môže zistiť masku podsiete pomocou ICMP Mask reply	&Vypnutie spáv ICMP Mask reply	&D	&H	&CORE/EDGE
DIST
COLCOREDIST
COLDISTACC
COLALL	&Hardening Cisco Routers \cite{Akin2002}
\\
6	&Umožňuje DOS Smurf útok, mapovanie siete pomocou ping na broadcast adresu vzdialenej siete	&Vypnutie ICMP echo správ na broadcast adresu, vypnutie directed broadcasts	&D	&C	&CORE/EDGE
DIST
COLCOREDIST
COLDISTACC
COLALL	&Cisco Router Hardening \cite{Graesser2001}
Hardening Cisco Routers \cite{Akin2002}
\\
7	&Útočník môže zistiť smerovacie informácie alebo vyťažiť CPU	&Vypnutie spáv ICMP Redirects	&D	&H	&CORE/EDGE
DIST
COLCOREDIST
COLDISTACC
COLALL	&Cisco Guide to Harden Cisco IOS Devices \cite{Singh2018}
\\
8	&Mapovanie sete pomocou pingu na multicast adresu všetkých uzlov a MLD/IGMP Query Overload a Smurf útok	& ACL blokujúce ICMP echo request na multicast adresu všetkých uzlov a MLD/IGMP Query na prístupových portoch	&C	&M	&DIST
COLDISTACC
ACC	&Bezpečné IPv6: trable s multicastem \cite{Podermanski532015}
MLD Considered Harmful \cite{Rey2016}
\\
%%%%%%%%%%%%%%%%%%%%%%%%%%%%%%%%%%%%%%%%%%%%%%%%%%%%%%%%%%%%%%%%%%%%%%
%%                                                                  %%
%%  This is a LaTeX2e table fragment exported from Gnumeric.        %%
%%                                                                  %%
%%%%%%%%%%%%%%%%%%%%%%%%%%%%%%%%%%%%%%%%%%%%%%%%%%%%%%%%%%%%%%%%%%%%%%
Riadok č.	&Útok / Problém	&Mitigácia / Nastavenie	&Plane 	&Severity	&Facility layer	&Zdroj\\
1	&Nekonzistencia časov v logoch a problém pričlenenia logov k relevantným incidentom	&Nastavenie NTP serveru pre aktuálny čas v logoch	&M	&H	&A	&Network Security Auditing \cite{Jackson2010}
Cisco Guide to Harden Cisco IOS Devices \cite{Singh2018}
CIS Cisco IOS 15 Benchmark \cite{CIS_DrTLsgXv24lxeIIM}\\
2	&Pripojenie servera s rovnakou IP adresou, ale falošným časom	&Nastavenie NTP autentizácie	&M	&H	&A	&Network Security Auditing \cite{Jackson2010}
Cisco Guide to Harden Cisco IOS Devices \cite{Singh2018}
CIS Cisco IOS 15 Benchmark \cite{CIS_DrTLsgXv24lxeIIM}\\
3	&NTP zdrojové rozhranie nie je rovnaké pri každom reštarte	& Definovanie loopback zdrojového rozhrania pre NTP	&M	&M	&A	&Network Security Auditing \cite{Jackson2010}
Cisco Guide to Harden Cisco IOS Devices \cite{Singh2018}
CIS Cisco IOS 15 Benchmark \cite{CIS_DrTLsgXv24lxeIIM}\\
4	&Väčšia bezpečnosť (pub/priv key) NTP a podpora IPv6	&Použitie NTP verzie 4	&M	&M	&A	&The NTP FAQ and HOWTO \cite{s0goWNnWp5OjqREE}\\
5	&Falošný čas od podvrhnutého NTP zdroja	&Nastavenie NTP peer s inými sieťovými zariadeniami na krížovú validáciu času a záložný zdroj času	&M	&M	&A	&Hardening Cisco Routers \cite{Akin2002}\\
%%%%%%%%%%%%%%%%%%%%%%%%%%%%%%%%%%%%%%%%%%%%%%%%%%%%%%%%%%%%%%%%%%%%%%
%%                                                                  %%
%%  This is a LaTeX2e table fragment exported from Gnumeric.        %%
%%                                                                  %%
%%%%%%%%%%%%%%%%%%%%%%%%%%%%%%%%%%%%%%%%%%%%%%%%%%%%%%%%%%%%%%%%%%%%%%
Riadok č.	&Útok / Problém	&Mitigácia / Nastavenie	&Plane 	&Severity	&Facility layer	&Zdroj\\
1	&Zlyhanie zariadenia alebo linky môže viest k nefunkčnosti siete 	&Povolenie FHRP s autentizáciou a aktuálnou verziou	&C	&M	&CORE/EDGE
COLCOREDIST
COLALL	&CCNA Routing and Switching Study Guide \cite{Lammle2013}\\
2	&Nepoužívané, staré a nezabezpečené služby môžu byť použité na škodlivé účely	&Vypnutie nepoužívaných služieb z bezpečnostných dôvodov a na šetrenie CPU a pamäte 	&VARY – SEE SPECIFIC VENDOR CHECKLIST	&H	&Záleží na výrobcovi, operačnom systéme a verzii	&Cisco IOS features that you should disable or restrict\\
3	&Odpočúvanie komunikácie  cez nezabezpečené tunely	&Vypnúť tunely ktoré nie sú zabezpečené alebo zabezpečiť tunely	&D	&C	&CORE/EDGE
DIST
COLCOREDIST
COLDISTACC
COLALL	&CIS Cisco IOS 15 Benchmark \cite{CIS_DrTLsgXv24lxeIIM}
Cisco router configuration handbook \cite{Hucaby2010}\\
4	&Môže byť zneužité odpočúvanie pokiaľ sa používa monitorovanie prevádzky a monitorovanie prevádzky kvôli legislatívnym potrebám	&Monitorovanie výkonnosti siete a zber sieťového prenosu kvôli legislatívnym potrebám	&C	&N	&A	&Cisco Guide to Harden Cisco IOS Devices \cite{Singh2018}\\
5	&Útočník s fyzickým prístupom k zariadeniu alebo portu môže odpočúvať alebo posielať škodlivý obsah	&Explicitne zakázať nepoužívané porty	&D	&C	&A	&Cisco Router Hardening \cite{Graesser2001}
Network Security Auditing \cite{Jackson2010}\\
6	&Zdrojové rozhranie pre management a control protokoly	&Vytvorť Loopback rozhranie s IP adresou	&M / C	&M	&A	&
Cisco Guide to Harden Cisco IOS Devices \cite{Singh2018}
CIS Cisco IOS 15 Benchmark \cite{CIS_DrTLsgXv24lxeIIM}\\
7	&Pretečenie pamäte	&Povoliť mechanizmy na detekciu pretečenia pamäte	&M	&M	&A	&Cisco Guide to Harden Cisco IOS Devices \cite{Singh2018}\\
8	&Načítanie škodlivej konfigurácie zo siete počas bootovania	&Vypnutie načítania operačného systému alebo konfigurácie zo siete pokiaľ to nie je nutné	&M	&M	&A	&Hardening Cisco Routers \cite{Akin2002}\\
9	&Odpočuvanie konfigurácií zariadení pri zálohe	&Zapnutie zabezpečenej zálohy na server (SFTP, SCP)	&M	&H	&A	&Cisco Guide to Harden Cisco IOS Devices \cite{Singh2018}\\
10	&Vymazanie konfigurácie	&Zapnutie ochrany pred výmazom konfigurácie	&M	&H	&A	&Cisco CCNA Security Study Guide \cite{McMillan2018}\\
11	&Možnosť urobiť diff zmien konfigurácií a jej návrat	&Periodické zálohovanie konfigurácie a logovanie jej zmien	&M	&M	&A	&Cisco CCNA Security Study Guide \cite{McMillan2018}
Cisco Guide to Harden Cisco IOS Devices \cite{Singh2018}\\
12	&Nemožnosť prihlásenia pri zaseknutom TCP spojení	&Terminovanie zaseknutého TCP spojenia	&M	&M	&A	&Cisco Guide to Harden Cisco IOS Devices \cite{Singh2018}\\
13	&Možnosť prečítať heslá z uniknutých konfigurácií	&Zašifrovanie hesiel v otvorenej podobe	&M	&C	&A	&CIS Cisco IOS 15 Benchmark \cite{CIS_DrTLsgXv24lxeIIM}\\
%%%%%%%%%%%%%%%%%%%%%%%%%%%%%%%%%%%%%%%%%%%%%%%%%%%%%%%%%%%%%%%%%%%%%%
%%                                                                  %%
%%  This is a LaTeX2e table fragment exported from Gnumeric.        %%
%%                                                                  %%
%%%%%%%%%%%%%%%%%%%%%%%%%%%%%%%%%%%%%%%%%%%%%%%%%%%%%%%%%%%%%%%%%%%%%%
Riadok č.	&Útok / Problém	&Mitigácia / Nastavenie	&Plane 	&Severity	&Facility layer	&Zdroj\\
1	&Vloženie a manipulácia so smerovacími informáciami	&Autentizácia smerovacích protokolov (nie heslá v otvorenej podobe)	&C	&H	&CORE/EDGE
DIST
COLCOREDIST
COLDISTACC
COLALL	&Cisco CCNA Security Study Guide \cite{McMillan2018}
Cisco Guide to Harden Cisco IOS Devices \cite{Singh2018}
CIS Cisco IOS 15 Benchmark \cite{CIS_DrTLsgXv24lxeIIM}\\
2	&OSPF virtuálne linky degradujú výkon	&Vypnutie virtuálnych liniek pre OSPF	&C	&H	&CORE/EDGE
DIST
COLCOREDIST
COLDISTACC
COLALL	&Designing Cisco network service architectures \cite{Tiso2012}\\
3	&Koncové zariadenie, užívateľ a útočník môžu vidiet smerovacie správy a topológiu siete alebo pripojenie škodlivého zariadenia, ktoré vysielať a prijímať smerovacie správy	&Špecifikovanie rozhraní, ktoré nebudú prijímať routovacie informácie	&C	&H	&CORE/EDGE
DIST
COLCOREDIST
COLDISTACC
COLALL	&OSPF Security: Attacks and Defenses \cite{Khandelwal2016}\\
4	&Nemožnosť sprevádzkovať procesy smerovacích protokolov v určitých prípadoch pri použití IPv6	&Špecifikovanie identifikátorov smerovacích protokolov pre každý router (router ID)	&C	&M	&CORE/EDGE
DIST
COLCOREDIST
COLDISTACC
COLALL	&Protocol-Independent Routing Properties Feature Guide \cite{q7WZuvqA1fZEsYyL}
CCNA Routing and Switching Study Guide \cite{Lammle2013}\\
5	&Vysledovateľnosť nefunkčnosti routovacieho protokolu a nesprávneho nastavenia	&Zaznamenie zmeny v logu pri zmenách v smerovaní	&C	&M	&CORE/EDGE
DIST
COLCOREDIST
COLDISTACC
COLALL	&Cisco SAFE Reference Guide \cite{uYLsMtQInofenpV3}
\\
6	& Škodlivé vloženie smerovacích informácií informácií, vzdialený útok	&TTL security	&C	&H	&CORE/EDGE
DIST
COLCOREDIST
COLDISTACC
COLALL	&OSPF Security: Attacks and Defenses \cite{Khandelwal2016}
Cisco Guide to Harden Cisco IOS Devices \cite{Singh2018}\\
7	&Nesprávne smerovanie kvôli sumarizácií	&Vypnutie automatickej sumarizácie smerovacích protokolov	&C	&H	&CORE/EDGE
DIST
COLCOREDIST
COLDISTACC
COLALL	&CCNA Routing and Switching Study Guide \cite{Lammle2013}\\
8	&DOS útok na stanicu, cez ktorú bola špecifikovaná cesta a teda nemožnosť komunikácie s koncovým bodom. Alebo zosnovanie MITM útoku	&Vypnutie IP source routing	&C	&C	&CORE/EDGE
DIST
COLCOREDIST
COLDISTACC
COLALL	&CIS Cisco IOS 15 Benchmark \cite{CIS_DrTLsgXv24lxeIIM}\\
9	&DOS útok pomocou podvrhnutej IP adresy alebo vzdialený útok na smerovací protokol	&Zapnutie reverse path forwarding strict/loose mode	&C	&H	&CORE/EDGE
DIST
COLCOREDIST
COLDISTACC
COLALL	&OSPF Security: Attacks and Defenses \cite{Khandelwal2016}
Network Security Auditing \cite{Jackson2010}
CIS Cisco IOS 15 Benchmark \cite{CIS_DrTLsgXv24lxeIIM}\\
%%%%%%%%%%%%%%%%%%%%%%%%%%%%%%%%%%%%%%%%%%%%%%%%%%%%%%%%%%%%%%%%%%%%%%
%%                                                                  %%
%%  This is a LaTeX2e table fragment exported from Gnumeric.        %%
%%                                                                  %%
%%%%%%%%%%%%%%%%%%%%%%%%%%%%%%%%%%%%%%%%%%%%%%%%%%%%%%%%%%%%%%%%%%%%%%
Riadok č.	&Útok / Problém	&Mitigácia / Nastavenie	&Plane 	&Severity	&Facility layer	&Zdroj\\
1	&Odpočúvanie SNMP verzie 1 a 2c	&Použitie SNMP verie 3 pokiaľ je SNMP používané	&M	&C	&A	&CIS Cisco IOS 15 Benchmark \cite{CIS_DrTLsgXv24lxeIIM}
Cisco Router Hardening \cite{Graesser2001}\\
2	&Modifikovanie konfigurácie pomocou SNMP	&Obmedzenie SNMP iba na čítanie	&M	&C	&A	&CIS Cisco IOS 15 Benchmark \cite{CIS_DrTLsgXv24lxeIIM}
Cisco Router Hardening \cite{Graesser2001}
Hardening Cisco Routers \cite{Akin2002}\\
3	&Neoprávnený prístup k SNMP informáciám	&Obmedzenie SNMP iba pre vybrané IP adresy	&M	&H	&A	&CIS Cisco IOS 15 Benchmark \cite{CIS_DrTLsgXv24lxeIIM}
Cisco Router Hardening \cite{Graesser2001}\\
4	&Administrátor nemá povedomie o problémoch na zariadení	&Povolenie asynchrónnych správ SNMP TRAP	&M	&M	&A	&CIS Cisco IOS 15 Benchmark \cite{CIS_DrTLsgXv24lxeIIM}
Cisco Router Hardening \cite{Graesser2001}
Hardening Cisco Routers \cite{Akin2002}\\
5	&Odpočúvanie SNMP sedenie z dôvodu slabého šifrovania a hashovacej  funkcie	&Vytvorenie SNMP verzie 3 užívateľa s minimálnym šifrovaním AES 128 bit a hashovacou funkciou SHA	&M	&C	&A	&Transitioning the Use of Cryptographic Algorithms and Key Lengths \cite{Barker2019}
CIS Cisco IOS 15 Benchmark \cite{CIS_DrTLsgXv24lxeIIM}
Hardening Cisco Routers \cite{Akin2002}\\
6	&Hard identification of SNMP messages from many IPs/ Sťažená identifikácia SNMP správ z rôznych IP	&Definovanie lokácie SNMP serveru	&M	&L	&A	&Cisco IOS Cookbook \cite{Dooley2007}\\
7	&SNMP zdrojové rozhranie nie je rovnaké pri každom reštarte	& Definovanie loopback zdrojového rozhrania pre SNMP	&M	&M	&A	&CIS Cisco IOS 15 Benchmark \cite{CIS_DrTLsgXv24lxeIIM}\\
8	&Zmeny názvov rozhraní medzi reštartami a nemožnosť monitorovania pomocou SNMP	&SNMP statické nemenné meno rozhrania aj po reštarte zariadenia	&M	&H	&A	&Cisco IOS Cookbook \cite{Dooley2007}\\
%%%%%%%%%%%%%%%%%%%%%%%%%%%%%%%%%%%%%%%%%%%%%%%%%%%%%%%%%%%%%%%%%%%%%%
%%                                                                  %%
%%  This is a LaTeX2e table fragment exported from Gnumeric.        %%
%%                                                                  %%
%%%%%%%%%%%%%%%%%%%%%%%%%%%%%%%%%%%%%%%%%%%%%%%%%%%%%%%%%%%%%%%%%%%%%%
Riadok č.	&Útok / Problém	&Mitigácia / Nastavenie	&Plane 	&Severity	&Facility layer	&Zdroj\\
1	&Rogue root bridge protection (root guard)	&Rogue root bridge 	&C	&C	&DIST
COLDISTACC
ACC	&Lan Switch Security \cite{Vyncke2008}\\
2	&BPDU protection (BPDU guard)	&Pripojenie pripínaču na koncový prístupový port	&C	&C	&DIST
COLDISTACC
ACC	&Lan Switch Security \cite{Vyncke2008}\\
3	&Prístupové porty by sa nemali podielať na STP procese	&Rýchlosť konvergencie	&C	&H	&DIST
COLDISTACC
ACC	&Lan Switch Security \cite{Vyncke2008}\\
4	&Špeciálne konfigurácie zaisťujúce bezslučkovú topológiu pomocou STP keď nastane jednosmerná komunikácia (Loop Guard)	&Jednosmerná komunikácia medzi prepínačmi môźe viesť k topoógií so slučkami	&C	&C	&DIST
COLDISTACC
ACC	&Designing Cisco network service architectures \cite{Tiso2012}\\
%%%%%%%%%%%%%%%%%%%%%%%%%%%%%%%%%%%%%%%%%%%%%%%%%%%%%%%%%%%%%%%%%%%%%%
%%                                                                  %%
%%  This is a LaTeX2e table fragment exported from Gnumeric.        %%
%%                                                                  %%
%%%%%%%%%%%%%%%%%%%%%%%%%%%%%%%%%%%%%%%%%%%%%%%%%%%%%%%%%%%%%%%%%%%%%%
Riadok č.	&Útok / Problém	&Mitigácia / Nastavenie	&Plane 	&Severity	&Facility layer	&Zdroj\\
1	&Administrátor nemá povedomie o problémoch na zariadení	&Povolenie logovania protokolom SYSLOG a špecifikovanie IP adresy SYSLOG serveru	&M	&H	&A	&CIS Cisco IOS 15 Benchmark \cite{CIS_DrTLsgXv24lxeIIM}
Cisco Router Hardening \cite{Graesser2001}\\
2	&Neprijímanie všetkých dôležitých incidentov na zariadení z protokolu SYSLOG	&Špecifikovanie dôležitosti oznámenií SYSLOG na INFORMATIONAL	&M	&M	&A	&CIS Cisco IOS 15 Benchmark \cite{CIS_DrTLsgXv24lxeIIM}\\
3	&SYSLOG zdrojové rozhranie nie je rovnaké pri každom reštarte	& Definovanie loopback zdrojového rozhrania pre SYSLOG	&M	&M	&A	&Cisco Guide to Harden Cisco IOS Devices \cite{Singh2018}
CIS Cisco IOS 15 Benchmark \cite{CIS_DrTLsgXv24lxeIIM}\\
4	&Insufficient and non-standard  time format in logging messages/ Nedostatočné a neštandardné formáty času pri logovacích správach	&Definovanie formátu času pre logovacie a ladiace výstupy	&M	&M	&A	&CIS Cisco IOS 15 Benchmark \cite{CIS_DrTLsgXv24lxeIIM}
Cisco Router Hardening \cite{Graesser2001}\\
5	&Administrátor nevidí dôležité incidenty pri prihlásení a konfigurovaní cez konzolu	&Vypisovanie SYSLOG správ CRITICAL a dôležitejších do terminálu	&M	&M	&A	&Cisco Guide to Harden Cisco IOS Devices \cite{Singh2018}
CIS Cisco IOS 15 Benchmark \cite{CIS_DrTLsgXv24lxeIIM}\\
6	&Malá vyrovnávacia pamäť pre SYSLOG je dôvodom zahadzovanie správ	&Definovanie veľkosti SYSLOG buffera dôležitosti oznámení na INFORMATIONAL	&M	&H	&A	&Cisco Guide to Harden Cisco IOS Devices \cite{Singh2018}\\
7	&Neprístupný SYSLOG server spôsobuje zahadzovanie dôležitých syslog správ	&Definovanie dočasného úložiska SYSLOG správ v prípade nedostupnosti servera	&M	&H	&A	&Cisco Guide to Harden Cisco IOS Devices \cite{Singh2018}\\
8	&Problém identifikácie SYSLOG správ s rovnakou časovou značkou	&Pridanie sekvenčného čísla ku každej syslog správe	&M	&L	&A	&Hardening Cisco Routers \cite{Akin2002}\\
%%%%%%%%%%%%%%%%%%%%%%%%%%%%%%%%%%%%%%%%%%%%%%%%%%%%%%%%%%%%%%%%%%%%%%
%%                                                                  %%
%%  This is a LaTeX2e table fragment exported from Gnumeric.        %%
%%                                                                  %%
%%%%%%%%%%%%%%%%%%%%%%%%%%%%%%%%%%%%%%%%%%%%%%%%%%%%%%%%%%%%%%%%%%%%%%
Riadok č.	&Útok / Problém	&Mitigácia / Nastavenie	&Plane 	&Severity	&Facility layer	&Zdroj\\
1	&Špeciálna VLAN pre manažment na obmedzenie prístupu iba pre administrátorov	&Vytvorenie separátnej VLAN pre manažment	&C	&M	&DIST, COLDISTACC, ACC	&CCNA Routing and Switching Study Guide \cite{Lammle2013}\\
2	&Útočníkovi s fyzickým prístupom k portu môže byť pridelený prístup do časti siete, ktorá zodpovedá príslušnej VLAN 	&Vytvorenie špeciálnej black hole VLAN pre nevyužité porty	&C	&C	&DIST, COLDISTACC, ACC	&Cisco SAFE Reference Guide \cite{uYLsMtQInofenpV3}\\
3	&Predvolenej VLAN je povolené prepnute na akýkoľvek port, VLAN hopping, double tagging	&Odobrať všetky porty z predvolenej VLAN	&C	&C	&DIST, COLDISTACC, ACC	&Cisco SAFE Reference Guide \cite{uYLsMtQInofenpV3}\\
4	&Predvolenej VLAN je povolené byť prepnutá na akýkoľvek port, VLAN hopping, double tagging	&Vytvorenie natívnej VLAN rozdielnej ako predvolená, priradeni k trunk portu a povolenie iba potrebných portov	&C	&C	&DIST, COLDISTACC, ACC	&Cisco SAFE Reference Guide \cite{uYLsMtQInofenpV3}\\
5	&DTP útok, Switch spoofing útok	&Vypnutie dynamického trunkovacieho protokolu a explicitne určiť porty ako prístupové a trunk	&C	&C	&DIST, COLDISTACC, ACC	&Cisco SAFE Reference Guide \cite{uYLsMtQInofenpV3}\\
6	&Nový prepínač s vyšším číslom revízie, ale s nesprávnou VLAN databázou môže šíriť falošné VLAN identifikátory a spôsobiť nefunkčnosť siete, veľa možnćh VTP útokov kvǒli zraniteľnostiam 	&Vypnutie MVRP. MRP, GARP, VTP po úspešnej propagácií VLAN	&C	&C	&DIST
COLDISTACC
ACC	&Lan Switch Security \cite{Vyncke2008}\\
7	&VTP musí byť používané	&Uprednostniť VTP verzie 3, špecifikovať skryté heslo a zapnúť VTP prunning pokiaľ musí byť VTP zapnuté	&C	&C	&DIST
COLDISTACC
ACC	&Lan Switch Security \cite{Vyncke2008}\\
8	&Vysoké zaťaženie linky	&Poslanie notifikácie pri prekročení prahovej hodnoty zaťaženia linky	&C	&M	&A	&Cisco SAFE Reference Guide \cite{uYLsMtQInofenpV3}\\
