%%%%%%%%%%%%%%%%%%%%%%%%%%%%%%%%%%%%%%%%%%%%%%%%%%%%%%%%%%%%%%%%%%%%%%
%%                                                                  %%
%%  This is a LaTeX2e table fragment exported from Gnumeric.        %%
%%                                                                  %%
%%%%%%%%%%%%%%%%%%%%%%%%%%%%%%%%%%%%%%%%%%%%%%%%%%%%%%%%%%%%%%%%%%%%%%
Útok / Problém	&Mitigácia / Nastavenie	&Príkazy\\
Nemožná identifikácia zariadenia	&Vytvoriť hostname	&hostname $<$hostname$>$\\
Nemožnosť vzdialeného prístupu	&Vytvoriť doménové meno	&ip domain-name $<$domain$>$\\
Nepovolený prístup k manažovaniu zariadenia	&Vytvoriť a aplikovať ACL pre OOB, Telnet, SSH a pod. a zaznamenať v logu prístupy	&ip access-list standard ACL\_NAME
 remark permit specifi ip and log
 permit $<$ip address$>$ $<$mask$>$ log-input
 remark deny other and log
 deny any log-input

ipv6 access-list ACL\_NAME
 remark permit specifi ip and log
 permit $<$ipv6 address$>$/$<$prefix$>$ any log-input
 remark deny other and log
 deny any any log-input

alebo v global config 

login on-failure log-input
login on-failure trap
login on-failure
login on-success log-input
login on-success trap
login on-success\\
Nepovolený prístup k manažovaniu zariadenia	&Vytvoriť a aplikovať ACL pre OOB, Telnet, SSH a pod. a zaznamenať v logu prístupy	&line vty $<$num$>$ $<$num$>$
 ip access-class $<$acl name$>$ in
 ipv6 access-class $<$acl name$>$ in

line tty $<$num$>$ $<$num$>$
 ip access-class $<$acl name$>$ in
 ipv6 access-class $<$acl name$>$ in

line con $<$num$>$
 ip access-class $<$acl name$>$ in
 ipv6 access-class $<$acl name$>$ in

line aux $<$num$>$
 ip access-class $<$acl name$>$ in
 ipv6 access-class $<$acl name$>$ in\\
Neautorizovaný prístup cez nepoužívané a nezabezpečené protokoly na manažment zariadení	&Vypnúť nepoužívané protokoly na prístup k manažovaniu zariadení (telnet a pod.)	&line aux 0
 no exec
 transport input none\\
Prístup bez požadovaných prístupových údajov	&Nakonfigurovanie protokolov na manažment zariadení, aby požadovali prístupové údaje (telnet a pod.)	&line vty $<$num$>$ $<$num$>$
 password
 login| login local

line tty $<$num$>$ $<$num$>$
 password
 login| login local

line con $<$num$>$
 password
 login| login local

line aux $<$num$>$
 password
 login| login local\\
Nepoužívanie zabezpečeného protokolu na manažment zariadení môže viesť k odposluchu	&Zapnutie SSH	&line vty $<$num$>$ $<$num$>$
  login local
  transport input ssh\\
Nebezpečná verzia 1 protokolu SSH	&SSH verzia 2	&ip ssh version 2\\
Dlhé neaktívne sedenie môže byť zneužité alebo aj fyzický prístup útočníka k aktívnemu sedeniu môže viesť k zmene konfigurácie	&SSH čas vypršania sedenia	&ip ssh timeout $<$timeout seconds$>$\\
Útok na krátky RSA kľúč	&Dĺžka RSA kľúča minimálne 2048 bitov	&crypto key generate rsa modulus 2048\\
Hádanie hesla k RSA kľúču	&SSH maximálny počet neúspešných pokusov	&ip ssh authentication-retries $<$max num$>$\\
Útok hrubou silou na zistenie prihlasovacích údajov	&Špecifikovať čas po ktorý nie je možné po N pokusoch sa prihlásiť	&login block-for 60 attempts 3 within 30\\
Prihlásenie na zariadenie nie je možné kvôli zablokovaniu pre príliš veľa neúspešných pokusov	&Povolenie prístupu administrátorovi na základe IP adresy, keď je protokol na manažovanie zariadení nedostupný kvôli DOS útoku	&login quiet-mode access-class $<$acl name$>$\\
Možné prihlásenie do zariadenia cez telnet keď je prítomné SSH	&Zakázať telnet ak je SSH aktívne	&line vty $<$num$>$ $<$num$>$
 no transport input all
 no transport input telnet

line tty $<$num$>$ $<$num$>$
 no transport input all
 no transport input telnet

line con $<$num$>$
 no transport input all
 no transport input telnet

line aux $<$num$>$
 no transport input all
 no transport input telnet\\
Útočník nie je informovaný o právnych následkoch	&Právne upozornenie pri prístupe k zariadeniu	&banner motd
banner login
banner exec\\
Dlhé neaktívne sedenie môže byť zneužité alebo aj fyzický prístup útočníka k aktívnemu sedeniu môže viesť k zmene konfigurácie	&Čas vypršania sedenia pre protokol na manažovanie zariadení	&line vty $<$num$>$ $<$num$>$
 exec-timeout 5

line tty $<$num$>$ $<$num$>$
 exec-timeout 5

line con $<$num$>$
 exec-timeout 5

line aux $<$num$>$
 exec-timeout 5\\
Možnosť prečítať heslá z uniknutých konfigurácií	&Zašifrovanie hesiel v otvorenej podobe	&service password-encryption\\
Nepovolená zmena konfigurácie zariadenia	&Vytvorenie hesla na editovanie konfigurácie zariadenia	&enable secret $<$secret password$>$\\
Nepovolená zmena konfigurácie zariadenia	&Vytvorenie hesla na editovanie konfigurácie zariadenia	&no enable password $<$password$>$\\
Nepovolený prístup k manažmentu konfigurácie zariadenia	&Lokálne zabezpečené účty	&username secret $<$username$>$ $<$secret password$>$\\
Nepovolený prístup k manažmentu konfigurácie zariadenia	&Lokálne zabezpečené účty	&no username password  $<$username$>$ $<$password$>$\\
Centrálna správa prihlásení a dohľadateľnosť zmien v konfigurácií	&Definovanie a povolenie AAA serveru na prihlásenie a definovanie záložného prihlásenia	&aaa new-model
radius server $<$radius server name$>$
  address ipv4 $<$ip adddress$>$ / address ipv6 $<$ipv6 adddress$>$
  key $<$password$>$

alebo

radius-server host $<$ip adddress$>$
radius-server key $<$password$>$

aaa group server radius $<$radius group$>$
  server name $<$radius server name$>$
aaa authentication login default / $<$radius login$>$ group $<$radius group$>$ local enable

line tty $<$num$>$ $<$num$>$
  login authentication default / $<$radius login$>$
line vty $<$num$>$ $<$num$>$
  login authentication default / $<$radius login$>$
line con $<$num$>$
  login authentication default / $<$radius login$>$
line aux $<$num$>$
  login authentication default / $<$radius login$>$\\
Centrálna správa prihlásení a dohľadateľnosť zmien v konfigurácií	&Definovanie a povolenie AAA serveru na prihlásenie a definovanie záložného prihlásenia	&aaa new-model
tacacs server $<$tacacs server name$>$
  address ipv4 $<$ip adddress$>$ / address ipv6 $<$ipv6 adddress$>$
  key $<$password$>$

alebo 

tacacs-server host $<$ip adddress$>$
tacacs-server key $<$password$>$

aaa group server tacacs $<$tacacs group$>$
  server name $<$tacacs server name$>$
aaa authentication login default / $<$tacacs login$>$ group $<$tacacs group$>$ local enable

line tty $<$num$>$ $<$num$>$
  login authentication default / $<$tacacs login$>$
line vty $<$num$>$ $<$num$>$
  login authentication default / $<$tacacs login$>$
line con $<$num$>$
  login authentication default / $<$tacacs login$>$
line aux $<$num$>$
  login authentication default / $<$tacacs login$>$\\
Centrálna správa prihlásení a dohľadateľnosť zmien v konfigurácií	&Definovanie a povolenie AAA serveru na editáciu konfigurácií a definovanie záložného prihlásenia	&aaa authentication enable default group $<$radius group$>$ enable
\\
Centrálna správa prihlásení a dohľadateľnosť zmien v konfigurácií	&Definovanie a povolenie AAA serveru na editáciu konfigurácií a definovanie záložného prihlásenia	&no aaa authentication enable default enable\\
Hádanie prístupových údajov	&Definovanie maximálneho počtu neúspešných pokusov o prihlásenie a následné zablokovanie účtu	&aaa authentication attempts login 3\\
Prihlásenie bez prihlasovacích údajov	&Zakázať záložné prihlásenie bez poskytnutia autentizačných prostriedkov	&vyhnúť sa aaa authentication login.*none.*\\
AAA používa primárne lokálne účty namiesto centralizovaných na serveri	&AAA nesmie používať ako prvú možnosť prihlásenia lokálny účet 	&vyhnúť sa authentication login default local\\
Používateľ prihlásený do zariadenia môže spúšťať akékoľvek príkazy	&Nastavenie AAA autorizácie pre spúšťanie príkazov. V prípade výpadku AAA serveru, bude užívateľ odhlásený a následne prihlásený podľa  záložného prihlásenia, aby mu nebolo pridelené vysoké oprávnenie umožňujúce vykonávať príkazy, na ktoré nemá právo	&aaa authorization exec $<$radius login$>$ group $<$radius group$>$ local if-authenticated\\
Používateľ prihlásený do zariadenia môže spúšťať akékoľvek príkazy	&Nastavenie AAA autorizácie pre spúšťanie príkazov. V prípade výpadku AAA serveru, bude užívateľ odhlásený a následne prihlásený podľa  záložného prihlásenia, aby mu nebolo pridelené vysoké oprávnenie umožňujúce vykonávať príkazy, na ktoré nemá právo	&aaa authorization commands 15 $<$radius login$>$ group $<$radius group$>$ local if-authenticated \\
Administrátor vloží zlý príkaz a po čase je ho nemožné dohľadať a zjednať nápravu	&Nastavenie AAA účtovania respektíve logovania pripojení a vykonaných príkazov	&aaa accounting connection
aaa accounting commands
aaa accounting exec
\\
Odpočúvanie SNMP verzie 1 a 2c	&Použitie SNMP verzie 3 pokiaľ je SNMP používané	&no snmp-server community
no snmp-server host  version 1/2c
snmp-server group $<$group name$>$ v3 priv \\
AAA zdrojové rozhranie nie je rovnaké pri každom reštarte	&Definovanie loopback zdrojového rozhrania pre AAA	&ip radius source interface loopback $<$id$>$
ip tacacs source interface loopback $<$id$>$\\
Modifikovanie konfigurácie pomocou SNMP	&Obmedzenie SNMP iba na čítanie	&snmp-server view $<$view name$>$ iso included
snmp-server group $<$group name$>$ v3 priv read $<$view name$>$\\
Neoprávnený prístup k SNMP informáciám	&Obmedzenie SNMP iba pre vybrané IP adresy	&ip access-list standard $<$acl name$>$
 remark permit only this IP 
 permit $<$ip address$>$ $<$wildcard mask$>$
 deny any log-input
ipv6 access-list $<$acl name$>$
 remark permit only this IP 
 permit $<$ipv6 address$>$/$<$prefix$>$ any
 remark deny other
 deny any any log-input
snmp-server group $<$group name$>$ v3 priv read $<$view name$>$  access $<$acl name$>$\\
Administrátor nemá povedomie o problémoch na zariadení	&Povolenie asynchrónnych správ SNMP TRAP	&snmp-server host $<$ip adddress$>$ traps version 3 priv $<$user$>$
snmp-server host $<$ip adddress$>$ version 3 priv $<$user$>$\\
Odpočúvanie SNMP sedenie z dôvodu slabého šifrovania a hashovacej  funkcie	&Vytvorenie SNMP verzie 3 užívateľa s minimálnym šifrovaním AES 128 bit a hashovacou funkciou SHA	&snmp-server user $<$user$>$ $<$group name$>$ v3 auth sha $<$password$>$ pri aes 128 $<$password$>$\\
 Sťažená identifikácia SNMP správ z rôznych IP	&Definovanie lokácie SNMP serveru	&snmp-server location $<$location$>$\\
SNMP zdrojové rozhranie nie je rovnaké pri každom reštarte	& Definovanie loopback zdrojového rozhrania pre SNMP	&snmp-server trap-source loopback $<$id$>$\\
Zmeny názvov rozhraní medzi reštartami a nemožnosť monitorovanie pomocou SNMP	&SNMP statické nemenné meno rozhrania aj po reštarte zariadenia	&snmp-server ifindex persist\\
Administrátor nemá povedomie o problémoch na zariadení	&Povolenie logovania protokolom SYSLOG a špecifikovanie IP adresy SYSLOG serveru	&logging on
logging host $<$ip adddress$>$\\
Neprijímanie všetkých dôležitých incidentov na zariadení z protokolu SYSLOG	&Špecifikovanie dôležitosti oznámení SYSLOG na INFORMATIONAL	&logging trap informational\\
SYSLOG zdrojové rozhranie nie je rovnaké pri každom reštarte	& Definovanie loopback zdrojového rozhrania pre SYSLOG	&logging source-interface loopback $<$id$>$\\
Nedostatočné a neštandardné formáty času pri logovacích správach	&Definovanie formátu času pre logovacie a ladiace výstupy	&service timestamp log datetime
service timestamp debug datetime\\
Administrátor nevidí dôležité incidenty pri prihlásení a konfigurovaní cez konzolu	&Vypisovanie SYSLOG správ CRITICAL a dôležitejších do terminálu	&logging console critical\\
Malá vyrovnávacia pamäť pre SYSLOG je dôvodom zahadzovanie správ	&Definovanie veľkosti SYSLOG buffera dôležitosti oznámení na INFORMATIONAL	&logging buffered 64000 6\\
Neprístupný SYSLOG server spôsobuje zahadzovanie dôležitých syslog správ	&Definovanie dočasného úložiska SYSLOG správ v prípade nedostupnosti servera	&logging persistent url flash:/syslog\\
Skenovanie a zistenie informácií o sieti za pomoci protokolu CDP a využitie bezpečnostných chýb	&Zakázanie protokolu CDP	&no CDP run 
no cdp enable\\
Skenovanie a zistenie informácií o sieti za pomoci protokolu LLDP a využitie bezpečnostných chýb	&Zakázanie protokolu LLDP	&no LLDP run 
no lldp receive 
no lldp transmit\\
Nekonzistencia časov v logoch a problém pričlenenia logov k relevantným incidentom	&Nastavenie NTP serveru pre aktuálny čas v logoch	&ntp server $<$ip adddress$>$\\
Pripojenie servera s rovnakou IP adresou, ale falošným časom	&Nastavenie NTP autentizácie	&ntp authenticate
ntp authentication-key 1 md5 $<$password$>$
trusted-key 1\\
NTP zdrojové rozhranie nie je rovnaké pri každom reštarte	& Definovanie loopback zdrojového rozhrania pre NTP	&ntp source loopback $<$id$>$\\
Väčšia bezpečnosť (pub/priv key) NTP a podpora IPv6	&Použitie NTP verzie 4	&ntp server $<$ip adddress$>$ version 4\\
Falošný čas od podvrhnutého NTP zdroja	&Nastavenie NTP peer s inými sieťovými zariadeniami na krížovú validáciu času a záložný zdroj času	&ntp peer $<$ip adddress$>$
ip access-list standard $<$acl name$>$
 remark permit only this IP 
 permit $<$ip adddress$>$ $<$wildcard mask$>$
 remark deny other 
 deny any log-input
ntp access-group serve-only $<$acl name$>$
interface $<$interface name$>$ $<$interface id$>$
 ntp disable\\
Útočník s fyzickým prístupom k zariadeniu alebo portu môže odpočúvať alebo posielať škodlivý obsah	&Explicitne zakázať nepoužívané porty	&interface $<$interface name$>$ $<$interface id$>$
 shutdown\\
Zdrojové rozhranie pre management a control protokoly	&Vytvoriť Loopback rozhranie s IP adresou	&interface loopback $<$id$>$
ip address $<$ip paddress$>$\\
Identifikácia pravidla v ACL	&Popis každého pravidla v ACL pre lepšiu identifikáciu	&ip access-list standard $<$acl name$>$
  remark Deny SNMP from VLAN 20
  deny ip $<$ip address$>$ $<$wildcard mask$>$

ipv6 access-list $<$acl name$>$
  remark Deny SNMP from VLAN 20
  deny $<$ipv6 address$>$ $<$prefix$>$ any\\
Identifikácia rozhrania	&Popis každého rozhrania	&interface $<$interface name$>$ $<$interface id$>$
 description PRODUCTION\_SERVER\\
SSH zdrojové rozhranie nie je rovnaké pri každom reštarte	& Definovanie loopback zdrojového rozhrania pre SSH	&ip ssh source-interface loopback $<$id$>$\\
DOS útok na štandardný SSH port 22	&Špecifikovanie iného portu pre SSH ako štandardného alebo aplikovanie port knocking	&ip ssh port 2223

alebo

ip access-list extended $<$acl name$>$
 remark *** KNOCK ***
 permit udp any any eq 65535 log-input
 remark *** TRUSTED ***
 permit tcp any any established
 remark *** DENIED ***
 deny   tcp any any log input
 remark *** PERMITED ***
 permit ip any any
 
!! WAN Interface !!
 
interface $<$interface name$>$ $<$interface id$>$
 ip access-group $<$acl name$>$
 ipv6 traffic-filter $<$acl name$>$
 
!! KNOCK\_ACL env Variable !!
 
event manager environment $<$env name$>$ $<$acl name$>$
 
!! Port Knocking EEM applet !!
 
event manager applet KNOCK
 event syslog pattern "\%SEC-6-IPACCESSLOGP: list \$KNOCK\_ACL permitted *"
 action 1.0 regexp "[0-9]+$\backslash$.[0-9]+$\backslash$.[0-9]+$\backslash$.[0-9]+" \$\_syslog\_msg ADDR
 action 1.1 regexp "$\backslash$([0-9]+$\backslash$)," "\$\_syslog\_msg" PORT
 action 1.2 regexp "[0-9]+" "\$PORT" PORT 
 action 2.0 syslog msg "Received a knock from \$ADDR on port \$PORT..."
 action 2.1 syslog msg "Adding \$ADDR to the \$KNOCK\_ACL ACL"
 action 3.0 cli command "enable"
 action 3.1 cli command "configure terminal"
 action 3.2 cli command "ip access-list extended \$KNOCK\_ACL"
 action 3.3 cli command "1 permit tcp host \$ADDR any eq 22"
 action 4.0 WAIT 15
 action 5.0 syslog msg "Removing \$ADDR to the \$KNOCK\_ACL ACL"
 action 6.0 cli command "no permit tcp host \$ADDR any eq 22"
 action 6.1 cli command "exit"\\
Nepovolený prístup k manažmentu konfigurácie zariadenia	&Vypnutie odchádzajúcich spojení pre protokoly na manažment zariadení pokiaľ sa nepoužívajú (telnet a pod.)	&line vty $<$num$>$ $<$num$>$
 transport output none
line tty $<$num$>$ $<$num$>$
 transport output none
line con $<$num$>$
 transport output none
line aux $<$num$>$
 transport output none\\
Odpočúvanie konfigurácií zariadení pri zálohe	&Zapnutie zabezpečenej zálohy na server (SFTP, SCP)	&ip scp server enable
copy startup-config scp://$<$username$>$@$<$ip address$>$/backup\\
Vymazanie konfigurácie	&Zapnutie ochrany pred výmazom konfigurácie	&secure boot config\\
Možnosť urobiť diff zmien konfigurácií a jej návrat	&Periodické zálohovanie konfigurácie a logovanie jej zmien	&archive
write-memory
time-period $<$num$>$
log changes
log config
logging enable
logging size $<$num$>$
hidekeys
notify syslog
maximum $<$num$>$\\
DOS útok alebo pokus o prístup k tomu, čo nie je povolené	&Logovanie pravidiel zahodenia paketov v ACL	&ip access-list standard $<$acl name$>$
 deny any log-input

ipv6 access-list $<$acl name$>$
 deny any any log-input\\
Nízky stav voľnej pamäte	&Nastavenie notifikácie pri dochádzaní pamäte	&memory free low-watermark processor $<$threshold$>$
memory free low-watermark io $<$threshold$>$\\
Logovacie správy nemôžu byť zaznamenané kvôli nedostatku pamäte	&Rezervovanie pamäte pre kritické notifikácie pri nedostatku pamäte	&memory reserve critical $<$value$>$ \\
Vysoké zaťaženie CPU	&Nastavenie notifikácie vysokom zaťažení CPU	&snmp-server enable traps cpu threshold
snmp-server host $<$ip adddress$>$ version 3 priv $<$user$>$ cpu
process cpu threshold type $<$type$>$ rising $<$percentage$>$ interval $<$seconds$>$
process cpu statistics limit entry-percentage 

\\
Vysoké zaťaženie zariadenia spôsobilo nemožnosť prihlásenia k nemu	&Rezervovanie pamäte pre protokoly na manažment zariadení pri nedostatku pamäte	&memory reserve console 4096\\
Pretečenie pamäte	&Povoliť mechanizmy na detekciu pretečenia pamäte	&exception memory ignore overflow io
exception memory ignore overflow processor
exception crashinfo maximum files $<$number-of-files$>$\\
Načítanie škodlivej konfigurácie zo siete počas bootovania	&Vypnutie načítania operačného systému alebo konfigurácie zo siete pokiaľ to nie je nutné	&no boot network
no service config\\
Proxy ARP môže viesť k obídeniu PVLAN a rozširuje broadcast doménu	&Vypnutie Proxy ARP	&no proxy-arp\\
DOS útok na stanicu, cez ktorú bola špecifikovaná cesta a teda nemožnosť komunikácie s koncovým bodom. Alebo zosnovanie MITM útoku	&Vypnutie IP source routing	&no ip source-route\\
DOS útok pomocou podvrhnutej IP adresy alebo vzdialený útok na smerovací protokol	&Zapnutie reverse path forwarding strict/loose mode	&ip verify unicast source reachable-via rx\\
Nepoužívané, staré a nezabezpečené služby môžu byť použité na škodlivé účely	&Vypnutie nepoužívaných služieb z bezpečnostných dôvodov a na šetrenie CPU a pamäte 	&no ip bootp server\\
Nepoužívané, staré a nezabezpečené služby môžu byť použité na škodlivé účely	&Vypnutie nepoužívaných služieb z bezpečnostných dôvodov a na šetrenie CPU a pamäte 	&no service pad\\
Nepoužívané, staré a nezabezpečené služby môžu byť použité na škodlivé účely	&Vypnutie nepoužívaných služieb z bezpečnostných dôvodov a na šetrenie CPU a pamäte 	&no ip identd\\
Nepoužívané, staré a nezabezpečené služby môžu byť použité na škodlivé účely	&Vypnutie nepoužívaných služieb z bezpečnostných dôvodov a na šetrenie CPU a pamäte 	&no vstack\\
Nepoužívané, staré a nezabezpečené služby môžu byť použité na škodlivé účely	&Vypnutie nepoužívaných služieb z bezpečnostných dôvodov a na šetrenie CPU a pamäte 	&no ip http server
no ip http secure server\\
Nepoužívané, staré a nezabezpečené služby môžu byť použité na škodlivé účely	&Vypnutie nepoužívaných služieb z bezpečnostných dôvodov a na šetrenie CPU a pamäte 	&no service tcp-small-server\\
Nepoužívané, staré a nezabezpečené služby môžu byť použité na škodlivé účely	&Vypnutie nepoužívaných služieb z bezpečnostných dôvodov a na šetrenie CPU a pamäte 	&no service udp-small-server\\
Nepoužívané, staré a nezabezpečené služby môžu byť použité na škodlivé účely	&Vypnutie nepoužívaných služieb z bezpečnostných dôvodov a na šetrenie CPU a pamäte 	&no service finger\\
Nepoužívané, staré a nezabezpečené služby môžu byť použité na škodlivé účely	&Vypnutie nepoužívaných služieb z bezpečnostných dôvodov a na šetrenie CPU a pamäte 	&int FastEthernet 0/1
 no mop enabled\\
Nepoužívané, staré a nezabezpečené služby môžu byť použité na škodlivé účely	&Vypnutie nepoužívaných služieb z bezpečnostných dôvodov a na šetrenie CPU a pamäte 	&no ip domain lookup\\
Útočník môže zistiť, že IP adresa, na ktorú skúšal ping je nesprávna	&Vypnutie správ ICMP Unreachable	&interface $<$interface name$>$ $<$interface id$>$
 no ip unreachables\\
Útočník môže zistiť masku podsiete pomocou ICMP Mask reply	&Vypnutie správ ICMP Mask reply	&interface $<$interface name$>$ $<$interface id$>$
 no ip mask-reply\\
Umožňuje DOS Smurf útok, mapovanie siete pomocou ping na broadcast adresu vzdialenej siete	&Vypnutie ICMP echo správ na broadcast adresu, vypnutie directed broadcasts	&interface $<$interface name$>$ $<$interface id$>$
 no ip directed-broadcast\\
Útočník môže zistiť smerovacie informácie alebo vyťažiť CPU	&Vypnutie správ ICMP Redirects	&interface $<$interface name$>$ $<$interface id$>$
 no ip redirects\\
Nekonzistencia konfiguračných súborov pri zmenách konfigurácie viac ako jedným administrátorom	&Povoliť súčasne iba jednému administrátorovi vykonávanie zmien v konfigurácii	&configuration mode exclusive auto\\
Problém identifikácie SYSLOG správ s rovnakou časovou značkou	&Pridanie sekvenčného čísla ku každej syslog správe	&service sequence-numbers\\
Nemožnosť prihlásenia pri zaseknutom TCP spojení	&Terminovanie zaseknutého TCP spojenia	&service tcp-keepalives-in
service tcp-keepalives-out\\
Vloženie a manipulácia so smerovacími informáciami	&Autentizácia smerovacích protokolov (nie heslá v otvorenej podobe)	&router bgp 65200
 neighbor 80.80.80.80 password cisco   \\
Vloženie a manipulácia so smerovacími informáciami	&Autentizácia smerovacích protokolov (nie heslá v otvorenej podobe)	&key chain $<$chain name$>$
 key$<$id$>$
  key-string $<$password$>$
interface $<$interface name$>$ $<$interface id$>$
 ip authentication mode eigrp $<$as$>$ md5
 ip authentication keyc-chain eigrp $<$as$>$ $<$chain name$>$
 ipv6 authentication mode eigrp $<$as$>$ md5
 ipv6 authentication keyc-chain eigrp $<$as$>$ $<$chain name$>$
\\
Vloženie a manipulácia so smerovacími informáciami	&Autentizácia smerovacích protokolov (nie heslá v otvorenej podobe)	&key chain $<$chain name$>$
 key$<$id$>$
  key-string $<$password$>$
router ospf $<$process id$>$
 area $<$ared id$>$ authentication message-digest
 area $<$area id$>$ authentication key-chain $<$chain name$>$
ipv6 router ospf $<$process id$>$
 area $<$area id$>$ authentication message-digest
interface $<$interface name$>$ $<$interface id$>$
 ip ospf message-digest-key $<$key id$>$ md5|sha $<$password$>$
 ip ospf authentication message-digest
 ospfv3 authentication md5 0 27576134094768132473302031209727
 no ip ospf authentication-key OPENKEY
  \\
Vloženie a manipulácia so smerovacími informáciami	&Autentizácia smerovacích protokolov (nie heslá v otvorenej podobe)	&key chain $<$chain name$>$
 key $<$id$>$
  key-string $<$password$>$
interface $<$interface name$>$ $<$interface id$>$
 ip rip authentication key-chain $<$chain name$>$
 ip rip authentication mode md5\\
OSPF virtuálne linky degradujú výkon	&Vypnutie virtuálnych liniek pre OSPF	&no area $<$area id$>$ virtual-link $<$ip address$>$\\
Koncové zariadenie, užívateľ a útočník môžu vidieť smerovacie správy a topológiu siete alebo pripojenie škodlivého zariadenia, ktoré vysielať a prijímať smerovacie správy	&Špecifikovanie rozhraní, ktoré nebudú prijímať smerovacie informácie	&router rip
 passive-interface default
 no passive-interface $<$interface name$>$ $<$interface id$>$

router ospf $<$process$>$
 passive-interface default
 no passive-interface $<$interface name$>$ $<$interface id$>$

router eigrp $<$as$>$
 passive-interface default
 no passive-interface $<$interface name$>$ $<$interface id$>$\\
Nemožnosť sprevádzkovať procesy smerovacích protokolov v určitých prípadoch pri použití IPv6	&Špecifikovanie identifikátorov smerovacích protokolov pre každý router (router ID)	&router ospf 1|eigrp1|bgp
  router-id $<$ip-address$>$ 
ipv6 router ospf  $<$process-id$>$
  router-id $<$ip-address$>$ \\
Vysledovateľnosť nefunkčnosti smerovacieho protokolu a nesprávneho nastavenia	&Zaznamenanie zmeny v logu pri zmenách v smerovaní	&router eigrp $<$as$>$
 (eigrp) log-neighbor-changes\\
Vysledovateľnosť nefunkčnosti smerovacieho protokolu a nesprávneho nastavenia	&Zaznamenanie zmeny v logu pri zmenách v smerovaní	&router ospf $<$process$>$
 log-neighbor-changes\\
Vysledovateľnosť nefunkčnosti smerovacieho protokolu a nesprávneho nastavenia	&Zaznamenanie zmeny v logu pri zmenách v smerovaní	&router bgp $<$as$>$
 log-neighbor-changes\\
Škodlivé vloženie smerovacích informácií informácií, vzdialený útok	&TTL security	&hostname $<$hostname$>$\\
Nesprávne smerovanie kvôli sumarizácií	&Vypnutie automatickej sumarizácie smerovacích protokolov	&router rip
 no auto-summary

router eigrp $<$as$>$
 no auto-summary

router bgp $<$as$>$
 no auto-summary\\
Pakety budú spracovávané v CPU, ktoré môže byť preťažené a môže byť zmenené smerovanie na obídenie bezpečnostnej kontroly	&Zahadzovanie IPv4 paketov s rozšírenou hlavičkou (IP Options filtering)	&ip options drop\\
Odpočúvanie komunikácie  cez nezabezpečené tunely	&Vypnúť tunely ktoré nie sú zabezpečené alebo zabezpečiť tunely	&crypto isakmp policy $<$policy id$>$
 encryption aes
 authentication pre-shared
 group $<$group id$>$
crypto isakmp key $<$key$>$ address $<$ip address$>$ 
crypto ipsec transform-set $<$set name$>$ esp-aes esp-sha-hmac
 crypto map $<$map name$>$ 10 ipsec-isakmp
  set peer $<$peer ip$>$
  set transform $<$set name$>$ 
  match address $<$acl name$>$
ip access-list extended $<$acl name$>$
  permit ip $<$source ip$>$ $<$wildcard mask$>$ $<$destination ip$>$ $<$ wildcard mask$>$
interface $<$interface name$>$
 crypto map $<$map name$>$

interface tunnel $<$number$>$
 ip address $<$ip address$>$ $<$mask$>$
 tunnel source $<$ip address$>$ $<$mask$>$
 tunnel destination $<$ip address$>$ $<$mask$>$\\
Môže byť zneužité odpočúvanie pokiaľ sa používa monitorovanie prevádzky a monitorovanie prevádzky kvôli legislatívnym potrebám	&Monitorovanie výkonnosti siete a zber sieťového prenosu kvôli legislatívnym potrebám	&ip flow-export version 9
ip flow-export destination $<$ip address$>$ $<$port$>$
interface $<$interface name$>$ $<$interface id$>$
 ip flow ingress
 ip flow egress\\
Môže byť zneužité odpočúvanie pokiaľ sa používa monitorovanie prevádzky a monitorovanie prevádzky kvôli legislatívnym potrebám	&Monitorovanie výkonnosti siete a zber sieťového prenosu kvôli legislatívnym potrebám	&monitor session $<$session id$>$ source $<$interface name$>$ $<$interface id$>$ 
monitor session $<$session id$>$ destination $<$interface name$>$ $<$interface id$>$\\
IP spoofing	&Špecifikácia ACL na zakázanie a logovanie privátnych a špeciálnych IP adries z RFC 6890, RFC 8190	&ip access-list standard $<$acl name$>$
 remark BLOCK\_ADDRESSES RFC 1918
 deny $<$ip address$>$ $<$wildcard mask$>$ log-input

interface $<$interface name$>$ $<$interface id$>$
 ip access-group $<$acl name$>$ in\\
IP spoofing	&Špecifikácia ACL na zakázanie a logovanie špeciálnych IPv6 adries z RFC 6890, RFC 8190, RFC 5156	&
ipv6 access-list $<$acl name$>$
 remark BLOCK\_ADDRESSES RFC 5156
 deny $<$ipv6 address$>$ $<$prefix$>$ any log-input

interface $<$interface name$>$ $<$interface id$>$
 ipv6 traffic-filter $<$acl name$>$ in\\
Rogue root bridge 	&Rogue root bridge protection (root guard)	&interface $<$interface name$>$ $<$interface id$>$
 spanning-tree rootguard
 spanning-tree guard root\\
Pripojenie prepínaču na koncový prístupový port	&BPDU protection (BPDU guard)	&spanning-tree portfast bpduguard default

alebo

interface $<$interface name$>$ $<$interface id$>$
  spanning-tree bpduguard enable

vyhnúť sa

spanning-tree portfast bpdufilter enable

interface $<$interface name$>$ $<$interface id$>$
  spanning-tree bpdufilter enable

\\
Rýchlosť konvergencie	&Prístupové porty by sa nemali podieľať na STP procese	&spanning-tree portfast default

alebo

interface $<$interface name$>$ $<$interface id$>$
 spanning-tree portfast\\
Jednosmerná komunikácia medzi prepínačmi môže viesť k topológii so slučkami	& Špeciálne konfigurácie zaisťujúce bezslučkovú topológiu pomocou STP keď nastane jednosmerná komunikácia (Loop Guard)	&spanning-tree loopguard default

alebo 

interface $<$interface name$>$ $<$interface id$>$
 spanning-tree guard loop\\
Nemožnosť identifikácie účelu VLAN	&Pridanie mena k VLAN	&vlan $<$id$>$
 name $<$name$>$\\
Špeciálna VLAN pre manažment na obmedzenie prístupu iba pre administrátorov	&Vytvorenie separátnej VLAN pre manažment	&vlan $<$id$>$  
name MANAGEMENT\_VLAN\\
Útočníkovi s fyzickým prístupom k portu môže byť pridelený prístup do časti siete, ktorá zodpovedá príslušnej VLAN 	&Vytvorenie špeciálnej black hole VLAN pre nevyužité porty	&vlan $<$id$>$  
name BLACKHOLE\_VLAN\\
Predvolenej VLAN je povolené prepnuté na akýkoľvek port, VLAN hopping, double tagging	&Odobrať všetky porty z predvolenej VLAN	&interface $<$interface name$>$ $<$interface id$>$
 switchport mode access
 switchport access vlan $<$id$>$\\
Predvolenej VLAN je povolené byť prepnutá na akýkoľvek port, VLAN hopping, double tagging	&Vytvorenie natívnej VLAN rozdielnej ako predvolená, priradenie k trunk portu a povolenie iba potrebných portov	&vlan $<$id$>$  
name NATIVE\_VLAN

interface $<$interface name$>$ $<$interface id$>$
 switchport mode trunk
 switchport trunk native vlan $<$id$>$
 switchport trunk allowed vlan $<$id$>$\\
DTP útok, Switch spoofing útok	&Vypnutie dynamického trunkovacieho protokolu a explicitne určiť porty ako prístupové a trunk	&
interface $<$interface name$>$ $<$interface id$>$
 switchport mode trunk
 no switchport mode dynamic desirable
 no switchport mode dynamic auto
 switchport nonegotiate
\\
MAC Spoofing, MAC Flooding 	&Definovanie maximálne 1 MAC adresy na port, priradenie MAC adresy na port	&
interface $<$interface name$>$ $<$interface id$>$
 switchport port-security maximum 1
 switchport port-security mac-address sticky

 alebo

 switchport port-security mac-address static $<$mac address$>$
 switchport port-security\\
MAC Spoofing, MAC Flooding 	&Nastavenie režimu narušenia, ktorý vypne port alebo informuje správcu o pripojení nepovoleného zariadenia	&interface $<$interface name$>$ $<$interface id$>$
 switchport port-security violation mode shutdown
 switchport port-security violation mode restrict
 no switchport port-security violation mode protect\\
Nový prepínač s vyšším číslom revízie, ale s nesprávnou VLAN databázou môže šíriť falošné VLAN identifikátory a spôsobiť nefunkčnosť siete, veľa možných VTP útokov kvôli zraniteľnostiam 	&Vypnutie MVRP. MRP, GARP, VTP, GVRP po úspešnej propagácií VLAN	&vtp mode transparent

alebo

vtp off\\
VTP musí byť používané	&Uprednostniť VTP verzie 3, špecifikovať skryté heslo a zapnúť VTP prunning pokiaľ musí byť VTP zapnuté	&vtp version 3
vtp password $<$password$>$ hidden
vtp prunning\\
Vysoké zaťaženie linky	&Poslanie notifikácie pri prekročení prahovej hodnoty zaťaženia linky	&storm-control unicast level $<$top level$>$ $<$down level$>$
storm-control broadcast level $<$top level$>$ $<$down level$>$
storm-control multicast level $<$top level$>$ $<$down level$>$
storm-control action trap\\
Využívanie siete nepovolenými používateľmi	&Zapnutie 802.1x 	&dot1x system-auth-control
identity profile default
interface $<$interface name$>$ $<$interface id$>$
  dot1x port-control auto
  
  alebo

  access-session port-control auto

  alebo

  authentication port-control auto
  dot1x pae authenticator|supplicant 

  no dot1x port-control force-authorized

  alebo
  
  no access-session port-control force-authorized

  alebo

  no authentication port-control force-authorized
\\
Útok hrubou silou hádaním prístupových údajov pre 802.1x 	&Limitovanie maximálneho počtu neúspešných pokusov o autentizáciu 802.1x	&dot1x auth-fail max-attempts $<$number$>$\\
IPv6 ND Spoofing	&IPv6 ND Inspection	&ipv6 nd inspection policy $<$policy name$>$
 drop unsecure
 device-role monitor
 tracking disable stale-lifetime infinite
 trusted-port
interface $<$interface name$>$ $<$interface id$>$
 ipv6 nd inspection attach-policy  $<$policy name$>$\\
Rogue RA
RA Flood
Route Information Option injection
RA RouterLifeTime=0
	&RA Guard	&ipv6 nd raguard policy $<$polic name$>$
 device-role host|router
 hop-limit maximu $<$number$>$
 managed-config-flag on|off
 other-config-flag on|off
 match ipv6 access-list $<$acl name$>$
 match ra prefix-list $<$prefix list name$>$
 trusted-port
interface $<$interface name$>$ $<$interface id$>$
 ipv6 nd raguard attach-policy $<$policy name$>$\\
DHCP spoofing	&DHCP snooping, IPv6 Snooping, DHCPv6 Guard	&ip dhcp snooping
ip dhcp snooping vlan $<$vlan-id$>$ 
interface $<$interface name$>$ $<$interface id$>$
 ip dhcp snooping trust
 no ip dhcp snooping trust\\
DHCP spoofing	&DHCP snooping, IPv6 Snooping, DHCPv6 Guard	&ipv6 snooping policy $<$policy name$>$
  ipv6 snooping attach-policy $<$policy name$>$
  prefix-glean\\
DHCP spoofing	&DHCP snooping, IPv6 Snooping, DHCPv6 Guard	&ipv6 access-list $<$acl name$>$
 permit host $<$ipv6 address$>$ any
ipv6 prefix-list $<$prefix list name$>$ permit $<$ipv6 address$>$  le 128
ipv6 dhcp guard policy $<$policy name$>$
 device-role server|client
 match server access-list $<$acl name$>$
 match reply prefix-list $<$prefix list name$>$
 trusted-port
interface $<$interface name$>$ $<$interface id$>$
 ipv6 dhcp guard attach-policy $<$policy name$>$\\
Príliš veľa DHCP paketov, zaplavenie DHCP paketmi	&Obmedziť počet DHCP paketov na nedôveryhodných rozhraniach	&ip dhcp snooping limit rate 100 \\
ARP Spoofing	&Dynamic ARP Inspection	&ip arp inspection vlan $<$vlan id$>$ 
ip arp inspection validate src-mac dst-mac ip

na uplink

ip arp inspection trust


\\
IP spoofing	&IPv4/IPv6 Source Guard	&ip verify source port-security
ip verify source
ip verify source vlan dhcp-snooping
ip verify source vlan dhcp-snooping port-security

ipv6 source-guard policy $<$policy name$>$
 permit link-local
 deny global-autoconf
 trusted
interface $<$interface name$>$ $<$interface id$>$
 ipv6 source-guard attach-policy $<$policy name$>$\\
IPv6 Next Header  a IPv6 Fragmentation útok	&ACL blokujúce nerozpoznateľné rozšírené hlavičky	&ipv6 access-list $<$acl name$>$
 remark deny undetermined next headers
 deny any any undetermined-transport log-input\\
Mapovanie siete pomocou pingu na multicast adresu všetkých uzlov a MLD/IGMP Query Overload a Smurf útok	& ACL blokujúce ICMP echo request na multicast adresu všetkých uzlov a MLD/IGMP Query na prístupových portoch	&ip access-list extended $<$acl name$>$
 remark deny all node ipv4 address
 deny icmp any host 224.0.0.1 echo log-input

ipv6 access-list $<$acl name$>$
 remark deny all node ipv6 address
 deny icmp any host ff02::1 echo-request log-input
 remark deny mld query
 deny icmp any any mld-query\\
Mobilné zariadenia pripojené bezdrôtovo spotrebovávajú veľa energie kvôli častým RA správam	&RA Throttling	&ipv6 nd ra-throttle policy $<$policy name$>$
 allow at-least $<$value$>$ at-most $<$value$>$
 interval-option inherit
 max-through $<$value$>$
 media-type wired|access-point|wire|wifi
 throttle-period $<$value$>$
vlan configuration $<$vlan id$>$
 ipv6 nd ra-throttle attach-policy $<$policy name$>$

alebo 

interface $<$interface name$>$ $<$interface id$>$
 ipv6 nd ra-throttle policy $<$policy name$>$\\
Zlyhanie zariadenia alebo linky môže viest k nefunkčnosti siete 	&Povolenie FHRP s autentizáciou a aktuálnou verziou	&key chain $<$key chain$>$
 key $<$id$>$
  key-string $<$key string$>$
track $<$value$>$ interface $<$interface name$>$ $<$interface id$>$ line-protocol
fhrp version vrrp 3
interface $<$interface name$>$ $<$interface id$>$
 vrrp $<$group id$>$ ip $<$ip  address$>$
 vrrp priority $<$value$>$
 vrrp $<$group id$>$ track $<$value$>$ decrement $<$value$>$
 vrrp $<$group id$>$ authentication md5 key-string $<$key$>$

 alerbo

 $<$group id$>$ authentication md5 key-chain $<$key chain$>$\\
Zlyhanie zariadenia alebo linky môže viest k nefunkčnosti siete 	&Povolenie FHRP s autentizáciou a aktuálnou verziou	&key chain $<$key chain$>$
 key $<$id$>$
  key-string $<$key string$>$
interface $<$interface name$>$ $<$interface id$>$
 standby $<$group id$>$ ip $<$ip address$>$
 standby $<$group id$>$ priority $<$value$>$
 standby $<$group id$>$ preempt
 standby version 2
 standby $<$group id$>$ authentication md5 key-string $<$key$>$
 
 alebo
 
 standby $<$group id$>$ authentication md5 key-chain $<$key chain$>$

track $<$id$>$  interface $<$interface name$>$ $<$interface id$>$
interface $<$interface name$>$ $<$interface id$>$
  standby $<$group id$>$ track $<$id$>$ decrement $<$value$>$
 \\
Zlyhanie zariadenia alebo linky môže viest k nefunkčnosti siete 	&Povolenie FHRP s autentizáciou a aktuálnou verziou	&key chain $<$key chain$>$
 key $<$id$>$
  key-string $<$key string$>$
track $<$value$>$ interface $<$interface name$>$ $<$interface id$>$ line-protocol
interface $<$interface name$>$ $<$interface id$>$
 glbp $<$group id$>$ ip $<$ip  address$>$
 glbp $<$group id$>$ priority $<$value$>$
 glbp $<$group id$>$ preempt
 glbp $<$group id$>$ weighting $<$value$>$ lower $<$value$>$ upper $<$value$>$ 
 glbp $<$group id$>$ weighting track $<$value$>$ decrement $<$value$>$
 glbp $<$group id$>$ authentication md5 key-string $<$key$>$

 or

 glbp $<$group id$>$ authentication md5 key-chain $<$key chain$>$\\
Vyčerpanie cache susedov	&Statický záznam pre kritické zariadenia (servery) spájajúce IP a MAC adresu a VLAN
	&ipv6 neighbor $<$ipv6 address$>$ vlan $<$vlan id$>$ $<$mac address$>$\\
Vyčerpanie cache susedov	&Na zabránenie vzdialeného útoku na cache susedov cez internet je potreba nastaviť ACL, kde povoľujeme iba komunikáciu s cieľovými IPv6 adresami, ktoré sa nachádzajú v našej sieti	&ipv6 access-list $<$acl name$>$
 remark permit only this ip  
 permit any $<$ipv6$>$/$<$prefix$>$
 remark deny other
 deny ipv6 any any 

interface $<$interface name$>$ $<$interface id$>$
 ipv6 traffic-filter $<$acl name$>$ in\\
Vyčerpanie cache susedov	&IP destination Guard (First Hop Security)


	&ipv6 destination-guard policy $<$policy name$>$
  enforcement always
interface $<$interface name$>$ $<$interface id$>$
 ipv6 destination-guard attach-policy $<$policy name$>$

\\
Vyčerpanie cache susedov	&Limitovanie počtu IPv6 adries v cache susedov	&interface $<$interface name$>$ $<$interface id$>$
 ipv6 neighbors max-learning-num $<$num$>$\\
Vyčerpanie cache susedov	&Limitovanie času IPv6 adresy v cache susedov	&ipv6 nd cache expire $<$time in seconds$>$\\
SYN Flood 	&Nastavenie zachytávanie firewallom pre útok flagu SYN	&ip tcp intercept list $<$acl name$>$
ip tcp intercept mode intercept|watch
ip tcp intercept drop-mode oldest|random
ip tcp intercept watch-timeout $<$seconds$>$
ip tcp intercept finrst-timeout $<$second$>$
ip tcp intercept connection-timeout $<$seconds$>$
ip tcp intercept max-incomplete high|low $<$number$>$
ip tcp intercept one-minute high|low $<$number$>$\\
Komplexné bezpečnostné hrozby a narušenie bezpečnosti	&Nastavenie IDS/IPS	&ip ips sdf location $<$signature location$>$
ip ips fail  open|close
ip ips $<$signature name$>$ list $<$alc name$>$
ip ips $<$signature name$>$ in|out\\
